\documentclass[pdftex,12pt,a4paper]{article}

\input{/home/boris/science/tex_general/title_bor_utf8}


\def \useR{$[$R$]$ }

%% эконометрические сокращения
\def \hb{\hat{\beta}}
\def \b{\beta}
\def \hs{\hat{s}}
\def \hy{\hat{y}}
\def \hY{\hat{Y}}
\def \he{\hat{\varepsilon}}
\def \v1{\vec{1}}
\def \e{\varepsilon}
\def \z{z}
\def \hVar{\widehat{\Var}}
\def \hCorr{\widehat{\Corr}}
\def \hCov{\widehat{\Cov}}
\def \cN{\mathcal{N}}
\renewcommand{\P}{\mathbb{P}}

%% лаг
\renewcommand{\L}{\mathrm{L}}



\begin{document}
\parindent=0 pt % отступ равен 0

\begin{center}
Контрольная работа, ML
\end{center}

\RedRoseLine

\begin{enumerate}
\item Купив пачку мэндэмс я насчитал в ней 1 жёлтую, 7 зелёных, 4 оранжевых, 3 коричневых, 2 синих и 1 красную мэндэмсину. С помощью теста отношения правдоподобия проверьте гипотезу, что мэндэмсины всех цветов встречаются равновероятно. 
\item \useR Фактическое распределение часовой и десятиминутной скорости ветра хорошо приближается распределением Вейбулла. Случайная величина имеет распределение Вейбулла, если её функция плотности при $x>0$ имеет вид
\[
f(x)=\frac{1}{\lambda^k}kx^{k-1}\exp(-x^k/\lambda^k)
\]
\begin{enumerate}
\item Найдите функцию распределения $F(x)$
\item Выразите медиану распределение Вейбулла, $m$, через параметры $k$ и $\lambda$
\item Оцените параметры $k$ и $\lambda$ методом максимального правдоподобия
\item Постройте 95\%-ые доверительные интервалы для $k$ и $\lambda$
\item Выпишите функцию плотности распределения Вейбулла через $m$ и $k$
\item Проверьте гипотезу о том, что медиана равна 1 м/сек с помощью трёх тестов
\end{enumerate}
Часовые данные я не нашёл, нашёл дневные. Данные по среднедневной скорости ветра содержатся в \verb|weather_nov_2012_moskow.csv| в стобике \verb|wind|. Данные взяты с сайта \url{http://www.atlas-yakutia.ru/weather/climate_russia-I.html}.

\end{enumerate}


\end{document}