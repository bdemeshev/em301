\documentclass[pdftex,12pt,a4paper]{article}

\input{/home/boris/science/tex_general/title_bor_utf8}


\def \useR{$[$R$]$ }

%% эконометрические сокращения
\def \hb{\hat{\beta}}
\def \b{\beta}
\def \hs{\hat{s}}
\def \hy{\hat{y}}
\def \hY{\hat{Y}}
\def \he{\hat{\varepsilon}}
\def \v1{\vec{1}}
\def \e{\varepsilon}
\def \z{z}
\def \hVar{\widehat{\Var}}
\def \hCorr{\widehat{\Corr}}
\def \hCov{\widehat{\Cov}}
\def \cN{\mathcal{N}}
\renewcommand{\P}{\mathbb{P}}

%% лаг
\renewcommand{\L}{\mathrm{L}}



\title{Домашка <<Титаник>>}
\date{}



\begin{document}

\pagestyle{empty}
\begin{center}
Домашка <<Титаник>>
\end{center}
%\parindent=0 pt % отступ равен 0

\vspace{15pt}

Нужно зарегистрироваться на сайте \url{www.kaggle.com} и принять участие в конкурсе <<Titanic: Machine Learning from Disaster>>. Крайний срок сдачи отчёта: в ночь с 14 на 15 апреля 2013 года.

\vspace{15pt}
\WhiteRoseLine
\vspace{15pt}

\begin{enumerate}
\item Домашнее задание можно делать в одиночку или группой из двух человек.
\item А можно всё-таки группой из трёх человек? Нет :)
\item Письменный отчёт  должен содержать как-минимум:
\begin{enumerate}
\item Логин группы
\item Графический анализ имеющихся данных
\item Результаты оценивания logit и probit моделей
\item Графический анализ logit и probit моделей
\item <<Если бы я был пассажиром Титаника, то я спасся бы с вероятностью\ldots>>. 

С помощью logit и probit моделей необходимо построить 95\%-ый доверительный интервал для вероятности спасения каждого из участников группы, сдающей домашку. Пол и возраст взять фактические, а остальные объясняющие переменные --- по своему желанию.
\end{enumerate}
\end{enumerate}


\vspace{15pt}
\RedRoseLine
\vspace{15pt}


\begin{enumerate}
\item Домашнее задание можно делать только в одиночку :)
\item Нет, нельзя :)
\item Письменный отчёт  должен содержать как-минимум:
\begin{enumerate}
\item Логин 
\item Графический анализ имеющихся данных
\item Результаты оценивания logit и probit моделей
\item Прогнозирование с использованием Random Forest
\item Прогнозирование с использованием метода опорных векторов (SVM)
\item Графический анализ оценённых моделей
\item <<Если бы я был пассажиром Титаника, то я спасся бы с вероятностью\ldots>>. 

С помощью логит и пробит моделей необходимо построить 95\%-ый доверительный интервал для своей вероятности спасения. Для Random Forest требуется только точечная оценка вероятности спасения. Пол и возраст взять фактические, а остальные объясняющие переменные --- по своему желанию.
\end{enumerate}
\end{enumerate}



 



\end{document}