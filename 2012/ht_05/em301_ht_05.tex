\documentclass[pdftex,12pt,a4paper]{article}

\input{/home/boris/science/tex_general/title_bor_utf8}

% чисто эконометрические сокращения:
\def \hb{\hat{\beta}}
\def \hy{\hat{y}}
\def \hY{\hat{Y}}
\def \he{\hat{\varepsilon}}
\def \v1{\vec{1}}
\def \e{\varepsilon}
\def \hVar{\widehat{\Var}}
\def \hCorr{\widehat{\Corr}}
\def \hCov{\widehat{\Cov}}
\def \hs2{\hat{\sigma}^2}




\begin{document}
\parindent=0 pt % отступ равен 0

Домашнее задание \No $(n+1)$ по эконометрике-1.

\textbf{Задача 1}. <<CAPM>>

Оценим модель CAPM по реальным данным:
\begin{enumerate}
\item Коротко сформулируйте теоретические положения модели CAPM. За корректное отделение выводов от предпосылок --- дополнительный бонус.
\item Соберите реальные данные по трём показателям: $R_i$ --- доходность некоей акции за $i$-ый период, $R_{m,i}$ --- рыночная доходность за $i$-ый период, $R_{f,i}$ --- безрисковая доходность за $i$-ый период. Статья \href{http://quantile.ru/06/06-AT.pdf}{quantile.ru/06/06-AT.pdf} в помощь.
\item Представьте информацию графически
\item С помощью МНК оцените модель без константы, $R_i-R_{f,i}=\beta (R_{m,i}-R_{f,i})+\e_i$. Предположим, что $\e_i \sim N(0,\sigma_{\e}^2)$.
\item Прокомментируйте результаты оценивания. В частности, проверьте гипотезы о значимости коэффициента и регрессии в целом.
\item С помощью МНК оцените модель с константой, $R_i-R_{f,i}=\beta_1 + \beta_2 (R_{m,i}-R_{f,i})+\e_i$. Предположим, что $\e_i \sim N(0,\sigma_{\e}^2)$.
\item Прокомментируйте результаты оценивания. В частности, проверьте гипотезы о значимости коэффициентов и регрессии в целом. 
\item Труднее всего измерить безрисковую ставку процента. Поэтому предположим, что имеющиеся у нас наблюдения --- это безрисковая ставка, измеренная с ошибкой. Т.е. имеющиеся у нас наблюдения $R_{f,i}$ представимы в виде $R_{f,i}=R_{f,i}^{true}+u_i$, где $u_i \sim N(0,\sigma^2_u)$. Величина $R_{f,i}^{true}$ ненаблюдаема, но именно она входит в модель CAPM. Получается, что оцениваемая модель имеет вид $R_i-R_{f,i}^{true}=\beta (R_{m,i}-R_{f,i}^{true})+\e_i$.
\begin{enumerate}
\item Выпишите функцию правдоподобия для оценки данной модели
\item Найдите оценки $\hb$, $\hs2_{u}$, $\hs2_{\e}$
\item Постройте 95\%-ые доверительные интервалы
\item Проделайте аналогичные действия для модели с константой
\item Сделайте выводы
\end{enumerate}

\end{enumerate}



\textbf{Задача 2}. <<Циф\'{и}рьки на мониторе>>

При входе на каждую станцию метро есть турникеты. Рядом с турникетами в будке сидит бабушка божий одуванчик. В будке у бабушки висит монитор. На этом мониторе --- прямоугольники с циф\'{и}рьками. 
\begin{enumerate}
\item Понаблюдав за изменением циф\'{и}рек, догадайтесь, что они означают.
\item Вечером какого-нибудь буднего дня запишите все цифирьки с монитора на своей родной станции метро.
\item Представьте информацию графически
\item Будем моделировать величину $i$-ой циф\'{и}рьки пуассоновским распределением с математическим ожиданием $\lambda_i$. Предположим также, что $\lambda_i=\beta_1+\beta_2 \cdot i$, где $i$ --- номер турникета считая от будки с бабушкой. 
\begin{enumerate}
\item Выпишете функцию правдоподобия
\item Оцените параметры $\beta_1$ и $\beta_2$ 
\item Оцените ковариационную матрицу оценок $\hb_1$ и $\hb_2$
\item Постройте 95\%-ые асимптотические доверительные интервалы для параметров
\item Проверьте гипотезу о том, что $\beta_2=0$. Альтернативную гипотезу сформулируйте самостоятельно.
\end{enumerate}
\end{enumerate}



PS. Своё смелое творчество в задачах поощряется!

\end{document}

