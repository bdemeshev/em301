\documentclass[pdftex,12pt,a4paper]{article}

\input{/home/boris/science/tex_general/title_bor_utf8}


\begin{document}
\parindent=0 pt % отступ равен 0

Большой Устный ЗАчёт по эконометрике


\begin{enumerate}

\item Метод Наименьших Квадратов. 

Задача Димы Борзых

МНК-картинка

\item Теорема Гаусса-Маркова
\begin{enumerate}
\item Формулировка с детерминистическими регрессорами
\item Доказательство с детерминистическими регрессорами
\item Формулировки со стохастическими регрессорами
\item Что даёт дополнительное предположение о нормальности $\varepsilon$?
\end{enumerate}

\item Проверка гипотез о линейных ограничениях 
\begin{enumerate}
\item Проверка гипотезы о значимости коэффициента
\item Проверка гипотезы о значимости регрессии в целом
\item Проверка гипотезы об одном линейном соотношении с помощью ковариационной матрицы
\item Ограниченная и неограниченная модель
\item Тест Чоу на стабильность коэффициентов
\item Тест Чоу на прогнозную силу (?)
\end{enumerate}

\item Метод максимального правдоподобия

\begin{enumerate}
\item Свойства оценок
\item Два способа получения оценки дисперсии
\item Три теста (LM, Wald, LR)
\item Выписать функцию ML для обычной регрессии
\item для AR(1) процесса
\item для MA(1) процесса
\item для логит модели
\item для пробит модели
\item для модели с заданным видом гетероскедастичности
\end{enumerate}

\item Мультиколлинеарность
\begin{enumerate}
\item Определение, последствия
\item Величины, измеряющие силу мультиколлинеарность
\item Методы борьбы
\end{enumerate}


\item Гетероскедастичность
\begin{enumerate}
\item Определение, последствия
\item Тесты, график
\item Стьюдентизированные остатки
\item HC оценки ковариации
\item GLS и FGLS
\end{enumerate}

\item Временные ряды
\begin{enumerate}
\item Стационарный временной ряд
\item ACF, PACF
\item Модель ARMA
\item Модель GARCH
\end{enumerate}


\item Логит и пробит
\begin{enumerate}
\item Предельные эффекты
\item Чувствительность, специфичность
\item Кривая ROC
\end{enumerate}



\item Альтернативные методы прогнозирования
\begin{enumerate}
\item Метод опорных векторов
\item Классификационные деревья и случайный лес
\item Ridge regression
\item LASSO
\item Квантильная регрессия
\item Байесовская регрессия
\end{enumerate}


\end{enumerate}




\end{document}