\documentclass[pdftex,12pt,a4paper]{article}

\input{/home/boris/science/tex_general/title_bor_utf8}


\begin{document}
\parindent=0 pt % отступ равен 0

Большой Устный ЗАчёт по эконометрике


\begin{enumerate}

\item Метод Наименьших Квадратов. 

\begin{enumerate}
\item МНК-картинка
\item Задача 2.1. в обязательном порядке (!)
  \end{enumerate}

\item Теорема Гаусса-Маркова
\begin{enumerate}
\item Формулировка с детерминистическими регрессорами
\item Доказательство с детерминистическими регрессорами
\item Формулировки со стохастическими регрессорами
\item Что даёт дополнительное предположение о нормальности $\varepsilon$?
\end{enumerate}

\item Проверка гипотез о линейных ограничениях 
\begin{enumerate}
\item Проверка гипотезы о значимости коэффициента
\item Проверка гипотезы о значимости регрессии в целом
\item Проверка гипотезы об одном линейном соотношении с помощью ковариационной матрицы
\item Ограниченная и неограниченная модель
\item Тест Чоу на стабильность коэффициентов
\item Тест Чоу на прогнозную силу 
\end{enumerate}

\item Метод максимального правдоподобия

\begin{enumerate}
\item Свойства оценок
\item Два способа получения оценки дисперсии
\item Три теста (LM, Wald, LR)
\item Выписать функцию ML для обычной регрессии
\item для AR(1) процесса
\item для MA(1) процесса
\item для логит модели
\item для пробит модели
\item для модели с заданным видом гетероскедастичности
\end{enumerate}

\item Мультиколлинеарность
\begin{enumerate}
\item Определение, последствия
\item Величины, измеряющие силу мультиколлинеарности
\item Методы борьбы
\item Сюда же: метод главных компонент, хотя он используется и для других целей
\end{enumerate}


\item Гетероскедастичность
\begin{enumerate}
\item Определение, последствия
\item Тесты, график
\item Стьюдентизированные остатки
\item HC оценки ковариации
\item GLS и FGLS
\end{enumerate}

\item Временные ряды
\begin{enumerate}
\item Стационарный временной ряд
\item ACF, PACF
\item Модель ARMA
\item Модель GARCH (не будет, не успели)
\end{enumerate}


\item Логит и пробит
\begin{enumerate}
\item Описание моделей
\item Предельные эффекты
\item Чувствительность, специфичность
\item Кривая ROC
\end{enumerate}



\item Альтернативные методы. Уметь объяснить суть метода. Уметь реализовать его в R. Если не считать упоминания Ridge regression, эти методы официально не входят в программу. Поэтому наивысшую оценку за Большой Устный Зачет можно получить не зная их. Но зная их можно подстраховать себя от ошибки на остальных задачах.
\begin{enumerate}
\item Метод опорных векторов
\item Классификационные деревья и случайный лес
\item Ridge regression
\item LASSO
\item Квантильная регрессия
\item Байесовская регрессия (не будет, не успели)
\end{enumerate}


\item В R нужно в течении 5 минут уметь выполнить любые 3 пункта по желанию спрашивающего:
\begin{enumerate}
\item Загрузить данные из \verb|.csv| файла в R
\item Посчитать описательные статистики (среднее, мода, медиана и т.д.)
\item Построить описательные графики: график временного ряда, гистограмму или оцененную функцию плотности, диаграмму рассеяния, мозаичный график для качественных переменных, violin plot \url{}
\item Оценить линейную регрессию с помощью МНК. Провести диагностику на что-нибудь (гетероскедастичность, автокорреляцию, мультиколлинеарность).
\item Оценить logit, probit модели, посчитать предельные эффекты
\item Оценить ARMA модель
\item Выделить главные компоненты
\end{enumerate}

При этом можно использовать Интернет для поиска информации, но не для общения. Например, погуглить <<estimate logit in R>> можно, а сделать <<звонок другу>> нельзя. 

\end{enumerate}




\end{document}