\documentclass[pdftex,12pt,a4paper]{article}

\input{/home/boris/science/tex_general/title_bor_utf8}


\begin{document}
\parindent=0 pt % отступ равен 0

{\Large Пролетарий, на коня! }
\begin{center}
\includegraphics[height=3in]{proletarii.jpg}
\end{center}
\begin{enumerate}
\item Найдите длины векторов $a=(1,2,3)$ и $b=(1,0,-1)$ и косинус угла между ними.
\item Сформулируйте теорему о трёх перпендикулярах.
\item Сформулируйте и докажите теорему Пифагора.
\item Для матрицы 

$A=\left(%
\begin{array}{ccc}
  2 & 3 & 0 \\
  3 & 10 & 0 \\
  0 & 0 & -1 \\
\end{array}%
\right)$ \\

\begin{enumerate}
\item Найдите собственные числа и собственные векторы матрицы.
\item Найдите обратную матрицу, $A^{-1}$, ее собственные векторы и собственные числа.
\item Представьте матрицу $A$ в виде $A=CDC^{-1}$, где $D$ --- диагональная матрица.
\item Представьте $A^{2012}$ в виде произведения трёх матриц.
\end{enumerate}

\item Вася и Петя независимо друг от друга решают тест по теории вероятностей. В тесте всего два вопроса. На каждый вопрос два варианта ответа. Петя знает решение каждого вопроса с вероятностью $0{,}7$. Если Петя не знает решения, то он отвечает равновероятно наугад. Вася знает решение каждого вопроса с вероятностью $0{,}5$. Если Вася не знает решения, то он отвечает равновероятно наугад.
\begin{enumerate}
\item Какова вероятность того, что Петя правильно ответил на оба вопроса?
\item Какова вероятность того, что Петя правильно ответил на оба вопроса, если его ответы совпали с Васиными?
\item Чему равно математическое ожидание числа Петиных верных ответов?
\item Чему равно математическое ожидание числа Петиных верных ответов, если его ответы совпали с Васиными?
\end{enumerate}

\item Для случайных величин $X$ и $Y$ заданы следующие значения: $\E(X)=1$, $\E(Y)=4$, $\E(XY)=8$, $Var(X)=Var(Y)=9$. Для случайных величин $U=X+Y$ и $V=X-Y$ вычислите: 
\begin{enumerate}
\item $\E(U)$, $Var(U)$, $\E(V)$, $\Var(V)$, $\Cov(U,V)$ 
\item Можно ли утверждать, что случайные величины U и V независимы? 
\end{enumerate}

\item Вася ведёт блог. Обозначим $X_i$ --- количество слов в $i$--ой записи. После первого года он по своим записям обнаружил, что $\bar{X}_{200}=95$ и выборочное стандартное отклонение равно 282 слова. На уровне значимости $\alpha=0.10$ проверьте гипотезу о том, что $\mu=100$ против альтернативной гипотезы $\mu\neq 100$. Найдите также P-значение.


\end{enumerate}

\end{document}