\documentclass[12pt, a4paper]{article}

\usepackage[top=1.5cm, left=2cm, right=2cm, bottom=1.5cm]{geometry} % размер текста на странице

\usepackage{tikz} % картинки в tikz
\usepackage{microtype} % свешивание пунктуации

\usepackage{array} % для столбцов фиксированной ширины

\usepackage{indentfirst} % отступ в первом параграфе

\usepackage{sectsty} % для центрирования названий частей
\allsectionsfont{\centering}

\usepackage{amsmath} % куча стандартных математических плюшек
\usepackage{amssymb} % и символов
\usepackage{bbm}

\usepackage{multicol} % текст в несколько колонок

\usepackage{lastpage} % чтобы узнать номер последней страницы

\usepackage{enumitem} % дополнительные плюшки для списков
%  например \begin{enumerate}[resume] позволяет продолжить нумерацию в новом списке




\usepackage{fontspec} % хз
\usepackage{polyglossia} % для выбора языка в xelatex

\setmainlanguage{russian}
\setotherlanguages{english}

% download "Linux Libertine" fonts:
% http://www.linuxlibertine.org/index.php?id=91&L=1
\setmainfont{Linux Libertine O} % or Helvetica, Arial, Cambria
% why do we need \newfontfamily:
% http://tex.stackexchange.com/questions/91507/
\newfontfamily{\cyrillicfonttt}{Linux Libertine O}

\AddEnumerateCounter{\asbuk}{\russian@alph}{щ} % для списков с русскими буквами
\setlist[enumerate, 2]{label=\asbuk*),ref=\asbuk*} % списки уровня 2 будут буквами а) б) ...

\usepackage{todonotes} % для вставки в документ заметок о том, что осталось сделать
% \todo[inline]{Здесь надо коэффициенты исправить}
% \missingfigure{Здесь будет картина Последний день Помпеи}
% команда \listoftodos — печатает все поставленные \todo'шки

\usepackage{booktabs} % красивые таблицы
% заповеди из документации:
% 1. Не используйте вертикальные линии
% 2. Не используйте двойные линии
% 3. Единицы измерения помещайте в шапку таблицы
% 4. Не сокращайте .1 вместо 0.1
% 5. Повторяющееся значение повторяйте, а не говорите "то же"


% \usepackage[left=1cm,right=1cm,top=1cm,bottom=1cm]{geometry}

\usepackage{fancyhdr} % весёлые колонтитулы
\pagestyle{fancy}
\lhead{Эконометрика, проверка дз 2}
\chead{}
\rhead{2017-11-06}
\lfoot{}
\cfoot{}
\rfoot{\thepage/\pageref{LastPage}}
\renewcommand{\headrulewidth}{0.4pt}
\renewcommand{\footrulewidth}{0.4pt}

\DeclareMathOperator{\E}{\mathbb{E}}
\let\P\relax
\DeclareMathOperator{\P}{\mathbb{P}}
\DeclareMathOperator{\Var}{\mathbb{V}ar}
\DeclareMathOperator{\Cov}{\mathbb{C}ov}



%% эконометрические сокращения
\def \hb{\hat{\beta}}
\DeclareMathOperator{\sVar}{sVar}
\DeclareMathOperator{\sCov}{sCov}
\DeclareMathOperator{\sCorr}{sCorr}

\def \1{\mathbbm{1}}

\def \hs{\hat{s}}
\def \hy{\hat{y}}
\def \hY{\hat{Y}}
\def \he{\hat{\varepsilon}}
\def \v1{\vec{1}}
\def \cN{\mathcal{N}}
\def \e{\varepsilon}
\def \z{z}

\def \hVar{\widehat{\Var}}
\def \hCorr{\widehat{\Corr}}
\def \hCov{\widehat{\Cov}}

\DeclareMathOperator{\tr}{tr}
\DeclareMathOperator*{\plim}{plim}

%% лаг
\renewcommand{\L}{\mathrm{L}}


\begin{document}


\begin{enumerate}

\item Рассмотрим векторы: $x = (2, 0, 1)$, $z = (1, 0, 1)$ и $y= (1, 2, 3)$.
\begin{enumerate}
  \item Найдите матрицу-шляпницу, проецирующую любой вектор на линейную оболочку векторов $x$ и $z$.
  \item Найдите $\partial \hat y_2 / \partial y_3$ и $\partial \hat y_3/ \partial y_2$.
\end{enumerate}



\item В рамках классической регрессионной модели $y=X\beta + u$, где $\E(u)=0$, $\Var(u)=\sigma^2 \cdot I$, вектор $\hb$ оценивается с помощью МНК. Обозначим $\hy=X\hb$, $\hat{u}=y-\hy$.

Найдите $\E(\hy)$, $\E(\hat{u})$, $\Var(\hy)$, $\Cov(\hy, \hb)$.

\item Рассмотрим модель парной регрессии $y = \beta_1 \cdot \1 + \beta_2 x + u$.

\begin{enumerate}
  \item Нарисуйте векторы $x$, $\1$, $y$, $\hy$, $\bar y \cdot \1$.
  \item Укажите все прямые углы на рисунке.
  \item Отметьте угол, квадрат косинуса которого равен $R^2$.
  \item Закончите фразу так, чтобы она была корректной
  \begin{enumerate}
    \item Вектор $y - \bar{y}\cdot \1$ — это проекция вектора $y$ на \ldots
    \item Вектор $\hb_2 (x - \bar x\cdot \1)$ — это проекция вектора $y$ на \ldots
  \end{enumerate}
\end{enumerate}

\item Для регрессии $\hat y_i = \hb_1 + \hb_2 x_i$ геометрически докажите, что $\bar y = \hb_1 + \hb_2 \bar x$.

Попытайтесь обобщить это геометрическое доказательство на случай $k$ коэффициентов $\hb_j$.

\item Рассмотрим парную регрессию $\hat y_i = \hb_1 + \hb_2 x_i$. Исследователь Василий посчитал величину $t/\sqrt{n-2}$, где $t$ — это $t$-статистика, проверяющая гипотезу $H_0$: $\beta_2 = 0$.

Какой геометрический смысл имеет эта величина?

Подсказка: отношение катетов называется\ldots

\item
Пусть $y_i = \beta_1 + \beta_2 x_{i2} + \beta_3 x_{i3} + \e_i$ — регрессионная модель, где $X = \begin{pmatrix} 1 & 0 & 0 \\ 1 & 0 & 0 \\ 1 & 0 & 0 \\ 1 & 1 & 0 \\ 1 & 1 & 1 \end{pmatrix}$, $y = \begin{pmatrix} 1 \\ 2 \\ 3 \\ 4 \\ 5 \end{pmatrix}$, $\beta = \begin{pmatrix} \beta_1 \\ \beta_2 \\ \beta_3 \end{pmatrix}$, $\e = \begin{pmatrix} \e_1 \\ \e_2 \\ \e_3 \\ \e_4 \\ \e_5  \end{pmatrix}$, ошибки $\e_i$ независимы и нормально распределены с $\E(\e)$ = 0, $Var(\e) = \sigma^2 I$. Для удобства расчётов даны матрицы: $X'X = \begin{pmatrix} 5 & 2 & 1 \\ 2 & 2 & 1\\ 1 & 1 & 1 \end{pmatrix}$ и $(X'X)^{-1}= \begin{pmatrix} 0.3333 & -0.3333 & 0.0000 \\ -0.3333 & 1.3333 & -1.0000 \\ 0.0000 & -1.0000 & 2.0000 \end{pmatrix}$.



\begin{enumerate}
\item Укажите число наблюдений.
\item Методом МНК найдите оценку для вектора неизвестных коэффициентов.

\item Укажите число регрессоров в модели, учитывая константу.
\item Найдите $TSS = \sum_{i=1}^n (y_i - \bar y)^2$.
\item Найдите $RSS = \sum_{i=1}^n (y_i - \hy_i)^2$.
\end{enumerate}


\end{enumerate}








\end{document}
