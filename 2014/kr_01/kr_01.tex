\documentclass[12pt,a4paper]{article}
\usepackage[utf8]{inputenc}
\usepackage[russian]{babel}

\usepackage{amsmath}
\usepackage{amsfonts}
\usepackage{amssymb}
\usepackage[left=2cm,right=2cm,top=2cm,bottom=2cm]{geometry}

\providecommand{\tr}{\mathrm{tr}}

\begin{document}
\parindent=0 pt % отступ равен 0

{\Large Вперёд, в рукопашную! }

\begin{enumerate}
\item Сформулируйте теорему о трёх перпендикулярах и обратную к ней.
\item Для матрицы 

$A=\left(%
\begin{array}{cc}
  3 & 4 \\
  4 & 9 \\
\end{array}%
\right)$ \\

\begin{enumerate}
\item Найдите собственные числа и собственные векторы матрицы.
\item Найдите обратную матрицу, $A^{-1}$, ее собственные векторы и собственные числа.
\item Представьте матрицу $A$ в виде $A=CDC^{-1}$, где $D$ --- диагональная матрица.
\item Найдите $A^{42}$
\item Не находя $A^{100}$ найдите $\tr(A^{100})$ и $\det(A^{100})$ 
\end{enumerate}



\item Игрок получает случайным образом 13 карт из колоды в 52 карты. 
\begin{enumerate}
\item Какова вероятность, что у него как минимум два туза?
\item Каково ожидаемое количество тузов у игрока?
\item Какова вероятность, что у него как минимум два туза, если
известно, что у него есть хотя бы один туз?
\item Каково ожидаемое количество тузов у игрока, если известно, что у него на руках хотя бы один туз?
\end{enumerate}


\item В ходе анкетирования 100 сотрудников банка <<Омега>> ответили на вопрос о том, сколько времени они проводят на работе ежедневно. Среднее выборочное оказалось равно $9.5$ часам при выборочном стандартном отклонении $0.5$ часа. 
\begin{enumerate}
\item Постройте 95\% доверительный интервал для математического ожидания времени проводимого сотрудниками на работе 
\item Проверьте гипотезу о том, что в среднем люди проводят на работе 10 часов, против альтернативной гипотезы о том, что в среднем люди проводят на работе меньше 10 часов, укажите точное Р-значение.
\end{enumerate}



\end{enumerate}

\end{document}
