\documentclass[pdftex,12pt,a4paper]{article}\usepackage[]{graphicx}\usepackage[]{color}
%% maxwidth is the original width if it is less than linewidth
%% otherwise use linewidth (to make sure the graphics do not exceed the margin)
\makeatletter
\def\maxwidth{ %
  \ifdim\Gin@nat@width>\linewidth
    \linewidth
  \else
    \Gin@nat@width
  \fi
}
\makeatother

\definecolor{fgcolor}{rgb}{0.345, 0.345, 0.345}
\newcommand{\hlnum}[1]{\textcolor[rgb]{0.686,0.059,0.569}{#1}}%
\newcommand{\hlstr}[1]{\textcolor[rgb]{0.192,0.494,0.8}{#1}}%
\newcommand{\hlcom}[1]{\textcolor[rgb]{0.678,0.584,0.686}{\textit{#1}}}%
\newcommand{\hlopt}[1]{\textcolor[rgb]{0,0,0}{#1}}%
\newcommand{\hlstd}[1]{\textcolor[rgb]{0.345,0.345,0.345}{#1}}%
\newcommand{\hlkwa}[1]{\textcolor[rgb]{0.161,0.373,0.58}{\textbf{#1}}}%
\newcommand{\hlkwb}[1]{\textcolor[rgb]{0.69,0.353,0.396}{#1}}%
\newcommand{\hlkwc}[1]{\textcolor[rgb]{0.333,0.667,0.333}{#1}}%
\newcommand{\hlkwd}[1]{\textcolor[rgb]{0.737,0.353,0.396}{\textbf{#1}}}%

\usepackage{framed}
\makeatletter
\newenvironment{kframe}{%
 \def\at@end@of@kframe{}%
 \ifinner\ifhmode%
  \def\at@end@of@kframe{\end{minipage}}%
  \begin{minipage}{\columnwidth}%
 \fi\fi%
 \def\FrameCommand##1{\hskip\@totalleftmargin \hskip-\fboxsep
 \colorbox{shadecolor}{##1}\hskip-\fboxsep
     % There is no \\@totalrightmargin, so:
     \hskip-\linewidth \hskip-\@totalleftmargin \hskip\columnwidth}%
 \MakeFramed {\advance\hsize-\width
   \@totalleftmargin\z@ \linewidth\hsize
   \@setminipage}}%
 {\par\unskip\endMakeFramed%
 \at@end@of@kframe}
\makeatother

\definecolor{shadecolor}{rgb}{.97, .97, .97}
\definecolor{messagecolor}{rgb}{0, 0, 0}
\definecolor{warningcolor}{rgb}{1, 0, 1}
\definecolor{errorcolor}{rgb}{1, 0, 0}
\newenvironment{knitrout}{}{} % an empty environment to be redefined in TeX

\usepackage{alltt}

\input{title_bor_knitr}



%
\def \useR{$[$R$]$ }

%% эконометрические сокращения
\def \hb{\hat{\beta}}
\def \b{\beta}
\def \hs{\hat{s}}
\def \hy{\hat{y}}
\def \hY{\hat{Y}}
\def \he{\hat{\varepsilon}}
\def \v1{\vec{1}}
\def \e{\varepsilon}
\def \z{z}
\def \hVar{\widehat{\Var}}
\def \hCorr{\widehat{\Corr}}
\def \hCov{\widehat{\Cov}}
\def \cN{\mathcal{N}}
\renewcommand{\P}{\mathbb{P}}

%% лаг
\renewcommand{\L}{\mathrm{L}}

\IfFileExists{upquote.sty}{\usepackage{upquote}}{}
\begin{document}
\parindent=0 pt % отступ равен 0

\section*{Паниковать на контрольной строго воспрещается! :)}
\thispagestyle{empty}
\begin{enumerate}

\item  По 47 наблюдениям оценивается зависимость доли мужчин занятых в сельском хозяйстве от уровня образованности и доли католического населения по Швейцарским кантонам в 1888 году.

\[Agriculture_i=\beta_1+\beta_2 Examination_i+\beta_3 Catholic_i+\varepsilon_i\]





% latex table generated in R 3.1.0 by xtable 1.7-4 package
% Thu Oct 23 09:00:25 2014
\begin{table}[ht]
\centering
\begin{tabular}{rrrr}
  \hline
 & Оценка & Ст. ошибка & t-статистика \\ 
  \hline
(Intercept) &  & 8.72 & 9.44 \\ 
  Examination & -1.94 &  & -5.08 \\ 
  Catholic & 0.01 & 0.07 &  \\ 
   \hline
\end{tabular}
\end{table}


\begin{enumerate}
\item Заполните пропуски в таблице
\item Укажите коэффициенты, значимые на 10\% уровне значимости.
\item Постройте 99\%-ый доверительный интервал для коэффициента при переменной Catholic 
\end{enumerate}

\item В рамках классической линейной модели с неслучайными регрессорами найдите $\Var(\hat{\varepsilon})$, $\Cov(\hat{\beta},\hat{\varepsilon})$. Верно ли, что $\Cov(\hat{\varepsilon}_1,\hat{\varepsilon}_2)=0$?





\item Эконометресса Ефросинья оценивала модель $y_i=\beta_1 + \beta_2 x_i + \beta_3 z_i + \varepsilon_i$. Найдя матрицы $X'X$ и $(X'X)^{-1}$, она призадумалась...

$X'X = \begin{bmatrix}{}
  47 & 775 & 1934 \\ 
  775 & 15707 & 23121 \\ 
  1934 & 23121 & 159570 \\ 
  \end{bmatrix}$, 
$(X'X)^{-1}=\begin{bmatrix}{}
  0.26653 & -0.01067 & -0.00168 \\ 
  -0.01067 & 0.00051 & 0.00006 \\ 
  -0.00168 & 0.00006 & 0.00002 \\ 
  \end{bmatrix}$


\begin{enumerate}
\item Помогите Ефросинье найти количество наблюдений, $\bar{z}$, $\sum x_i z_i$, $\sum(x_i-\bar{x})(z_i-\bar{z})$
\item (*) Ефросинья решила зачем-то также оценить модель $x_i = \gamma_1 + \gamma_2 z_i + u_i$. Как она может найти RSS в новой модели в одно арифметическое действие?
\end{enumerate}

\item Регрессионная модель  задана в матричном виде при помощи уравнения $y=X\beta+\varepsilon$, где $\beta=(\beta_1,\beta_2,\beta_3)'$.
Известно, что $\E(\varepsilon)=0$  и  $\Var(\varepsilon)=\sigma^2\cdot I$.
Известно также, что 

$y=\left(
\begin{array}{c} 
1\\ 
2\\ 
3\\ 
4\\ 
2
\end{array}\right)$, 
$X=\left(\begin{array}{ccc}
1 & 0 & 0 \\
1 & 0 & 0 \\
1 & 1 & 0 \\
1 & 1 & 0 \\
1 & 1 & 1 
\end{array}\right)$.


Для удобства расчетов приведены матрицы 


$X'X=\left(
\begin{array}{ccc} 
5 & 3 & 1\\ 
3 & 3 & 1\\ 
1 & 1 & 1 
\end{array}\right)$ и $(X'X)^{-1}=\frac{1}{2}\left(
\begin{array}{ccc} 
1 & -1 & 0 \\
-1 & 2 & -1 \\
0 & -1 & 3
\end{array}\right)$.

\begin{enumerate}
\item Найдите вектор МНК-оценок коэффициентов $\hat{\beta}$.
\item Найдите несмещенную оценку для неизвестного параметра $\sigma^2$.
\item Проверьте гипотезу $\beta_2=0$ против альтернативной о неравенстве на уровне значимости 5\%

\end{enumerate}


\end{enumerate}


\end{document}
