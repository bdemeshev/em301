\documentclass[pdftex,11pt,openany]{book}

%\documentclass[11pt,openany]{book}



\input{title_bor_utf8_knitr}


% чисто эконометрические сокращения:

\def \useR{$[$R$]$ }

%% эконометрические сокращения
\def \hb{\hat{\beta}}
\def \b{\beta}
\def \hs{\hat{s}}
\def \hy{\hat{y}}
\def \hY{\hat{Y}}
\def \he{\hat{\varepsilon}}
\def \v1{\vec{1}}
\def \e{\varepsilon}
\def \z{z}
\def \hVar{\widehat{\Var}}
\def \hCorr{\widehat{\Corr}}
\def \hCov{\widehat{\Cov}}
\def \cN{\mathcal{N}}
\renewcommand{\P}{\mathbb{P}}

%% лаг
\renewcommand{\L}{\mathrm{L}}



% стандартизация
% эпсилон во временных рядах --- белый шум, а в остальных сюжетах --- остатки, подумать
% транспонирование --- штрих

% задачи типа "воспроизведите тест такой-то ручками в R" -> в тему согласно тесту, а не в доп. задачи по программированию

% идеи задач:
% * Задача на корреляционную матрицу по реальным данным require(quantmod)
% Задача про суеверную Мырли. Можно ли там что-то про se сказать?
% теорему FWL в массы!
% симуляционные задачи на ошибки 1, 2 рода, мощность
% E_t(X) или с указанием сигма-алгебры

% выложить преамбулу


% перегнанные банки:
% 1 ---20

% осталось:
% задача 12, компьютерное про пи (алгоритмы вычисления)
% распределить проверить наличие программистко-сюжетных упражнений


% для отдельной упаковки решений
\input{prob_and_sol_2chap}


\title{Эконометрика \\ {\small с Монте-Карло и эконометрессами} \\ в задачах и упражнениях}
\author{Дмитрий Борзых, Борис Демешев}
\date{\today}

\makeindex % команда для создания предметного указателя
\bibliographystyle{plain} % стиль оформления ссылок


\begin{document}

\setcounter{page}{3}

%\maketitle % печатаем заголовок
\tableofcontents

\parindent=0 pt % отступ равен 0


\vspace{30pt}


\begin{knitrout}
\definecolor{shadecolor}{rgb}{0.969, 0.969, 0.969}\color{fgcolor}\begin{kframe}
\begin{alltt}
\hlkwd{require}\hlstd{(knitr)}
\hlstd{opts_chunk}\hlopt{$}\hlkwd{set}\hlstd{(}\hlkwc{cache}\hlstd{=}\hlnum{FALSE}\hlstd{,}
               \hlkwc{dev}\hlstd{=}\hlstr{"png"}\hlstd{,}\hlkwc{dpi}\hlstd{=}\hlnum{300}\hlstd{,}
               \hlkwc{warning}\hlstd{=}\hlnum{FALSE}\hlstd{,}
               \hlkwc{tidy}\hlstd{=}\hlnum{FALSE}\hlstd{,}
               \hlkwc{echo}\hlstd{=}\hlnum{TRUE}\hlstd{,}
               \hlkwc{out.height}\hlstd{=}\hlstr{"7cm"}\hlstd{,}\hlkwc{out.width}\hlstd{=}\hlstr{"7cm"}\hlstd{)}

\hlkwd{library}\hlstd{(}\hlstr{"ggplot2"}\hlstd{)}
\hlkwd{require}\hlstd{(}\hlstr{"Hmisc"}\hlstd{)}
\hlkwd{library}\hlstd{(}\hlstr{"lmtest"}\hlstd{)}
\hlkwd{library}\hlstd{(}\hlstr{"apsrtable"}\hlstd{)}
\hlkwd{library}\hlstd{(}\hlstr{"xtable"}\hlstd{)}
\hlkwd{library}\hlstd{(}\hlstr{"MASS"}\hlstd{)}
\hlkwd{library}\hlstd{(}\hlstr{"car"}\hlstd{)}
\hlkwd{library}\hlstd{(}\hlstr{"texreg"}\hlstd{)}

\hlkwd{library}\hlstd{(}\hlstr{"econru"}\hlstd{)}

\hlkwd{theme_set}\hlstd{(}\hlkwd{theme_bw}\hlstd{())}

\hlkwd{load}\hlstd{(}\hlstr{'pset_data.Rdata'}\hlstd{)}
\end{alltt}


{\ttfamily\noindent\bfseries\color{errorcolor}{\#\# Error: cannot open the connection}}\end{kframe}
\end{knitrout}

%\listoftodos

% МНК


\chapter{Решения и ответы к избранным задачам}
\chaptermark{Избранные решения}

\solutiononly


% МНК


\chapter{МНК без матриц и вероятностей}

\begin{problem}
 Верно ли, что для любых векторов $a = (a_1,\dots,a_n)$ и $b = (b_1,\dots,b_n)$ справедливы следующие равенства?
\begin{enumerate}
\item $\sum_{i=1}^n {(a_i-\bar a)} = 0$
\item $\sum_{i=1}^n {(a_i-\bar a)^2} = \sum_{i=1}^n {(a_i-\bar a)a_i}$
\item $\sum_{i=1}^n {(a_i-\bar a)(b_i-\bar b)} = \sum_{i=1}^n {(a_i-\bar a)b_i}$
\item $\sum_{i=1}^n {(a_i-\bar a)(b_i-\bar b)} = \sum_{i=1}^n {a_i b_i}$
\end{enumerate}
\end{problem}

\begin{solution}
 да, да, да, нет
\end{solution}




\begin{problem}
 При помощи метода наименьших квадратов найдите оценку неизвестного параметра $\theta$ в следующих моделях:

\begin{enumerate}
\item $y_i = \theta + \theta x_i + \varepsilon_i$
\item $y_i = \theta - \theta x_i + \e_i$
\item $\text{ln} y_i = \theta + \text{ln} x_i + \e_i$
\item $y_i = \theta + x_i + \e_i$
\item $y_i = 1 + \theta x_i + \e_i$
\item $y_i = \theta / x_i + \e_i$
\item $y_i = \theta x_{i1} + (1-\theta)x_{i2}+\e_i$
\end{enumerate}
\end{problem}

\begin{solution}
\end{solution}

\begin{problem}
 Покажите, что для моделей $y_i= \alpha + \beta x_i + \e_i$, $z_i = \gamma + \delta x_i + \upsilon_i$ и $y_i + z_i = \mu + \lambda x_i + \xi_i$ МНК-оценки связаны соотношениями $\hat{\mu}=\hat{\alpha}+\hat{\gamma}$ и $\hat{\lambda}=\hat{\beta} + \hat{\delta}$.
\end{problem}

\begin{solution}
\end{solution}

\begin{problem}
 Найдите МНК-оценки параметров $\alpha$ и $\beta$ в модели $y_i = \alpha + \beta y_i + \e_i$.
\end{problem}

\begin{solution}
\end{solution}

\begin{problem}
 Рассмотрите модели $y_i = \alpha + \beta (y_i + z_i) + \e_i$, $z_i = \gamma + \delta(y_i+z_i) + \e_i$. 
\begin{enumerate}
\item Как связаны между собой $\hat{\alpha}$ и $\hat{\gamma}$?
\item  Как связаны между собой $\hat{\beta}$ и $\hat{\delta}$?
\end{enumerate} 
\end{problem}

\begin{solution}
$\hat{\alpha} + \hat{\gamma} = 0$ и $\hat{\beta} + \hat{\delta} = 1$
\end{solution}



\begin{problem}
 Как связаны МНК-оценки параметров $\alpha, \beta$ и $\gamma, \delta$ в моделях $y_i = \alpha + \beta x_i + \e_i$ и $z_i = \gamma + \delta x_i + \upsilon_i$, если $z_i = 2 y_i$.
\end{problem}

\begin{solution}
\end{solution} 

\begin{problem}
 Для модели $y_i = \beta_1 x_{i1} + \beta_2 x_{i2} + \e_i$ решите условную задачу о наименьших квадратах: $Q(\beta_1, \beta_2) := \sum_{i=1}^n (y_i - \beta_1 x_{i1} - \beta_2 x_{i2})^2 \rightarrow \underset{\beta_1 + \beta_2 = 1}{\min}$
\end{problem}

\begin{solution}
\end{solution} 
 


\begin{problem}
Даны $n$ пар чисел: $(x_1, y_1)$, \ldots, $(x_n,y_n)$. Мы прогнозируем $y_i$ по формуле $\hy_i=\hb x_i$. Найдите $\hb$ методом наименьших квадратов. 
\end{problem} 

\begin{solution}
$\hb=\sum x_i y_i/\sum x_i^2$
\end{solution}

\begin{problem}
Даны $n$ чисел: $y_1$, \ldots, $y_n$. Мы прогнозируем $y_i$ по формуле $\hy_i=\hb$. Найдите $\hb$ методом наименьших квадратов. 
\end{problem}
\begin{solution}
$\hb=\bar{y}$
\end{solution}

\begin{problem}
Даны $n$ пар чисел: $(x_1, y_1)$, \ldots, $(x_n,y_n)$. Мы прогнозируем $y_i$ по формуле $\hy_i=\hb_1+\hb_2 x_i$. Найдите $\hb_1$ и $\hb_2$ методом наименьших квадратов. 
\end{problem}
\begin{solution}
$\hb_2=\sum (x_i-\bar{x})(y_i-\bar{y})/\sum(x_i-\bar{x})^2$, $\hb_1=\bar{y}-\hb_2\bar{x}$
\end{solution}

\begin{problem}
Даны $n$ пар чисел: $(x_1, y_1)$, \ldots, $(x_n,y_n)$. Мы прогнозируем $y_i$ по формуле $\hy_i=1+\hb x_i$. Найдите $\hb$ методом наименьших квадратов. 
\end{problem}
\begin{solution}
$\hb=\sum x_i (y_i-1)/\sum x_i^2$
\end{solution}

\begin{problem}
 Перед нами два золотых слитка и весы, производящие взвешивания с ошибками. Взвесив первый слиток, мы получили результат $300$ грамм, взвесив второй слиток --- $200$ грамм, взвесив оба слитка --- $400$ грамм. Оцените вес каждого слитка методом наименьших квадратов.
 \end{problem}
\begin{solution}
 $(300-\hb_1)^2+(200-\hb_2)^2+(400-\hb_1-\hb_2)^2\to\min$ 
\end{solution}


\begin{problem}
 Аня и Настя утверждают, что лектор опоздал на 10 минут. Таня считает, что лектор опоздал на 3 минуты. С помощью мнк оцените на сколько опоздал лектор. 
\end{problem}
\begin{solution}
 $2\cdot (10-\hb)^2+(3-\hb)^2\to\min$ 
\end{solution}


\begin{problem}
 Функция $f(x)$ непрерывна на отрезке $[0;1]$. Найдите аналог МНК-оценок для регрессии без свободного члена в непрерывном случае. Более подробно: найдите минимум по $\hb$ для функции
\begin{equation}
Q(\hb)= \int_0^1 (f(x)-\hb x)^2\,dx
\end{equation}
\end{problem}

\begin{solution}
\end{solution}


\begin{problem}
 Есть двести наблюдений. Вовочка оценил модель $\hy=\hb_1+\hb_2 x$ по первой сотне наблюдений. Петечка оценил модель $\hy=\hat{\gamma}_1+\hat{\gamma}_2 x$ по второй сотне наблюдений. Машенька оценила модель $\hy=\hat{m}_1+\hat{m}_2 x$ по всем наблюдениям.
\begin{enumerate}
\item Возможно ли, что $\hb_2>0$, $\hat{\gamma}_2>0$, но $\hat{m}_2<0$?
\item Возможно ли, что $\hb_1>0$, $\hat{\gamma}_1>0$, но $\hat{m}_1<0$?
\item Возможно ли одновременное выполнение всех упомянутых условий?
\end{enumerate}
\end{problem}

\begin{solution}
да, возможно. Два вытянутых облачка точек. Первое облачко даёт первую регрессию, второе --- вторую. Прямая, соединяющая центры облачков, --- общую.
\end{solution}



\begin{problem}
 Вася оценил модель $y=\beta_1+\beta_2 d+\beta_3 x+\varepsilon$. Дамми-переменная $d$ обозначает пол, 1 для мужчин и 0 для женщин. Оказалось, что $\hat{\beta}_2>0$. Означает ли это, что для мужчин $\bar{y}$ больше, чем $\bar{y}$ для женщин?
\end{problem}

\begin{solution}
Нет. Коэффициенты можно интерпретировать только <<при прочих равных>>, т.е. при равных $x$. Из-за разных $x$ может оказаться, что у мужчин $\bar{y}$ меньше, чем $\bar{y}$ для женщин.
\end{solution}



\begin{problem}
 Какие из указанные моделей можно представить в линейном виде?
\begin{enumerate}
\item $y_i=\beta_1+\frac{\beta_2}{x_i}+\e_i$
\item $y_i=\exp(\beta_1+\beta_2 x_i+\e_i)$
\item $y_i=1+\frac{1}{\exp(\beta_1+\beta_2 x_i+\e_i)}$
\item $y_i=\frac{1}{1+\exp(\beta_1+\beta_2 x_i+\e_i)}$
\item $y_i=x_i^{\beta_2}e^{\beta_1+\e_i}$
\item $y_i=\beta_1\exp(\beta_2 x_i + \e_i)$
\end{enumerate}
\end{problem}

\begin{solution}
\newpage
\end{solution}



\begin{problem}
 У эконометриста Вовочки есть переменная $1_f$, которая равна 1, если $i$-ый человек в выборке --- женщина, и 0, если мужчина. Есть переменная $1_m$, которая равна 1, если $i$-ый человек в выборке --- мужчина, и 0, если женщина. Какие $\hy$ получатся, если Вовочка попытается построить регрессии:
\begin{enumerate}
\item $y$ на константу и $1_f$
\item $y$ на константу и $1_m$
\item $y$ на $1_f$ и $1_m$ без константы
\item $y$ на константу, $1_f$ и $1_m$
\end{enumerate}
\end{problem}

\begin{solution}
\end{solution}
 


\begin{problem}
 У эконометриста Вовочки есть три переменных: $r_i$ --- доход $i$-го человека в выборке, $m_i$ --- пол (1 --- мальчик, 0 --- девочка) и $f_i$ --- пол (1 --- девочка, 0 --- мальчик). Вовочка оценил две модели
\begin{enumerate}
\item[] Модель A: $m_i=\beta_1+\beta_2 r_i+\ve_i$
\item[] Модель B: $f_i=\gamma_1+\gamma_2 r_i+u_i$
\end{enumerate}
\begin{enumerate}
\item Как связаны между собой оценки $\hb_1$ и $\hat{\gamma}_1$?
\item Как связаны между собой оценки $\hb_2$ и $\hat{\gamma}_2$? 
\end{enumerate}
\end{problem}

\begin{solution}
 Оценки МНК линейны по объясняемой переменной. Если сложить объясняемые переменные в этих двух моделях, то получится вектор из единичек. Если строить регрессию вектора из единичек на константу и $r$, то получатся оценки коэффициентов 1 и 0. Значит, $\hb_1+\hat{\gamma}_1=1$, $\hb_2+\hat{\gamma}_2=0$ 
\end{solution}



\begin{problem}
 Эконометрист Вовочка оценил линейную регрессионную модель, где $y$ измерялся в тугриках. Затем он оценил ту же модель, но измерял $y$ в мунгу (1 тугрик = 100 мунгу). Как изменятся оценки коэффициентов?
\end{problem}

\begin{solution}
 Увеличатся в 100 раз
\end{solution}


\begin{problem}
 Возможно ли, что при оценке парной регрессии $y=\beta_1+\beta_2 x+\e$ оказывается, что $\hb_2>0$, а при оценке регрессии без константы, $y=\gamma x+\e$, оказывается, что $\hat{\gamma}<0$?
\end{problem}

\begin{solution}
 да
\end{solution}



\begin{problem}
 Эконометрист Вовочка оценил регрессию $y$ только на константу. Какой коэффициент $R^2$ он получит?
\end{problem}

\begin{solution}
$R^2=0$
\end{solution}



\begin{problem}
 Эконометрист Вовочка оценил методом наименьших квадратов модель 1, $y=\b_1+\b_2 x+\b_3 z+\e$, а затем модель 2, $y=\b_1+\b_2 x+\b_3 z+\b_4 w+\e$. Сравните полученные $ESS$, $RSS$, $TSS$ и $R^2$.
\end{problem}

\begin{solution}
 $TSS_1=TSS_2$, $R_2^2\geq R_2^1$, $ESS_2\geq ESS_1$, $RSS_2\leq RSS_1$
\end{solution}




\begin{problem}
  Создайте набор данных с тремя переменными $y$, $x$ и $z$ со следующими свойствами. При оценке модели $\hy=\hb_1+\hb_2 x$ получается $\hb_2>0$. При оценке модели $\hy=\hat{\gamma}_1+ \hat{\gamma}_2 x+\hat{\gamma}_3 z$ получается $\hat{\gamma}_2<0$. Объясните принцип, руководствуясь которым легко создать такой набор данных.
\end{problem}

\begin{solution}
\end{solution}


\begin{problem}
 У меня есть набор данных с выборочным средним $\bar{y}$ и выборочной дисперсией $s_y^2$. Как нужно преобразовать данные, чтобы выборочное среднее равнялось $7$, а выборочная дисперсия --- $9$? 
\end{problem}

\begin{solution}
 $y_i^*=7+3(y_i-\bar{y})/s_y$ 
% эта задача не использует понятия вероятностей, хотя близка. Пусть будет в невероятностной секции.
\end{solution}



\chapter{Парный МНК без матриц}


\begin{problem}
 Рассмотрим модель $y_t=\b_1+\b_2 \cdot t + \e_t$, где ошибки $\e_t$ независимы и равномерны на $[-1;1]$. С помощью симуляций на компьютере оцените и постройте график функции плотности для $\hb_1$, $\hb_2$, $\hs^2$, $\hVar(\hb_1)$, $\hVar(\hb_2)$ и $\hCov(\hb_1,\hb_2)$. 
\end{problem}
 
\begin{solution}
\end{solution}


\begin{problem}
 Пусть $y_i=\mu+\e_i$, где $\E(\e_i)=0$, $\Var(\e_i)=\sigma^2$, $\Cov(\e_i,\e_j)=0$ при $i \ne j$. Найдите:
\begin{enumerate}
\item $\E(\overline{y})$
\item $\Var(\overline{y})$
\item $\E(\frac{1}{n}\sum_{i=1}^n {(y_i-\overline{y})}^2)$
\item $\Var(\frac{1}{n}\sum_{i=1}^n {(y_i-\overline{y})}^2)$, если дополнительно известно, что $\e_i$ нормально распределены
\end{enumerate}
\end{problem}
 
\begin{solution}
\end{solution}


\begin{problem}
 Рассматривается модель $y_i=\beta x_i+\e_i$, $\E(\e_i)=0$, $\Var(\e_i)=\sigma^2$, $\Cov(\e_i,\e_j)=0$ при $i \ne j$. 
При каких значениях параметров $c_i$ несмещённая оценка $\hat{\beta}=\frac{\sum_{i=1}^n {c_i y_i}}{\sum_{i=1}^n {c_i x_i}}$ имеет наименьшую дисперсию? 
\end{problem}
 
\begin{solution}
$c_i=c\cdot x_i$, где $c\neq 0$ 
\end{solution}


\begin{problem}
 Пусть $y_i = \beta_1 + \beta_2 x_i + \e_i$ и $i = 1, \dots, 5$ --- классическая регрессионная модель. Также имеются следующие данные: $\sum_{i=1}^5 y_i^2 = 55, \sum_{i=1}^5 x_i^2 = 3, \sum_{i=1}^5 x_iy_i = 12, \sum_{i=1}^5 y_i = 15, \sum_{i=1}^5 x_i = 3.$ Используя их, найдите:

\begin{enumerate}
\item $\hat{\beta_1}$ и $\hat{\beta_2}$
\item $\Corr(\hat{\beta_1}, \hat{\beta_2})$
\item $TSS$
\item $ESS$
\item $RSS$
\item $R^2$
\item $\hat{\sigma}^2$
\end{enumerate}

Проверьте следующие гипотезы:
\begin{enumerate}
\item $\begin{cases}  H_0: \beta_2 = 2  \\ H_a: \beta_2 \not= 2 \end{cases}$
\item $\begin{cases}  H_0: \beta_1 + \beta_2 = 1  \\ H_a: \beta_1 + \beta_2 \not= 1 \end{cases}$
\end{enumerate}
\end{problem}

\begin{solution}
\end{solution}


\begin{problem}
 Пусть $y_i = \beta_1 + \beta_2 x_i + \e_i$ и $i = 1, \dots, 5$ --- классическая регрессионная модель. Также имеются следующие данные: $\sum_{i=1}^5 y_i^2 = 55, \sum_{i=1}^5 x_i^2 = 2, \sum_{i=1}^5 x_iy_i = 9, \sum_{i=1}^5 y_i = 15, \sum_{i=1}^5 x_i = 2.$ Используя их, найдите:

\begin{enumerate}
\item $\hat{\beta_1}$ и $\hat{\beta_2}$
\item $\Corr(\hat{\beta_1}, \hat{\beta_2})$
\item $TSS$
\item $ESS$
\item $RSS$
\item $R^2$
\item $\hat{\sigma}^2$
\end{enumerate}

Проверьте следующие гипотезы:
\begin{enumerate}
\item $\begin{cases}  H_0: \beta_2 = 2  \\ H_a: \beta_2 \not= 2 \end{cases}$
\item $\begin{cases}  H_0: \beta_1 + \beta_2 = 1  \\ H_a: \beta_1 + \beta_2 \not= 1 \end{cases}$
\end{enumerate}
\end{problem}
 
\begin{solution}
\end{solution}



\begin{problem}
 Рассмотрите классическую линейную регрессионную модель $y_i = \beta x_i + \e_i$. Найдите $\E \hat{\beta}$. Какие из следующих оценок параметра $\beta$ являются несмещенными:

\begin{enumerate}
\item $\hat{\beta} = \frac{y_1}{x_1}$
\item $\hat{\beta} = \frac{1}{2} \frac{y_1}{x_1} + \frac{1}{2} \frac{y_n}{x_n}$
\item $\hat{\beta} = \frac{1}{n}  \frac{y_1}{x_1} + \ldots + \frac{y_n}{x_n} $
\item $\hat{\beta} = \frac{\overline{y}}{\overline{x}}$
\item $\hat{\beta} = \frac{y_n - y_1}{x_n - x_1}$
\item $\hat{\beta} = \frac{1}{2} \frac{y_2 - y_1}{x_2 - x_1} + \frac{1}{2} \frac{y_n - y_{n-1}}{x_n - x_{n-1}}$
\item $\hat{\beta} = \frac{1}{n} \frac{y_2 - y_1}{x_2 - x_1} + \frac{1}{n} \frac{y_3 - y_2}{x_3 - x_2} + \ldots + \frac{1}{n} \frac{y_n - y_{n-1}}{x_n - x_{n-1}}$
\item $\hat{\beta} = \frac{1}{n-1}  \frac{y_2 - y_1}{x_2 - x_1} + \frac{y_3 - y_2}{x_3 - x_2} + \ldots + \frac{y_n - y_{n-1}}{x_n - x_{n-1}} $
\item $\hat{\beta} = \frac{x_1 y_1 + \ldots + x_n y_n}{x_1^2 + \ldots + x_n^2}$
\item $\hat{\beta} = \frac{1}{2} \frac{y_n - y_1}{x_n - x_1} + \frac{1}{2n}  \left( \frac{y_1}{x_1} + \ldots + \frac{y_n}{x_n} \right) $
\item $\hat{\beta} =  \frac{1}{2} \frac{y_n - y_1}{x_n - x_1} + \frac{1}{2} \frac{x_1 y_1 + \ldots + x_n y_n}{x_1^2 + \ldots + x_n^2}$
\item $\hat{\beta} = \frac{\sum_{i=1}^n (x_i - \overline{x})(y_i - \overline{y})}{\sum_{i=1}^n (x_i - \overline{x})^2}$
\item $\hat{\beta} = \frac{\sum_{i=1}^n (x_i - \overline{x})(\overline{y} - y_i)}{\sum_{i=1}^n (x_i - \overline{x})^2}$
\item $\hat{\beta} = \frac{y_1 + 2 y_2 + \ldots + n y_n}{x_1 + 2 x_2 + \ldots + n x_n}$
\item $\hat{\beta} = \frac{\sum_{i=1}^n i(y_i - \overline{y})}{\sum_{i=1}^n i(x_i - \overline{x})}$
\item $\hat{\beta} = \frac{1}{n} \sum_{i=1}^n \frac{y_i}{x_i}$
\item $\hat{\beta} = \frac{1}{n} \sum_{i=1}^n \frac{y_i - \overline{y}}{x_i - \overline{x}}$
\end{enumerate}
\end{problem}

\begin{solution}
\end{solution}



\begin{problem}
 Рассмотрите классическую линейную регрессионную модель $y_i = \beta x_i + \e_i$. Найдите $\Var(\hat{\beta})$.

\begin{enumerate}
\item $\hat{\beta} = \frac{y_1}{x_1}$
\item $\hat{\beta} = \frac{1}{2} \frac{y_1}{x_1} + \frac{1}{2} \frac{y_n}{x_n}$
\item $\hat{\beta} = \frac{1}{n} \left( \frac{y_1}{x_1} + \ldots + \frac{y_n}{x_n} \right)$
\item $\hat{\beta} = \frac{\overline{y}}{\overline{x}}$
\item $\hat{\beta} = \frac{y_n - y_1}{x_n - x_1}$
\item $\hat{\beta} = \frac{1}{2} \frac{y_2 - y_1}{x_2 - x_1} + \frac{1}{2} \frac{y_n - y_{n-1}}{x_n - x_{n-1}}$
\item $\hat{\beta} = \frac{x_1 y_1 + \ldots + x_n y_n}{x_1^2 + \ldots + x_n^2}$
\item $\hat{\beta} = \frac{\sum_{i=1}^n (x_i - \overline{x})(y_i - \overline{y})}{\sum_{i=1}^n (x_i - \bar{x})^2}$
\item $\hat{\beta} = \frac{\sum_{i=1}^n (x_i - \overline{x})(\overline{y} - y_i)}{\sum_{i=1}^n (x_i - \bar{x})^2}$
\item $\hat{\beta} = \frac{y_1 + 2 y_2 + \ldots + n y_n}{x_1 + 2 x_2 + \ldots + n x_n}$
\item $\hat{\beta} = \frac{\sum_{i=1}^n i(y_i - \overline{y})}{\sum_{i=1}^n i(x_i - \overline{x})}$
\item $\hat{\beta} = \frac{1}{n} \sum_{i=1}^n \frac{y_i}{x_i}$
\item $\hat{\beta} = \frac{1}{n} \sum_{i=1}^n \frac{y_i - \overline{y}}{x_i - \overline{x}}$
\end{enumerate}
\end{problem}

\begin{solution}
\end{solution}



\begin{problem}
 Рассмотрите классическую линейную регрессионную модель $y_i = \beta \cdot i + \e_i$, $i=1, \ldots, n$. Какая из оценок $\hat{\beta}$ и $\tilde{\beta}$ является более эффективной?

\begin{enumerate}
\item $\hat{\beta} = y_1$ и $\tilde{\beta} = y_2/2$
\item $\hat{\beta} = y_1$ и $\tilde{\beta} = \frac{1}{2} y_1 + \frac{1}{2} \frac{y_2}{2}$
\item $\hat{\beta} = \frac{1}{n} \left(  \frac{y_1}{1} + \ldots + \frac{y_n}{n} \right) $ и $\tilde{\beta} = \frac{1 \cdot y_1 + \ldots + n \cdot y_n}{1^2 + \ldots + n^2}$
\end{enumerate} 
\end{problem}

\begin{solution}
\end{solution}


\begin{problem} На основе 100 наблюдений была оценена функция спроса:
\[
\underset{(s.e.)}{\widehat{\ln Q}} = \underset{(0.04)}{0.87} - \underset{(0.02)}{1.23}\ln P
\]
Значимо ли коэффициент эластичности спроса по цене отличается от $-1$? Рассмотрите уровень значимости $5\%$.
\end{problem}

\begin{solution}
\end{solution}


\begin{problem}
 На основе 100 наблюдений была оценена функция спроса: 
\[
\underset{(s.e.)}{\widehat{\ln Q}} = \underset{(0.04)}{2.87} - \underset{(0.02)}{1.12}\ln P
\]
На уровне значимости $5\%$ проверьте гипотезу  $H_0: \beta_{\ln P} = - 1$ против альтернативной $H_a: \beta_{\ln P} < -1$. Дайте экономическую интерпретацию проверяемой гипотезе и альтернативе.
\end{problem}

\begin{solution}
\end{solution}


\begin{problem}
 Используя годовые данные с 1960 по 2005 г., была построена кривая Филлипса, связывающая уровень инфляции $Inf$ и уровень безработицы $Unem$: 
\[
\widehat{Inf} = 2.34 - 0.23Unem 
\]
\[
\sqrt{\widehat{Var}(\hat{\beta}_{Unem})} = 0.04, R^2 = 0.12
\]
На уровне значимости $1\%$ проверьте гипотезу  $H_0: \beta_{Unem} = 0$ против альтернативной $H_a: \beta_{Unem} \not= 0$.
\end{problem}

\begin{solution}
\end{solution}


\begin{problem}
 Пусть $y_i = \beta_1 + \beta_2 x_i + \e_i$ и $i = 1, \dots, 18$ --- классическая регрессионная модель, где $\E(\e_i)$ = 0, $Var(\e_i) = \sigma^2$. Также имеются следующие данные: $\sum_{i=1}^{18} y_i^2 = 4256, \sum_{i=1}^{18} x_i^2 = 185, \sum_{i=1}^{18} x_iy_i = 814.25, \sum_{i=1}^{18} y_i = 225, \sum_{i=1}^{18} x_i = 49.5.$ Используя эти данные, оцените эту регрессию и на уровне значимости $5\%$ проверьте гипотезу $H_0: \beta_1 = 3.5$ против альтернативной $H_a: \beta_1 > 3.5$:
\begin{enumerate}
\item Приведите формулу для тестовой статистики 
\item Укажите распределение тестовой статистики
\item Вычислите наблюдаемое значение тестовой статистики
\item Укажите границы области, где основная гипотеза не отвергается
\item Сделайте статистический вывод
\end{enumerate}
\end{problem}

\begin{solution}
\end{solution}


\begin{problem}
 Рассматривается модель $y_i=\mu+\e_i$, где $\E(\e_i)=0$, $\Var(\e_i)=\sigma^2$ и $\Cov(\e_i,\e_j)=0$ при $i\neq j$. При каких $c_i$ несмещенная оцека 
\[
\hat{\mu}=\sum_{i=1}^{n} c_i y_i
\]
имеет наименьшую дисперсию?
\end{problem}
 
\begin{solution}
Через теорему Гаусса--Маркова или через условную минимизацию, $c_i=1/n$
\end{solution}

\begin{problem}
 Рассмотрим классическую линейную регрессионную модель, $y_t=\b\cdot t+\e_t$. Какая из оценок, $\hb$ или $\hb'$ является более эффективной?
\begin{enumerate}
\item $\hb=y_1$, $\hb'=y_2/2$
\item $\hb=y_1$, $\hb'=0.5y_1+0.5\frac{y_2}{2}$
\item $\hb=\frac{1}{n}\left(y_1+\frac{y_2}{2}+\frac{y_3}{3}+\ldots+\frac{y_n}{n}\right)$, $\hb'=\frac{y_1+2y_2+\ldots+ny_n}{1^2+2^2+\ldots+n^2}$
\end{enumerate}
\end{problem}
 
\begin{solution}
\end{solution}


\begin{problem}
 Ошибки регрессии $\e_i$ независимы и равновероятно принимают значения $+1$ и $-1$. Также известно, что $y_i=\beta \cdot i +\e_i$. Модель оценивается всего по двум наблюдениям. 
\begin{enumerate}
\item Найдите закон распределения $\hb$, $RSS$, $ESS$, $TSS$, $R^2$
\item Найдите $\E(\hb)$, $\Var(\hb)$, $\E(RSS)$, $\E(ESS)$, $\E(R^2)$
\item При каком $\beta$ величина $\E(R^2)$ достигает максимума?
\end{enumerate}
\end{problem}
 
\begin{solution}
\end{solution}



\begin{problem}
 Рассмотрим модель с линейным трендом без свободного члена, $y_t=\beta t +\e_t$. 
\begin{enumerate}
\item Найдите МНК оценку коэффициента $\beta$
\item Рассчитайте $\E(\hat{\beta})$ и $\Var(\hat{\beta})$ в предположениях теоремы Гаусса-Маркова
\item Верно ли, что оценка $\hat{\beta}$ состоятельна?
\end{enumerate}
\end{problem}
 

\begin{solution}
\begin{enumerate}
\item $\hb=\frac{\sum y_t t}{\sum t^2}$
\item $\E(\hat{\beta})=\beta$ и $\Var(\hat{\beta})=\frac{\sigma^2}{\sum_{t=1}^{T} t^2}$
\item Да, состоятельна
\end{enumerate}
\end{solution}


\begin{problem}
 В модели $y_t=\beta_1+\beta_2 x_t+\e_t$, где 
$x_t=\left\{
\begin{array}{l}
2,\, t=1 \\
1,\, t>1
\end{array}
\right.
$:
\begin{enumerate}
\item Найдите мнк-оценку $\hb_2$
\item Рассчитайте $\E(\hb_2)$ и $\Var(\hb_2)$ в предположениях теоремы Гаусса-Маркова
\item Верно ли, что оценка $\hb_2$ состоятельна?
\end{enumerate}
\end{problem}

\begin{solution}
несостоятельна
\end{solution}


\begin{problem}
 В модели $y_t=\beta_1+\beta_2 x_t$, где 
$x_t=\left\{
\begin{array}{l}
1,\, t=2k+1 \\
0,\, t=2k
\end{array}
\right.
$:
\begin{enumerate}
\item Найдите мнк-оценку $\hb_2$
\item Рассчитайте $\E(\hb_2)$ и $\Var(\hb_2)$ в предположениях теоремы Гаусса-Маркова
\item Верно ли, что оценка $\hb_2$ состоятельна?
\end{enumerate}
\end{problem}

\begin{solution}
\end{solution}



\begin{problem}
 Априори известно, что парная регрессия должна проходить через точку $(x_{0},y_{0})$.
\begin{enumerate}
\item  Выведите формулы МНК оценок;
\item В предположениях теоремы Гаусса-Маркова найдите дисперсии и средние оценок 
\end{enumerate}
\end{problem}

\begin{solution}
Вроде бы равносильно переносу начала координат и применению результата для регрессии без свободного члена. Должна остаться несмещенность. 
\end{solution}



\begin{problem}
 Мы предполагаем, что $y_t$ растёт с линейным трендом, т.е. $y_t=\b_1+\b_2 t+\e_t$. Все предпосылки теоремы Гаусса-Маркова выполнены. В качестве оценки $\hb_2$ предлагается $\hb_2=\frac{Y_T-Y_1}{T-1}$, где $T$ --- общее количество наблюдений. 
\begin{enumerate}
\item Найдите $\E(\hb_2)$ и $\Var(\hb_2)$
\item Совпадает ли оценка $\hb_2$ с классической мнк-оценкой?
\item У какой оценки дисперсия выше, у $\hb_2$ или классической мнк-оценки?
\end{enumerate}
\end{problem}

\begin{solution}
\end{solution}


\begin{problem}
 Вася считает, что выборочная ковариация $\sCov(y,\hy)=\frac{\sum (y_i-\bar{y})(\hy_i-\bar{y})}{n-1}$ это неплохая оценка для $\Cov(y_i,\hy_i)$. Прав ли он?
\end{problem}

\begin{solution}
Не прав. Ковариация $\Cov(y_i,\hy_i)$ зависит от $i$, это не одно неизвестное число, для которого можно предложить одну оценку.
\end{solution}


\begin{problem}
 В классической линейной регрессионной модели $y_i=\beta_1+\beta_2 x_i+\e_i$, дисперсия зависимой переменной не зависит от номера наблюдения, $\Var(y_i)=\sigma^2$. Почему для оценки $\sigma^2$ вместо известной из курса математической статистики формулы $\sum (y_i-\bar{y})^2/(n-1)$ используют $\sum \he_i^2/(n-2)$?
\end{problem}

\begin{solution}
формула $\sum (y_i-\bar{y})^2/(n-1)$ неприменима так как $\E(y_i)$ не является константой 
\end{solution}



\begin{problem}
 Оценка регрессии имеет вид $\hy_i=3-2x_i$. Выборочная дисперсия $x$ равна $9$, выборочная дисперсия $y$ равна $40$. Найдите $R^2$ и выборочные корреляции $\sCorr(x,y)$, $\sCorr(y,\hy)$.
\end{problem}

\begin{solution}
$R^2$ --- это отношение выборочных дисперсий $\hy$ и $y$. 
\end{solution}



\begin{problem}
 Слитки-вариант. Перед нами два золотых слитка и весы, производящие взвешивания с ошибками. Взвесив первый слиток, мы получили результат $300$ грамм, взвесив второй слиток --- $200$ грамм, взвесив оба слитка --- $400$ грамм. Предположим, что ошибки взвешивания --- независимые одинаково распределенные случайные величины с нулевым средним. 
\begin{enumerate}
\item Найдите несмещеную оценку веса первого слитка, обладающую наименьшей дисперсией.
\item Как можно проинтерпретировать нулевое математическое ожидание ошибки взвешивания? 
\end{enumerate} 
\end{problem}

\begin{solution}
 Как отсутствие систематической ошибки.
\end{solution}



\begin{problem}
 Рассмотрим линейную модель $y_i=\beta_1+\beta_2 x_i +\e_i$, где ошибки $\e_i$ нормальны $N(0;\sigma^2)$ и независимы.
\begin{enumerate}
\item Верно ли, что $y_i$ одинаково распределены?
\item Верно ли, что $\bar{y}$ --- это несмещенная оценка для $\E(y_i)$?
\item Верно ли, что $\sum (y_i-\bar{y})^2/(n-1)$ --- несмещенная оценка для $\sigma^2$? Если да, то докажите, если нет, то определите величину смещения
\end{enumerate} 
\end{problem}

\begin{solution}
нет, нет, нет 
\end{solution}



\begin{problem}
 Модель регрессии $y_i = \beta_1 + \beta_2 x_i + \e_i$, в которой ошибки
$\e_i$ независимы и нормальны $N(0;\sigma^2)$, оценивается по 22 наблюдениям. Найдите $\E(RSS)$, $\Var(RSS)$, $\P(10\sigma^2<RSS<30\sigma^2)$, $\P(10\hat{\sigma}^2<RSS<30\hat{\sigma}^2)$
\end{problem}

\begin{solution}
 $RSS/\sigma^2\sim\chi^2_{n-k}$, $\E(RSS)=(n-k)\sigma^2$, $\Var(RSS)=2(n-k)\sigma^4$, $\P(10\sigma^2<RSS<30\sigma^2)\approx 0.898$ 
\end{solution}


\begin{problem}
 Модель регрессии $y_i = \beta_1 + \beta_2 x_i + \e_i$, в которой ошибки
$\e_i$ независимы и нормальны $N(0;\sigma^2)$, оценивается по 12 наблюдениям. Найдите
\begin{enumerate}
\item $\P(\hb_1>\beta_1)$, $\P(\beta_1>0)$, $\P(|\hb_1-\beta_1|<se(\hb_1))$, $\P(\hb_2>\beta_2+se(\hb_2))$, $\P(\hb_2>\beta_2-se(\hb_2))$
\item $\E(\hb_1)$, $\E(\hb_2)$, $\E(\beta_2)$
\item Закон распределения, математическое ожидание и дисперсию величин $\frac{\hb_2-\beta_2}{\sqrt{\Var(\hb_2)}}$, $\frac{\hb_2-\beta_2}{\sqrt{\widehat{\Var}(\hb_2)}}$, $\frac{\hb_1+\hb_2-\beta_1-\beta_2}{\sqrt{\widehat{\Var}(\hb_1+\hb_2)}}$
\item $\P(\hs>\sigma)$, $\P(\hs>2\sigma)$
\end{enumerate}

\end{problem}

\begin{solution}
\end{solution}



\begin{problem}
 Для модели парной регрессии известны $y=(1, 2, 3, 4, 5)'$ и $\hy=(2, 2, 2, 4, 5)'$. Найдите $RSS$, $TSS$, $R^2$, $\hs^2$.
\end{problem}

\begin{solution}
\end{solution}


\begin{problem}
 В классической парной регрессионной модели $y_i = \beta_1 + \beta_2 x_i + \e_i$ с нормально распределенными ошибками, оцениваемой по 30 наблюдениям, дополнительно известно, что $\Var(\e_7)=9$. Найдите
\begin{enumerate}
\item $\E(\e_2)$, $\Cov(\e_1,\e_3)$, $\E(\e_3^5)$, $\E(e_5^3)$, $\Var(e_5)$, $\Var(y_3)$
\item $\P(e_2>\e_3)$, $\P(e_1>0)$, $\P(e_1>3)$
\item $\E(RSS)$, $\Var(RSS)$, $\P(RSS>200)$
\end{enumerate}
\end{problem}

\begin{solution}
\end{solution}


\begin{problem}
В модели парной регрессии придумайте такие наблюдения, чтобы:
\begin{itemize}
\item $R^2=0.9$
\item $R^2=0.8$ и регрессия имела вид $\hy=2+3x$
\end{itemize}
\end{problem}

\begin{solution}
 Можно взять четыре наблюдения равноотстоящих по вертикали от данной прямой. Подбирая остатки, добиваемся нужного $R^2$. 
 \end{solution}



\begin{problem}
Оцененная с помощью линейной модели $y_t = \beta_1 + \beta_2 t + \e_t$ методом наименьших квадратов зависимость расходов на питание $y$ от времени, определённого как $t = 1$ для 1995 г., $t = 2$ для 1996 г., \ldots, $t = 12$ для 2006 г., задана уравнением $\hat{y}_t = 95 + 2.5 t$.

Чему были бы равны оценки коэффициентов $\beta_1$ и $\beta_2$, если бы в качестве $t$ использовались фактические даты (1995 – 2006), а не числа от 1 до 12? 
\end{problem}

\begin{solution}
$\hat{\beta_1} = -4890$ и $\hat{\beta_2} = 2.5$

$X = \begin{bmatrix}
1 & 1 \\
1 & 2 \\
\ldots & \ldots \\
1 & 12 \\
\end{bmatrix}$ --- матрица исходных регрессоров; $\tilde{X} = \begin{bmatrix}
1 & 1+1994\\
1 & 2+1994 \\
\ldots & \ldots \\
1 & 12+1994 \\
\end{bmatrix}$ --- матрица новых регрессоров.

$\tilde{X} = X \cdot D$, где $D = \begin{bmatrix}
1 & 1994 \\
0 & 1 \\
\end{bmatrix}$.

Итак, уравнение регрессии с новыми регрессорами имеет вид $y = \tilde{X}\beta + \e$ и МНК-оценки коэффициентов равны:
\begin{multline}
\hat{\beta} = \left( \tilde{X}^T \tilde{X} \right)^{-1} \tilde{X}^T y = \left( [XD]^T [XD] \right)^{-1} [XD]^T y = \\
D^{-1} (X^T X)^{-1} (D^T)^{-1} D^T X^T y = D^{-1} (X^T X)^{-1}X^T y 
\end{multline}
\[
\hat{\beta} = D^{-1}\hat{\beta}_{old} = \begin{bmatrix}
1 & -1994 \\
0 & 1 \\
\end{bmatrix} \begin{bmatrix}
95 \\
2.5 \\
\end{bmatrix} = \begin{bmatrix}
-4890 \\
2.5 \\
\end{bmatrix}
\]
\end{solution}

\begin{problem}
Пусть есть набор данных $(x_i, y_i)$, $i = 1, \ldots, n$, $(x_i>0, y_i>0)$, порожденных уравнением $y_i = \beta_1 + \beta_2 x_i + \e_i$, удовлетворяющих условиям стандартной модели парной регрессии.
Рассматриваются следующие оценки параметра $\beta_2$:
$$\tilde{\beta}_2^a = \frac{1}{n}\sum_{i=1}^n \frac{y_i}{x_i}, \text{ } \tilde{\beta}_2^b = \frac{\overline{y}}{\overline{x}}$$
Найти дисперсию и смещение каждой из оценок.
\end{problem}

\begin{solution}
Мы можем существенно упростить решение, воспользовавшись матричным представлением:

\begin{multline}
\tilde{\beta}_2^a = \frac{1}{n}\sum_{i=1}^n \frac{y_i}{x_i} = \frac{1}{n} \begin{bmatrix}
\frac{1}{x_1} & \frac{1}{x_2} & \ldots & \frac{1}{x_n}
\end{bmatrix} \begin{bmatrix}
y_1\\
y_2\\
\vdots\\
y_n\\
\end{bmatrix} = \\
 \frac{1}{n} \begin{bmatrix}
\frac{1}{x_1} & \frac{1}{x_2} & \ldots & \frac{1}{x_n}
\end{bmatrix} y
\end{multline}

\begin{multline}
\E\tilde{\beta}_2^a = \frac{1}{n}\sum_{i=1}^n \frac{\E y_i}{x_i} = \frac{1}{n} \begin{bmatrix}
\frac{1}{x_1} & \frac{1}{x_2} & \ldots & \frac{1}{x_n}
\end{bmatrix} \begin{bmatrix}
\E y_1\\
\E y_2\\
\vdots\\
\E y_n\\
\end{bmatrix} = \\
 \frac{1}{n} \begin{bmatrix}
\frac{1}{x_1} & \frac{1}{x_2} & \ldots & \frac{1}{x_n}
\end{bmatrix} \begin{bmatrix}
\beta_1 + \beta_2 x_1\\
\beta_1 + \beta_2 x_2\\
\vdots\\
\beta_1 + \beta_2 x_n\\  
\end{bmatrix} = \\
\frac{1}{n} \begin{bmatrix}
\frac{1}{x_1} & \frac{1}{x_2} & \ldots & \frac{1}{x_n}
\end{bmatrix} \left[ \beta_1 \begin{bmatrix}
1\\
1\\
\vdots\\
1\\
\end{bmatrix} + \beta_2 \begin{bmatrix}
x_1\\
x_2\\
\vdots\\
x_n\\
\end{bmatrix} \right] = \frac{\beta_1}{n}\sum_{k=1}^n \frac{1}{x_k} + \beta_2
\end{multline}

Значит, смещение для первой оценки равно $\frac{\beta_1}{n}\sum_{k=1}^n \frac{1}{x_k}$.

\begin{multline}
\Var(\tilde{\beta}_2^a) = \Var\left(\frac{1}{n} \begin{bmatrix}
\frac{1}{x_1} & \frac{1}{x_2} & \ldots & \frac{1}{x_n}
\end{bmatrix} y\right) =\\
 \frac{1}{n^2} \begin{bmatrix}
\frac{1}{x_1} & \frac{1}{x_2} & \ldots & \frac{1}{x_n}
\end{bmatrix} \Var(y) \begin{bmatrix}
\frac{1}{x_1} & \frac{1}{x_2} & \ldots & \frac{1}{x_n}
\end{bmatrix}^T =\\
\frac{1}{n^2} \begin{bmatrix}
\frac{1}{x_1} & \frac{1}{x_2} & \ldots & \frac{1}{x_n}
\end{bmatrix} \Var(\e) \begin{bmatrix}
\frac{1}{x_1} & \frac{1}{x_2} & \ldots & \frac{1}{x_n}
\end{bmatrix}^T = \\
 \frac{1}{n^2} \begin{bmatrix}
\frac{1}{x_1} & \frac{1}{x_2} & \ldots & \frac{1}{x_n}
\end{bmatrix} \sigma_{\e}^2 I \begin{bmatrix}
\frac{1}{x_1} & \frac{1}{x_2} & \ldots & \frac{1}{x_n}
\end{bmatrix}^T =\\
\frac{\sigma^2_{\e}}{n^2}\sum_{k=1}^n \frac{1}{x_k^2}
\end{multline}

Перейдём ко второй оценке.

$\tilde{\beta}_2^b = \frac{\overline{y}}{\overline{x}} = \frac{1}{\overline{x}} \frac{1}{n} \begin{bmatrix}
1 & 1 & \ldots & 1
\end{bmatrix} y$


\begin{multline}
\E\tilde{\beta}_2^b = \frac{\overline{y}}{\overline{x}} = \frac{1}{\overline{x}} \frac{1}{n} \begin{bmatrix}
1 & 1 & \ldots & 1
\end{bmatrix} \E y = \frac{1}{\overline{x}} \frac{1}{n} \begin{bmatrix}
1 & 1 & \ldots & 1
\end{bmatrix} \begin{bmatrix}
\beta_1 + \beta_2 x_1\\
\beta_1 + \beta_2 x_2\\
\vdots\\
\beta_1 + \beta_2 x_n\\  
\end{bmatrix} =\\
\frac{1}{\overline{x}} \frac{1}{n} \begin{bmatrix}
1 & 1 & \ldots & 1
\end{bmatrix} \left[ \beta_1 \begin{bmatrix}
1\\
1\\
\vdots\\
1\\
\end{bmatrix} + \beta_2 \begin{bmatrix}
x_1\\
x_2\\
\vdots\\
x_n\\
\end{bmatrix} \right] = \\
\frac{1}{n} \frac{\beta_1 n}{\overline{x}} + \frac{1}{n} \frac{\beta_2 \sum x_i}{\overline{x}} = \frac{\beta_1}{\overline{x}} + \beta_2
\end{multline}

Значит, смещение равно $\frac{\beta_1}{\overline{x}}$.

\begin{multline}
\Var(\tilde{\beta}_2^b) = \frac{1}{\overline{x}^2} \frac{1}{n^2} \begin{bmatrix}
1 & 1 & \ldots & 1
\end{bmatrix} \Var(y) \begin{bmatrix}
1\\
1\\
\vdots\\
1\\
\end{bmatrix} = \\
\frac{1}{\overline{x}^2} \frac{1}{n^2} \begin{bmatrix}
1 & 1 & \ldots & 1
\end{bmatrix} \Var(\e) \begin{bmatrix}
1\\
1\\
\vdots\\
1\\
\end{bmatrix} = \frac{\sigma_{\e}^2}{\overline{x}^2 n}
\end{multline}
\end{solution}



\begin{problem}
Уравнение $y_t = \beta_1 + \beta_2 x_t + \e_t$ оценивается по МНК. Может ли коэффициент детерминации быть малым (<0.05), а статистика $t_{\hat{\beta}_2}$ большой (>10)?
\end{problem}

\begin{solution}
Известно, что для парной регрессии $t_{\hat{\beta}_2}^2 = \frac{R^2}{(1 - R^2)/(n-2)}$. Поэтому из выражения $t_{\hat{\beta}_2}^2 = \frac{0.05^2}{(1 - 0.05^2)/(n-2)} = \frac{0.05^2 (n-2)}{1 - 0.05^2}$ становится очевидным, что при надлежащем выборе числа наблюдений можно сделать величину $t_{\hat{\beta}_2}$ сколь угодно большой.
\end{solution}


\begin{problem}
 Докажите, что в случае, когда $|\sCorr(x, y)| = 1$, линия парной регрессии $y$ на $x$ совпадает с линией парной регрессии $x$ на $y$.
\end{problem}

\begin{solution}
Пусть $Y_i = \beta_1 + \beta_2 X_i + \e_i$, $i = 1, \ldots, n$.

Тогда $Y_i = \hat{\beta}_1 + \hat{\beta}_2 X_i + \hat{\e}_i$

$Y_i -  \overline{Y} + \overline{Y} = \hat{\beta}_1 + \hat{\beta}_2 (X_i -  \overline{X} + \overline{X}) + \hat{\e}_i$

$Y_i -  \overline{Y}  = \underbrace{\hat{\beta}_1 - \overline{Y} + \hat{\beta}_2 \overline{X}}_{=0} + \hat{\beta}_2 (X_i -  \overline{X}) + \hat{\e}_i$

$Y_i -  \overline{Y} = \hat{\beta}_2 (X_i -  \overline{X}) + \hat{\e}_i$

$y_i \equiv Y_i -  \overline{Y}$, $i = 1, \ldots, n$

$x_i \equiv X_i -  \overline{X}$, $i = 1, \ldots, n$

$y_i = \hat{\beta}_2 x_i + \hat{\e}_i$

$\textbf{y} = \hat{\beta}_2 \textbf{x} + \hat{\e}$, где $\textbf{y} = \begin{bmatrix}
y_1 & \ldots & y_n
\end{bmatrix}^T$, $\textbf{x} = \begin{bmatrix}
x_1 & \ldots & x_n
\end{bmatrix}^T$, $\e = \begin{bmatrix}
\e_1 & \ldots & \e_n
\end{bmatrix}^T$

$\textbf{x}^T \textbf{y} = \hat{\beta}_2 \textbf{x}^T \textbf{x} + \underbrace{\textbf{x}^T \hat{\e}}_{=0}$

\begin{equation}
\label{task20:direct_ols}\hat{\beta}_2 = \frac{\textbf{x}^T \textbf{y}}{\textbf{x}^T \textbf{x}}
\end{equation}

Аналогично получаем, что в обратной регрессии $X_i = \beta_3 + \beta_4 Y_i + \xi_i$, $i = 1, \ldots, n$

\begin{equation}
\label{task20:reverse_ols}\hat{\beta}_4 = \frac{\textbf{y}^T \textbf{y}}{\textbf{y}^T \textbf{y}}
\end{equation}

$ESS = (\hat{Y} - \overline{Y}_i)^T(\hat{Y} - \overline{Y}_i)$

Заметим, что $\hat{Y} - \overline{Y}_i = (I - \pi)(\hat{Y} - \overline{Y}_i)$.

Действительно, $(I - \pi)(P - \pi) = P - \pi$, следовательно,

$\hat{Y} - \overline{Y}_i = (P - \pi)Y = (I - \pi)(P - \pi)Y = (I-\pi)(\hat{Y} - \overline{Y}_i)$.

Далее, $\hat{Y} - \overline{Y}_i = (I - \pi)(\hat{Y} - \overline{Y}_i) = (I - \pi)(\hat{\beta}_1 + \hat{\beta}_2 X - \overline{Y}_i) = \hat{\beta}_2 \textbf{x}$

Значит, $ESS = \hat{\beta}_2^2 \textbf{x}^T \textbf{x}$.

Получаем:
\begin{equation}
\label{task20:corr}R^2 = \frac{ESS}{TSS} = \frac{\hat{\beta}_2^2 \textbf{x}^T \textbf{x}^{(2)}}{\textbf{y}^T \textbf{y}} = \frac{\textbf{x}^T \textbf{y}^{(2)}}{(\textbf{x}^T \textbf{x})(\textbf{y}^T \textbf{y})} = \Corr^2(X, Y)
\end{equation}

Заметим также, что из формул (\ref{task20:direct_ols}), (\ref{task20:reverse_ols}) и (\ref{task20:corr}) следует, что $R^2 = \hat{\beta}_2 \hat{\beta}_4$.

Если $\Corr^2(X, Y) = 1$, то $R^2 = \hat{\beta}_2 \hat{\beta}_4 = 1$.

Отметим также, что из $R^2 = 1$ следует, что $\hat{\e}_1 = \ldots = \hat{\e}_n = 0$ и $\hat{\xi}_1 = \ldots = \hat{\xi}_n = 0$.

Тогда $Y_i = \hat{\beta}_1 + \hat{\beta}_2 X_i + \underbrace{\hat{\e}_i}_{=0}$ и $X_i = \hat{\beta_3} + \hat{\beta_4} Y_i + \underbrace{\hat{\xi}_i}_{=0}$, $i = 1, \ldots, n$.

$X_i = \hat{\beta_3} + \hat{\beta_4} Y_i = (\overline{X} - \hat{\beta_4}\overline{Y}) + \hat{\beta_4} Y_i = \left( \overline{X} - \frac{1}{\hat{\beta}_2} \overline{Y} \right) + \frac{1}{\hat{\beta}_2} Y_i$

$\hat{\beta}_2 X_i = (\hat{\beta}_2 \overline{X} - \overline{Y}) + Y_i$

$Y_i = (\overline{Y} - \hat{\beta}_2 \overline{X}) + \hat{\beta}_2 X_i = \hat{\beta}_1 + \hat{\beta}_2 X_i$

Следовательно, в случае когда $\Corr^2(X, Y) = 1$, линия парной регрессии $Y$ на $X$ совпадает с линией парной регрессии $X$ на $Y$.
\end{solution}



\begin{problem}
Сгенерите выборку из двух зависимых но некоррелированных случайных величин. Можно ли <<поймать>> зависимость используя парную регрессию?
\end{problem}

\begin{solution}
Да, если строить регрессию функции от $y$ на функцию от $x$. А если строить регрессию просто $y$ на $x$, то оценка наклона будет распределена симметрично около нуля.
\end{solution}


\begin{problem}
Все предпосылки классической линейной модели выполнены, $y=\beta_1+\beta_2 x+\varepsilon$. Рассмотрим альтернативную оценку коэффициента $\beta_2$,
\begin{equation}
\hb_{2,IV}=\frac{\sum z_i(y_i-\bar{y})}{\sum z_i(x_i-\bar{x})}
\end{equation}
\begin{enumerate}
\item Является ли оценка несмещенной?
\item Любые ли $z_i$ можно брать?
\item Найдите $\Var(\hb_{2,IV})$
\end{enumerate}
\end{problem}

\begin{solution}
Да, является. Любые, кроме констант. $\Var(\hb_{2,IV})=\sigma^2 \sum (z_i-\bar{z})^2/ \left(\sum (z_i-\bar{z})x_i \right)^2 $.
\end{solution}

\begin{problem}
Напишите формулу для оценок коэффициентов в парной регрессии без матриц. Напишите формулу для дисперсий оценок коэффициентов.
\end{problem}

\begin{solution}
\end{solution}


\begin{problem}
Рассматривается модель линейной регрессии $y_i = \beta_1 + \beta_2x_i + \varepsilon_i$, в которой ошибки $\varepsilon_1, \ldots, \varepsilon_n$ являются независимыми нормально распределенными случайными величинами с математическим ожиданием $0$ и дисперсией $\sigma^2$. Найдите

\begin{enumerate}
  \item $\mathbb{P}\{\varepsilon_1 > 0\}$, 
  \item $\mathbb{P}\{\varepsilon_1^2 + \varepsilon_2^2 > 2\sigma^2\}$,
  \item $\mathbb{P}\left\{\frac{\varepsilon_1}{\sqrt{\varepsilon_2^2 + \varepsilon_3^2}} > 2 \right\}$,	
  \item $\mathbb{P}\left\{\frac{\varepsilon_1}{\sqrt{\varepsilon_2^2 + \varepsilon_3^2 + \varepsilon_4^2}} > \frac{5}{4\sqrt{3}} \right\}$,
  \item $\mathbb{P}\left\{\frac{\varepsilon_1 + 2\varepsilon_2}{\sqrt{\varepsilon_3^2 + \varepsilon_4^2 + \varepsilon_5^2}} < \frac{9}{2} \right\}$,
  \item $\mathbb{P}\left\{\frac{\varepsilon_1^2}{\varepsilon_2^2 + \varepsilon_3^2} > 17 \right\}$.
\end{enumerate}
\end{problem}

\begin{solution}
Вспомните про $t$, $\chi^2$, $F$ распределения
\end{solution}

%%%%%%%%%%%%%%%%%%%%%%%%%%%%%
%%%% новые задачи

\begin{problem}
В модели парной регрессии $y_i=\beta_1+\beta_2 x_i +\e_i$ ошибки $\e_i$ независимы и имеют пуассоновское распределение с параметром $\lambda$.
\begin{enumerate}
\item Предложите способ несмещенно оценить $\lambda$.
\item Являются ли МНК-оценки $\hb_1$ и $\hb_2$ несмещенными? Если оценки являются смещенными, то предложите несмещенные оценки
\end{enumerate}
\end{problem}


\begin{solution}
$\hat{\lambda}=RSS/(n-2)$ т.к. $\Var(\e_i)=\lambda$. Оценка $\hb_2$ является несмещенной, но $\E(\hb_1)=\beta_1+\lambda$. Можно предложить несмещенную оценку $\hb'_1=\hb_1-RSS/(n-2)$.
\end{solution}
































