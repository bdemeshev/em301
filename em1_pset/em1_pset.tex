\documentclass[pdftex,12pt,a4paper]{article}

\input{/home/boris/science/tex_general/title_bor_utf8}

% чисто эконометрические сокращения:
\def \hb{\hat{\beta}}
\def \hy{\hat{y}}
\def \hY{\hat{Y}}
\def \he{\hat{\varepsilon}}

% временное решение
\newcommand{\solution}[1]{ {\tiny #1} }
\newcommand{\problem}[1]{#1}

\title{Задачник по эконометрике-1 \\ {\small (с шахматами и поэтэссами)}}
\author{Дмитрий Борзых, Борис Демешев}
\date{\today}

\makeindex % команда для создания предметного указателя
\bibliographystyle{plain} % стиль оформления ссылок


\begin{document}

\maketitle % печатаем заголовок


\parindent=0 pt % отступ равен 0

\section{Неклассифицировано}

\begin{enumerate}
\item Регрессионная модель   задана в матричном виде при помощи уравнения $y=X\beta+\varepsilon$, где $\beta=(\beta_1,\beta_2,\beta_3)'$.
Известно, что $\E(\varepsilon)=0$  и  $\Var(\varepsilon)=\sigma^2\cdot I$.
Известно также, что $y=$, $X=$.
Для удобства расчетов ниже приведены матрицы 
 $X'X=$ и $(X'X)^{-1}=$.

\begin{enumerate}
\item Укажите число наблюдений.
\item Укажите число регрессоров с учетом свободного члена.
\item Рассчитайте $TSS=\sum (y_i-\bar{y})^2$, $RSS=\sum (y_i-\hat{y}_i)^2$ и $ESS=\sum (\hat{y}_i-\bar{y})^2$.
\item Рассчитайте при помощи метода наименьших квадратов $\hb$, оценку для вектора неизвестных коэффициентов.
\item Чему равен $\he_5$, МНК-остаток регрессии, соответствующий 5-ому наблюдению?
\item Чему равен $R^2$  в модели? Прокомментируйте полученное значение с точки зрения качества оцененного уравнения регрессии.
\item Используя приведенные выше данные, рассчитайте несмещенную оценку для неизвестного параметра $\sigma^2$ регрессионной модели.
\item Рассчитайте $\widehat{\Cov}(\hb)$, оценку для ковариационной матрицы вектора МНК-коэффициентов $\hb$.  
\item Найдите $\widehat{\Var}(\hb_1)$, несмещенную оценку дисперсии МНК-коэффициента $\hb_1$.
\item Найдите $\widehat{\Var}(\hb_2)$, несмещенную оценку дисперсии МНК-коэффициента $\hb_2$.
\item Найдите $\widehat{\Cov}(\hb_1,\hb_2)$, несмещенную оценку ковариации МНК-коэффициентов $\hb_1$ и $\hb_2$.
\item Найдите $\widehat{\Var}(\hb_1+\hb_2)$, $\widehat{\Var}(\hb_1-\hb_2)$, $\widehat{\Var}(\hb_1+\hb_2+\hb_3)$, $\widehat{\Var}(\hb_1+\hb_2-2\hb_3)$
\item Найдите $\Corr(\hb_1,\hb_2)$, оценку коэффициента корреляции МНК-коэффициентов $\hb_1$ и $\hb_2$.
\item Найдите $s_{\hb_1}$, стандартную ошибку МНК-коэффициента $\hb_1$.
\end{enumerate}

\item Априори известно, что парная регрессия должна проходить через точку $(x_{0},y_{0})$.
\begin{enumerate}
\item  Выведите формулы МНК оценок;
\item В предположениях теоремы Гаусса-Маркова найдите дисперсии и средние оценок 
\end{enumerate}

\solution{Вроде бы равносильно переносу начала координат и применению результата для регрессии без свободного члена. Должна остаться несмещенность. }




\item \problem{ Слитки-вариант. Перед нами два золотых слитка и весы, производящие взвешивания с ошибками. Взвесив первый слиток, мы получили результат $300$ грамм, взвесив второй слиток --- $200$ грамм, взвесив оба слитка --- $400$ грамм. Предположим, что ошибки взвешивания --- независимые одинаково распределенные случайные величины с нулевым средним. 
\begin{enumerate}
\item Найдите несмещеную оценку веса первого шара, обладающую наименьшей дисперсией.
\item Как можно проинтерпретировать нулевое математическое ожидание ошибки взвешивания? 
\end{enumerate} }
\solution{ Как отсутствие систематической ошибки.} 

\item Вася считает, что $\sCov(y,\hy)=\frac{\sum (y_i-\bar{y})(\hy_i-\bar{y})}{\sqrt{\sum (y_i-\bar{y})^2 \sum (\hy_i-\bar{y})^2}}$ это неплохая оценка для $\Cov(y_i,\hy_i)$. Прав ли он?
\solution{Не прав. Ковариация $\Cov(y_i,\hy_i)$ зависит от $i$, это не одно неизвестное число, для которого можно предложить одну оценку.}


\item Сгенерировать набор данных, обладающий следующим свойством. Если попытаться сразу выкинуть регрессоры $x$ и $z$, то гипотеза о их совместной незначимости отвергается. Если вместо этого попытаться выкинуть отдельно $x$, или отдельно $z$, то гипотеза о незначимости не отвергается.
\solution{Сгенерировать сильно коррелированные $x$ и $z$}


\item Сгенерировать набор данных, обладающий следующим свойством. Если попытаться сразу выкинуть регрессоры $x$ и $z$, то гипотеза о их совместной незначимости отвергается. Если вместо сначала выкинуть отдельно $x$, то гипотеза о незначимости не отвергается. Если затем выкинуть $z$, то гипотезы о незначимости тоже не отвергается.
\solution{??}


\end{enumerate}

\section{МНК без матриц и вероятностей}

\begin{enumerate}
\item \problem{Даны $n$ пар чисел: $(x_1, y_1)$, \ldots, $(x_n,y_n)$. Мы прогнозируем $y_i$ по формуле $\hy_i=\hb x_i$. Найдите $\hb$ методом наименьших квадратов. }
\solution{$\hb=\sum x_i y_i/\sum x_i^2$}

\item \problem{Даны $n$ чисел: $y_1$, \ldots, $y_n$. Мы прогнозируем $y_i$ по формуле $\hy_i=\hb$. Найдите $\hb$ методом наименьших квадратов. }
\solution{$\hb=\bar{y}$}

\item \problem{Даны $n$ пар чисел: $(x_1, y_1)$, \ldots, $(x_n,y_n)$. Мы прогнозируем $y_i$ по формуле $\hy_i=\hb_1+\hb_2 x_i$. Найдите $\hb_1$ и $\hb_2$ методом наименьших квадратов. }
\solution{$\hb_2=\sum (x_i-\bar{x})(y_i-\bar{y})/\sum(x_i-\bar{x})^2$, $\hb_1=\bar{y}-\hb_2\bar{x}$}

\item \problem{Даны $n$ пар чисел: $(x_1, y_1)$, \ldots, $(x_n,y_n)$. Мы прогнозируем $y_i$ по формуле $\hy_i=1+\hb x_i$. Найдите $\hb$ методом наименьших квадратов. }
\solution{$\hb=\sum x_i (y_i-1)/\sum x_i^2$}

\item \problem{ Перед нами два золотых слитка и весы, производящие взвешивания с ошибками. Взвесив первый слиток, мы получили результат $300$ грамм, взвесив второй слиток --- $200$ грамм, взвесив оба слитка --- $400$ грамм. Оцените вес каждого слитка методом наименьших квадратов.}
\solution{ $(300-\hb_1)^2+(200-\hb_2)^2+(400-\hb_1-\hb_2)^2\to\min$ }


\item Аня и Настя утверждают, что лектор опоздал на 10 минут. Таня считает, что лектор опоздал на 3 минуты. С помощью мнк оцените на сколько опоздал лектор. 
\solution{ $2\cdot (10-\hb)^2+(3-\hb)^2\to\min$ }

\item Регрессия на дамми-переменную...



\item Функция $f(x)$ дифференциируема на отрезке $[0;1]$. Найдите аналог МНК-оценок для регрессии без свободного члена в непрерывном случае. Более подробно: найдите минимум по $\hb$ для функции
\begin{equation}
Q(\hb)= \int_0^1 (f(x)-\hb x)^2\,dx
\end{equation}
\solution{}

\item Есть двести наблюдений. Вовочка оценил модель $\hy=\hb_1+\hb_2 x$ по первой сотне наблюдений. Петечка оценил модель $\hy=\hat{\gamma}_1+\hat{\gamma}_2 x$ по второй сотне наблюдений. Машенька оценила модель $\hy=\hat{m}_1+\hat{m}_2 x$ по всем наблюдениям.
\begin{enumerate}
\item Возможно ли, что $\hb_2>0$, $\hat{\gamma}_2>0$, но $\hat{m}_2<0$?
\item Возможно ли, что $\hb_1>0$, $\hat{\gamma}_1>0$, но $\hat{m}_1<0$?
\item Возможно ли одновременное выполнение всех упомянутых условий?
\end{enumerate}
\solution{да, возможно. Два вытянутых облачка точек. Первое облачко даёт первую регрессию, второе --- вторую. Прямая, соединяющая центры облачков, --- общую.}



\item Вася оценил модель $y=\beta_1+\beta_2 d+\beta_3 x+\varepsilon$. Дамми-переменная $d$ обозначает пол, 1 для мужчин и 0 для женщин. Оказалось, что $\hat{\beta}_2>0$. Означает ли это, что для мужчин $\bar{y}$ больше, чем $\bar{y}$ для женщин?
\solution{Нет. Коэффициенты можно интепретировать только <<при прочих равных>>, т.е. при равных $x$. Из-за разных $x$ может оказаться, что у мужчин $\bar{y}$ меньше, чем $\bar{y}$ для женщин.}




\end{enumerate}

\section{Инструментальные переменные}

\begin{enumerate}
\item Табличка 2 на 2. Найдите $\E(\varepsilon)$, $\E(\varepsilon|x)$, $\Cov(\varepsilon,x)$.
\solution{}

\item Все предпосылки классической линейной модели выполнены, $y=\beta_1+\beta_2 x+\varepsilon$. Рассмотрим альтернативную оценку коэффициента $\beta_2$,
\begin{equation}
\hb_{2,IV}=\frac{\sum z_i(y_i-\bar{y})}{\sum z_i(x_i-\bar{x})}
\end{equation}
\begin{enumerate}
\item Является ли оценка несмещенной?
\item Любые ли $z_i$ можно брать?
\item Найдите $\Var(\hb_{2,IV})$
\end{enumerate}
\solution{Да, является. Любые, кроме констант. $\Var(\hb_{2,IV})=\sigma^2 \sum (z_i-\bar{z})^2/ \left(\sum (z_i-\bar{z})x_i \right)^2 $.}

\item 
\end{enumerate}


\section{Проекция, Картинка}
\begin{enumerate}
\item Найдите на Картинке четыре прямоугольных треугольника. Сформулируйте четыре теоремы Пифагора.
\solution{$\sum y_i^2=\sum \hy_i^2+\sum \he_i^2$, $TSS=ESS+RSS$, }

\item Покажите на Картинке TSS, ESS, RSS, $R^2$, $\sCov(\hy,y)$
\solution{}


\item Предложите аналог $R^2$ для случая, когда константа среди регрессоров отсутствует. Аналог должен быть всегда в диапазоне $[0;1]$, совпадать с обычным $R^2$, когда среди регрессоров есть константа, равняться единице в случае нулевого $\he$.
\solution{Спроецируем единичный столбец на <<плоскость>>, обозначим его $1'$. Делаем проекцию $y$ на <<плоскость>> и на $1'$. Далее аналогично. }

\item Вася оценил регрессию $y$ на константу, $x$ и $z$. А затем, делать ему нечего, регрессию $y$ на константу и полученный $\hy$. Какие оценки коэффициентов у него получатся? Чему будет равна оценка дисперсии коэффицента при $\hy$? Почему оценка коэффициента неслучайна, а оценка её дисперсии положительна?
\solution{Проекция $y$ на $\hy$ это $\hy$, поэтому оценки коэффициентов будут 0 и 1. Оценка дисперсии $\frac{RSS}{(n-2)ESS}$. Нарушены предпосылки теоремы Гаусса-Маркова, например, ошибки новой модели в сумме дают 0, значит коррелированы. } 


\end{enumerate}


\section{Голая линейная алгебра}

Здесь будет собран минимум задач по линейной алгебре.

\begin{enumerate}
\item \problem{Приведите пример таких $A$ и $B$, что $\det(AB)\neq \det(BA)$.}
\solution{Например, $A=(1,2,3)$, $B=(1,0,1)'$}


 
\end{enumerate}


\section{Компьютерные упражнения}

\begin{enumerate}


\item Скачайте результаты двух контрольных работ по теории вероятностей, \url{} с описанием данных, \url{}. Скачайте табличку соответствия имени и пола, \url{}. Наша задача попытаться предсказать результат второй контрольной работы зная позадачный результат первой контрольной. 
\begin{enumerate}
\item Какая задача из первой контрольной работы наиболее существенно влияет на результат второй контрольной?
\item Влияет ли пол на результат второй контрольной?
\item Влияет ли редкость имени на результат второй контрольной?
\item Что можно сказать про влияние группы, в которой учится студент?
\end{enumerate}

\item Задача Макар-Лиманова. У торговца 55 пустых стаканчиков, разложенных в несколько стопок. Пока нет покупателей он развлекается: берет верхний стаканчик из каждой стопки и формирует из них новую стопку. Потом снова берет верхний стаканчик из каждой стопки и формирует из них новую стопку и т.д.
\begin{enumerate}
\item Напишите функцию `makar\_step`. На вход функции подаётся вектор количества стаканчиков в каждой стопке до перекладывания. На выходе функция возвращает количества стаканчиков в каждой стопке после одного перекладывания.
\item Изначально стаканчики были разложены в две стопки, из 25 и 30 стаканчиков. Как разложатся стаканчики если покупателей не будет достаточно долго?
\end{enumerate}

\end{enumerate}



\end{document}