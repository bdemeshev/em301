\documentclass[pdftex,12pt,a4paper]{article}

\input{/home/boris/science/tex_general/title_bor_utf8}

% чисто эконометрические сокращения:
\def \hb{\hat{\beta}}
\def \hy{\hat{y}}
\def \hY{\hat{Y}}
\def \he{\hat{\varepsilon}}

% временное решение
\newcommand{\solution}[1]{ {\tiny #1} }
\newcommand{\problem}[1]{#1}

\title{Задачник по эконометрике-1 \\ {\small (с шахматами и поэтэссами)}}
\author{Дмитрий Борзых, Борис Демешев}
\date{\today}

\makeindex % команда для создания предметного указателя
\bibliographystyle{plain} % стиль оформления ссылок


\begin{document}

\maketitle % печатаем заголовок


\parindent=0 pt % отступ равен 0



\begin{enumerate}
\item Регрессионная модель   задана в матричном виде при помощи уравнения $y=X\beta+\varepsilon$, где $\beta=$.
Известно, что $\E(\varepsilon)=0$  и  $\Var(\varepsilon)=\sigma^2\cdot I$.
Известно также, что $y=$, $X=$.
Для удобства расчетов ниже приведены матрицы 
 $X'X=$ и $(X'X)^{-1}=$.

\begin{enumerate}
\item Укажите число наблюдений.
\item Укажите число регрессоров с учетом свободного члена.
\item Рассчитайте $TSS=\sum (y_i-\bar{y})^2$, $RSS=\sum (y_i-\hat{y}_i)^2$ и $ESS=\sum (\hat{y}_i-\bar{y})^2$.
\item Рассчитайте при помощи метода наименьших квадратов $\hb$, оценку для вектора неизвестных коэффициентов.
\item Чему равен $\he_5$, МНК-остаток регрессии, соответствующий 5-ому наблюдению?
\item Чему равен $R^2$  в модели? Прокомментируйте полученное значение с точки зрения качества оцененного уравнения регрессии.
\item Используя приведенные выше данные, рассчитайте несмещенную оценку для неизвестного параметра $\sigma^2$ регрессионной модели.
\item Рассчитайте $\widehat{\Cov}(\hb)$, оценку для ковариационной матрицы вектора МНК-коэффициентов $\hb$.  
\item Найдите $\widehat{\Var}(\hb_1)$, несмещенную оценку дисперсии МНК-коэффициента $\hb_1$.
\item Найдите $\widehat{\Var}(\hb_2)$, несмещенную оценку дисперсии МНК-коэффициента $\hb_2$.
\item Найдите $\widehat{\Cov}(\hb_1,\hb_2)$, несмещенную оценку ковариации МНК-коэффициентов $\hb_1$ и $\hb_2$.
\item Найдите $\widehat{\Var}(\hb_1+\hb_2)$, $\widehat{\Var}(\hb_1-\hb_2)$, $\widehat{\Var}(\hb_1+\hb_2+\hb_3)$, $\widehat{\Var}(\hb_1+\hb_2-2\hb_3)$
\item Найдите $\Corr(\hb_1,\hb_2)$, оценку коэффициента корреляции МНК-коэффициентов $\hb_1$ и $\hb_2$.
\item Найдите $s_{\hb_1}$, стандартную ошибку МНК-коэффициента $\hb_1$.
\end{enumerate}

\item Априори известно, что парная регрессия должна проходить через точку $(x_{0},y_{0})$.
\begin{enumerate}
\item  Выведите формулы МНК оценок;
\item В предположениях теоремы Гаусса-Маркова найдите дисперсии и средние оценок 
\end{enumerate}

\solution{Вроде бы равносильно переносу начала координат и применению результата для регрессии без свободного члена. Должна остаться несмещенность. }




\item \problem{ Слитки-вариант. Перед нами два золотых слитка и весы, производящие взвешивания с ошибками. Взвесив первый слиток, мы получили результат $300$ грамм, взвесив второй слиток --- $200$ грамм, взвесив оба слитка --- $400$ грамм. Предположим, что ошибки взвешивания --- независимые одинаково распределенные случайные величины с нулевым средним. 
\begin{enumerate}
\item Найдите несмещеную оценку веса первого шара, обладающую наименьшей дисперсией.
\item Как можно проинтерпретировать нулевое математическое ожидание ошибки взвешивания? 
\end{enumerate} }
\solution{ Как отсутствие систематической ошибки.} 




\end{enumerate}

\section{МНК без матриц и вероятностей}

\begin{enumerate}
\item \problem{Даны $n$ пар чисел: $(x_1, y_1)$, \ldots, $(x_n,y_n)$. Мы прогнозируем $y_i$ по формуле $\hy_i=\hb x_i$. Найдите $\hb$ методом наименьших квадратов. }
\solution{$\hb=\sum x_i y_i/\sum x_i^2$}

\item \problem{Даны $n$ чисел: $y_1$, \ldots, $y_n$. Мы прогнозируем $y_i$ по формуле $\hy_i=\hb$. Найдите $\hb$ методом наименьших квадратов. }
\solution{$\hb=\bar{y}$}

\item \problem{Даны $n$ пар чисел: $(x_1, y_1)$, \ldots, $(x_n,y_n)$. Мы прогнозируем $y_i$ по формуле $\hy_i=\hb_1+\hb_2 x_i$. Найдите $\hb_1$ и $\hb_2$ методом наименьших квадратов. }
\solution{$\hb_2=\sum (x_i-\bar{x})(y_i-\bar{y})/\sum(x_i-\bar{x})^2$, $\hb_1=\bar{y}-\hb_2\bar{x}$}

\item \problem{Даны $n$ пар чисел: $(x_1, y_1)$, \ldots, $(x_n,y_n)$. Мы прогнозируем $y_i$ по формуле $\hy_i=1+\hb x_i$. Найдите $\hb$ методом наименьших квадратов. }
\solution{$\hb=\sum x_i (y_i-1)/\sum x_i^2$}

\item \problem{ Перед нами два золотых слитка и весы, производящие взвешивания с ошибками. Взвесив первый слиток, мы получили результат $300$ грамм, взвесив второй слиток --- $200$ грамм, взвесив оба слитка --- $400$ грамм. Оцените вес каждого слитка методом наименьших квадратов.}
\solution{ $(300-\hb_1)^2+(200-\hb_2)^2+(400-\hb_1-\hb_2)^2\to\min$ }


\item Регрессия на дамми-переменную...
\end{enumerate}

\section{Голая линейная алгебра}

Здесь будет собран минимум задач по линейной алгебре.

\begin{enumerate}
\item \problem{Приведите пример таких $A$ и $B$, что $\det(AB)\neq \det(BA)$.}
\solution{Например, $A=(1,2,3)$, $B=(1,0,1)'$}


 
\end{enumerate}




\end{document}