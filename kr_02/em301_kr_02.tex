\documentclass[pdftex,12pt,a4paper]{article}

\input{/home/boris/science/tex_general/title_bor_utf8}


% чисто эконометрические сокращения:
\def \hb{\hat{\beta}}
\def \hy{\hat{y}}
\def \hY{\hat{Y}}
\def \he{\hat{\varepsilon}}
\def \v1{\vec{1}}
\def \e{\varepsilon}
\def \hVar{\widehat{\Var}}
\def \hCorr{\widehat{\Corr}}



\begin{document}
\parindent=0 pt % отступ равен 0


\begin{comment}
Плывут облака \\
Отдыхать после знойного дня,\\

Стремительных птиц \\
Улетела последняя стая. \\
 
Гляжу я на горы, \\
И горы глядят на меня, \\

И долго глядим мы,\\
Друг другу не надоедая.\\

\quote{Ли Бо, Одиноко сижу в горах Цзинтиншань}

\vspace{30pt}
\end{comment}

\begin{enumerate}
\item Случайные величины $Z_i$ независимы и нормально распределены $N(0,1)$. Для их суммы $S=\sum_{i=1}^n Z_i$ найдите $\E(S)$ и $\Var(S)$.
\item Социологическим опросам доверяют 70\% жителей. Те, кто доверяют
опросам, на все вопросы отвечают искренне; те, кто не доверяют, отвечают равновероятно наугад. Социолог Петя в анкету очередного опроса включил вопрос <<Доверяете ли Вы социологическим опросам?>>
\begin{enumerate}
\item Какова вероятность, что случайно выбранный респондент ответит «Да»?
\item Какова вероятность того, что он действительно доверяет, если известно, что он ответил
«Да»?  
\end{enumerate}
\item Регрессионная модель  задана в матричном виде при помощи уравнения $y=X\beta+\varepsilon$, где $\beta=(\beta_1,\beta_2,\beta_3)'$.
Известно, что $\E(\varepsilon)=0$  и  $\Var(\varepsilon)=\sigma^2\cdot I$.
Известно также, что 

$y=\left(
\begin{array}{c} 
1\\ 
2\\ 
3\\ 
4\\ 
5
\end{array}\right)$, 
$X=\left(\begin{array}{ccc}
1 & 0 & 0 \\
1 & 0 & 0 \\
1 & 0 & 1 \\
1 & 1 & 0 \\
1 & 1 & 0 
\end{array}\right)$.


Для удобства расчетов приведены матрицы 


$X'X=\left(
\begin{array}{ccc} 
5 & 2 & 1\\ 
2 & 2 & 0\\ 
1 & 0 & 1 
\end{array}\right)$ и $(X'X)^{-1}=\frac{1}{2}\left(
\begin{array}{ccc} 
1 & -1 & -1 \\
-1 & 2 & 1 \\
-1 & 1 & 3
\end{array}\right)$.

\begin{enumerate}
\item Укажите число наблюдений.

\item Укажите число регрессоров с учетом свободного члена.

\item Рассчитайте при помощи метода наименьших квадратов $\hb$, оценку для вектора неизвестных коэффициентов.

\item Рассчитайте $TSS=\sum (y_i-\bar{y})^2$, $RSS=\sum (y_i-\hat{y}_i)^2$ и $ESS=\sum (\hat{y}_i-\bar{y})^2$.

\item Чему равен $\he_4$, МНК-остаток регрессии, соответствующий 4-ому наблюдению?

\item Чему равен $R^2$  в модели? 

\item Рассчитайте несмещенную оценку для неизвестного параметра $\sigma^2$ регрессионной модели.

\item Рассчитайте $\widehat{\Var}(\hb)$, оценку для ковариационной матрицы вектора МНК-коэффициентов $\hb$.  

\item Найдите $\widehat{\Var}(\hb_1)$, несмещенную оценку дисперсии МНК-коэффициента $\hb_1$.

\item Найдите $\widehat{\Cov}(\hb_1,\hb_2)$, несмещенную оценку ковариации МНК-коэффициентов $\hb_1$ и $\hb_2$.

\item Найдите $\widehat{\Var}(\hb_1+\hb_2)$

\item Найдите $\hCorr(\hb_1,\hb_2)$, оценку коэффициента корреляции МНК-коэффициентов $\hb_1$ и $\hb_2$.

\item Найдите $se(\hb_1)$, стандартную ошибку МНК-коэффициента $\hb_1$.

\end{enumerate}
\item В классической линейной модели предполагается, что $\E(\e)=0$, $\Var(\e)=\sigma^2 I$. Найдите $\Cov(y,\he)$, $\Cov(\hy,\he)$.

\end{enumerate}

\end{document}