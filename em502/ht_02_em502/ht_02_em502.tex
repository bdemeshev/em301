\documentclass[pdftex,12pt,a4paper]{article}

\input{/home/boris/science/tex_general/title_bor_utf8}


\title{Домашка-2}
\date{}



\begin{document}

\pagestyle{empty}
\maketitle



Домашнее задание можно делать в одиночку или группой из двух человек. А можно всё-таки группой из трёх человек? Нет :) Крайний срок сдачи письменного отчёта --- 11 марта 2013 года.

\vspace{15pt}

\textbf{Задача <<Титаник>> }

\vspace{15pt}

Нужно зарегистрироваться на сайте \url{www.kaggle.com} и принять участие в конкурсе <<Titanic: Machine Learning from Disaster>>. 

Письменный отчёт по этой задаче должен содержать как-минимум:
\begin{enumerate}
\item Логин группы
\item Графический анализ имеющихся данных
\item Результаты оценивания logit и probit моделей с интерпретацией
\item Графический анализ logit и probit моделей
\item <<Если бы я был пассажиром Титаника, то я спасся бы с вероятностью\ldots>>. 

С помощью logit и probit моделей необходимо построить 95\%-ый доверительный интервал для вероятности спасения каждого из участников группы, сдающей домашку. Пол и возраст взять фактические, а остальные объясняющие переменные --- по своему желанию.
\end{enumerate}

\vspace{15pt}

\textbf{Задача <<Вольный странник>>}

\vspace{15pt}

На наборе данных по своему выбору нужно продемонстрировать владением любым методом, более сложным, чем обычный МНК. Например, это может быть оценка моделей GARCH, Tobit, Heckit, панельных данных, одновременных уравнений, использование инструментальных переменных, GMM и прочие страшные слова. Возможно использование методов, не изучавшихся в курсе эконометрики (RandomForest, нейронные сети, поддерживающие вектора, непараметрическое оценивание и так далее).

Письменный отчёт по этой задаче должен содержать как-минимум:
\begin{enumerate}
\item Описание данных
\item Краткое описание применяемого метода
\item Графический анализ имеющихся данных
\item Результаты оценивания моделей с интерпретацией
\item Графический анализ оценённых моделей
\end{enumerate}


\end{document}