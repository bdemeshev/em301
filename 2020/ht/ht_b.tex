\documentclass[10pt, a4paper]{extarticle}
\setlength{\parskip}{0.5em}
%%% Работа с русским языком
\usepackage{cmap}					% поиск в PDF
\usepackage{mathtext} 				% русские буквы в формулах
\usepackage[T2A]{fontenc}			% кодировка
\usepackage[utf8]{inputenc}			% кодировка исходного текста
\usepackage[english,russian]{babel}	% локализация и переносы
\usepackage{mathtools}   % loads »amsmath«
\usepackage{graphicx}
\usepackage{caption}
\usepackage{physics}
\usepackage{subcaption}
\usepackage{tikz}
\usepackage{multicol}
\usepackage{enumitem}

\usepackage{xcolor}
\usepackage{hyperref}
\definecolor{linkcolor}{HTML}{8b00ff} % цвет ссылок
%\definecolor{linkcolor}{HTML}{120a8f} % цвет ссылок (лучше смотрится с ss)
\definecolor{urlcolor}{HTML}{6b039b} % цвет гиперссылок
\definecolor{citecol}{HTML}{8a2be2} % ссылки на литру

\hypersetup{pdfstartview=FitH,  linkcolor=linkcolor,urlcolor=urlcolor, citecolor=citecol, colorlinks=true}

%%% Дополнительная работа с математикой
\usepackage{amsmath,amsfonts,amssymb,amsthm,mathtools} % AMS
\usepackage{icomma} % "Умная" запятая: $0,2$ --- число, $0, 2$ --- перечисление

%% Шрифты
\usepackage{euscript}	 % Шрифт Евклид
\usepackage{mathrsfs} % Красивый матшрифт

\title{Домашнее задание по эконометрике для студентов исследовательского потока\vspace{-0.5em}}
\author{Deadline: ...}
\date{\today}

\usepackage{geometry}
\geometry{
	a4paper,
	left=20mm,
	top=20mm,
	right=20mm
}
\setlength{\parindent}{0cm}

\DeclareMathOperator{\Lin}{\mathrm{Lin}}
\DeclareMathOperator{\Linp}{\Lin^{\perp}}
\DeclareMathOperator*\plim{plim}
%\DeclareMathOperator{\grad}{grad}
\DeclareMathOperator{\card}{card}
\DeclareMathOperator{\sgn}{sign}
\DeclareMathOperator{\sign}{sign}

\DeclareMathOperator*{\argmin}{arg\,min}
\DeclareMathOperator*{\argmax}{arg\,max}
\DeclareMathOperator*{\amn}{arg\,min}
\DeclareMathOperator*{\amx}{arg\,max}
\DeclareMathOperator{\cov}{Cov}
\DeclareMathOperator{\Var}{Var}
\DeclareMathOperator{\Cov}{Cov}
\DeclareMathOperator{\Corr}{Corr}
\DeclareMathOperator{\pCorr}{pCorr}
\DeclareMathOperator{\E}{\mathbb{E}}
\let\P\relax
\DeclareMathOperator{\P}{\mathbb{P}}



\newcommand{\cN}{\mathcal{N}}
\newcommand{\cU}{\mathcal{U}}
\newcommand{\cBinom}{\mathcal{Binom}}
\newcommand{\cPois}{\mathcal{Pois}}
\newcommand{\cBeta}{\mathcal{Beta}}
\newcommand{\cGamma}{\mathcal{Gamma}}

\def \R{\mathbb{R}}
\def \N{\mathbb{N}}
\def \Z{\mathbb{Z}}





\newcommand{\dx}[1]{\,\mathrm{d}#1} % для интеграла: маленький отступ и прямая d
\newcommand{\ind}[1]{\mathbbm{1}_{\{#1\}}} % Индикатор события
%\renewcommand{\to}{\rightarrow}
\newcommand{\eqdef}{\mathrel{\stackrel{\rm def}=}}
\newcommand{\iid}{\mathrel{\stackrel{\rm i.\,i.\,d.}\sim}}
\newcommand{\const}{\mathrm{const}}

\usepackage{framed}
\definecolor{shadecolor}{gray}{0.9}

% вместо горизонтальной делаем косую черточку в нестрогих неравенствах
\renewcommand{\le}{\leqslant}
\renewcommand{\ge}{\geqslant}
\renewcommand{\leq}{\leqslant}
\renewcommand{\geq}{\geqslant}


%\renewcommand{\rmdefault}{cmss}
%\renewcommand{\ttdefault}{cmss}
%\usepackage{sfmath}

\newcommand{\code}[1]{{\tt #1}}

\usepackage{enumitem}

%%% Заголовок
\usepackage{fancyhdr}
%\setlength{\headheight}{14.2pt}
%\pagestyle{fancyplain}
\pagestyle{fancy}
\fancyhf{}
\fancyhead[L]{}
\fancyhead[R]{\thepage}

\usepackage[sf,sl,outermarks]{titlesec}
\titleformat{\section}
{\Large\bfseries\sffamily}
{\thesection}{0.5em}{}
\titleformat{\subsection}
{\large\sffamily}
{\thesubsection}{0.5em}{}

\begin{document}

\maketitle

\begin{shaded}
Задание следует выполнять в \code{R}. Итоговый файл должен представлять из себя скрипт \code{.R} с кодом и вашими комментариями к нему. Работу следует отправить ... с пометкой ...

Задание состоит из двух частей, каждая из которых оценивается в 5 баллов. Разбивка баллов внутри частей указана рядом с номером заданий в скобках \textbf{жирным} шрифтом. За некоторые задания можно получить бонусы, которыми можно восполнить потерянные баллы в других заданиях. При этом один бонус = 0.5 балла. Максимальная оценка за работу -- 10, то есть нельзя получить 11, если всё сделано верно и получено два бонуса.

Выполненные задания должны идти по порядку, и их следует разделять так, чтобы проверяющему было понятно, где заканчивается одно задание и начинается другое. Вы можете использовать любые пакеты и библиотеки, и их подключение должно быть в начале работы. Вы также можете использовать материалы (в том числе и код) из любых открытых источников, однако в месте использования стоит привести ссылку на источник, чтобы избежать подозрения в плагиате. За обнаруженный и доказанный плагиат за всю работу ставится 0. Такая же оценка ставится и при обнаружении списывания, причём всем участникам, даже если можно однозначно определить кто у кого списал. 

Перед выполнением заданий не забудьте зафиксировать \code{seed} для воспроизводимости результатов. Все графики должны быть визуально понятными: не забудьте подписать оси и заголовки. Также к каждому графику должен быть приведён комментарий: графики без пояснений того, какой вывод можно сделать на их основе, не оцениваются. По возможности, старайтесь писать как можно больше выводов и комментариев к тому, что вы делаете и почему.
\end{shaded}

\section{Вперёд и с песней!}
В этой части мы будем применять полученные в течение курса эконометрические навыки на практике. Формат этой части достаточно свободный, оцениваются любые разумные действия и выводы. 

Загрузите набор данных \href{https://www.kaggle.com/sudalairajkumar/undata-country-profiles}{Country Statistics -- UNData}, содержащий различные географические, экономические и социальные показатели по разным странам мира. Обратите внимание, что по ссылке для скачивания доступно два файла, и нам требуется файл {\tt country\_profile\_variables.csv} Для загрузки понадобится регистрация на kaggle. 

\begin{enumerate}
	\item \textbf{(0.5)} Сформулируйте исследовательский вопрос. В соответствии с ним выберите непрерывную зависимую переменную. Заметим, что в серьёзных научных работах выбор следует объяснять ссылкой на литературу.
	
	\textit{Например: <<Я хочу изучить, как связаны темпы экономического роста и площадь территории страны. Для этого в качестве зависимой переменной я беру темп прироста ВВП>>.}
	\item \textbf{(0.5)} Выберите и/или создайте объясняющие переменные. Итоговая матрица регрессоров должна включать:
	\begin{enumerate}
		\item Не менее одной непрерывной переменной.
		\item Не менее одной бинарной переменной.
		\item Не менее одной нелинейной переменной (квадрат, логарифм и т.д.)
		
		\vspace{1em}
		\hspace{-2em}И по желанию для самых смелых \textbf{(+1 бонус, если дальше правильно интерпретируется)}:
		
		\vspace{0.5em}
		\item Не менее одной переменной взаимодействия.
	\end{enumerate}
	Поясните логику выбора переменных.
	
	В дальнейших пунктах потребуется дать смысловую интерпретацию оценок коэффициентов при выбранных регрессорах. Заметим, что если в исследовательском вопросе фигурирует одна независимая переменная (как в примере выше), то прочие переменные можно интерпретировать как \textit{контрольные}, то есть позволяющие учесть влияние сторонних факторов, что может быть важно по каким-либо причинам. 
	
	\textit{Например: <<В качестве непрерывных регрессоров я беру площадь территории страны и её квадрат, потому что (здесь идёт объяснение вашего выбора, отсутствующее из-за странности используемого примера)>>.}
	
	\item \textbf{(1 + 1 бонус за особо красивые графики)} Проведите визуальный анализ данных:
	\begin{enumerate}
		\item На наличие выбросов.
		\item На наличие пропущенных значений.
	\end{enumerate}
	
	При необходимости обработайте (например, удалите) выбросы. При наличии пропущенных значений удалите их или замените на какое-то значение (например, среднее, медиану и т.д. по регрессору). В любом случае, поясните ваши действия.
	
	Результатом данного пункта являются воспроизводимые графики и пояснения к ним. Графики должны быть визуально понятными: не забудьте подписать оси и заголовки. 
	
	\item \textbf{(1)} Задайте спецификацию модели. Проведите тестирование на наличие:
	\begin{enumerate}
		\item Мультиколлинеарности.
		\item Гетероскедастичности.
		\item Эндогенности.
	\end{enumerate}
	Для тестирования наличия каждой проблемы используйте не менее двух статистических тестов. Для проведения тестов используйте готовую реализацию: нужные пакеты и функции в \code{R} достаточно легко ищутся в поисковике (проверено на собственном опыте!) Для того чтобы показать наличие или отсутствие мультиколлинеарности, можно использовать теоретические знания линейной алгебры (но это не обязательно!)
	
	Поясните, как найденные проблемы исказят оценки МНК. В зависимости от найденных проблем, выберите метод оценки модели и поясните ваш выбор.
	
	\textit{Например: <<Я провёл (названия тестов) и выявил наличие в данных проблемы эндогенности. Таким образом, если я буду оценивать модель при помощи МНК, оценки коэффициентов будут несостоятельными. В данном случае для устранения проблемы разумно использовать 2МНК>>.}
	
	\item \textbf{(1)} Оцените модель:
	\begin{enumerate}
		\item При помощи МНК.
		\item Выбранным вами методом.
	\end{enumerate}
	Прокомментируйте результаты: чем отличаются оценки вашего метода от оценок МНК? Насколько сильно проявляется влияние найденных проблем? Выберите уровень значимости, который вам больше нравится, и прокомментируйте значимость коэффициентов на этом уровне. Прокомментируйте адекватность регрессии в целом. Проинтерпретируйте полученные оценки коэффициентов банальным образом \textit{(при увеличении $X_1$ на единицу, $Y$ увеличивается на $0.5$)} и по смыслу \textit{($X_1$ положительно влияет на $Y$, что можно объяснить слеующим образом: (объяснение)}. Если это возможно, дайте ответ на исследовательский вопрос.
	
	\item \textbf{(1)} Теперь попробуем отвлечься от эконометрических задач и попробовать себя в роли machine learner'а. Допустим, что мы заинтересованы не в получении качественного ответа на некоторый исследовательский вопрос, а в достижении наибольшего качества предсказания зависимой переменной. В такой постановке, вообще говоря, нас не интересует, какие проблемы представлены в данных: нам важно построить некоторую базовую модель, а затем предложить другую спецификацию модели, лучшую по качеству предсказания. 
	
	Задача ставится следующим образом. Предположим, что у нас есть зависимая переменная из пункта 1 и регрессоры из пункта 2, и мы провели визуальный анализ и модификацию данных из пункта 3. Базовой будем считать модель:
	\[
	Y = X\beta + u,
	\]
	оцениваемую при помощи МНК.
	
	Поделите данные на обучающую и тестовую выборку в соотношении 8:2. Оцените базовую модель на обучающей выборке и получите прогноз на тестовой выборке. Рассчитайте среднеквадратичную ошибку прогноза (MSE).
	
	Выберите и оцените три других спецификации модели. Разрешаются любые модификации: добавление или отброс переменных, взятие функций от регрессоров и проч., использование различных методов оценки. Рассчитайте среднеквадратичную ошибку прогнозов этих моделей. Задание считается выполненным, когда найдена такая спецификация модели, среднеквадратичная ошибка прогноза которой меньше, чем у МНК. В зависимости от того, насколько удалось снизить ошибку прогноза, могут быть выставлены бонусные баллы \textbf{(до +2 бонусов)}. 
	
\end{enumerate}

\section{Табалуга и река времени}
В этой части мы будем работать с временными рядами. Так как анализ реальных временных рядов требует значительной подготовки, мы будем использовать искусственно созданные ряды, которые сами и сгенерируем.

Сейчас в \code{R} существует два распространённых стиля работы с временными рядами:
\begin{itemize}
	\item Хорошо устоявшийся пакет \href{https://otexts.com/fpp2/}{\code{forecast}} для работы с небольшим количеством рядов.
	\item Новый пакет \href{https://otexts.com/fpp3/}{\code{fable}} с кучей модных \href{https://education.rstudio.com/blog/2020/02/conf20-ts/}{плюшек} для работы с сотнями и тысячами рядов. 
\end{itemize}

Вы можете работать в рамках любого подхода.

\begin{enumerate}
	\item \textbf{(1)} Сгенерируйте временные ряды, задающиеся следующими уравнениями:
	\begin{align*}
		y_t &= 0.8y_{t-1} + \varepsilon_{t} \\
		y_t &= 0.1y_{t-1} + 0.2y_{t-2} + 0.3y_{t-3} + \varepsilon_t \\
		y_t &= \varepsilon_t + 1.2\varepsilon_{t-1} + 2\varepsilon_{t-2},
	\end{align*}
	 каждый из которых состоит из 120 наблюдений. 
	\begin{enumerate}
		\item Выпишите спецификацию моделей, задаваемых этими уравнениями. Например, AR(5).
		\item Постройте графики полученных временных рядов. 
		\item Используя графики и/или полученные знания, прокомментируйте, имеют ли данные уравнения стационарные решения. 
	\end{enumerate}
	\item \textbf{(1)} На основе предыдущего пункта, сгенерируйте временные ряды, уравнения которых специфицируется как ARIMA(0, 1, 2), ARIMA(0, 0, 0), ARIMA(3, 0, 0). Постройте графики и прокомментируйте, имеют ли соответствующие уравнения стационарные решения. 
	\item \textbf{(0.5)} Вспомните уравнение случайного блуждания. Сгенерируйте соответствующий временной ряд и постройте его график. Имеет ли это уравнение стационарные решения? 
	\item \textbf{(0.5)} Из созданных выше рядов выберите ряд, задаваемый моделью AR(1), уравнение которого имеет стационарные решения. Постройте автокорреляционную и частную автокорреляционную функции для этого ряда. Сравните их с ACF и PACF случайного блуждания. Прокомментируйте результаты. 
	
	\item \textbf{(2)}\begin{enumerate}
		\item Сгенерируйте ряд из 120 наблюдений, задаваемый моделью ARIMA(2, 0, 3).
		\item Разделите ряд на обучающую выборку из 100 наблюдений и тестовую выборку из 20 наблюдений.
		\item Оцените модель ARIMA(2, 0, 3) на обучающей выборке.
		\item Постройте прогноз на 20 периодов вперёд для этой модели, используя 95\% доверительные интервалы.
		\item Постройте полученные прогнозные значения и тестовую выборку на одном графике. Визуально оцените качество прогноза.
	\end{enumerate}
\end{enumerate}


\end{document}