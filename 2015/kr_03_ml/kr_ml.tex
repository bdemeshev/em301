\documentclass[12pt]{article}

\usepackage[utf8]{inputenc}
\usepackage[russian]{babel}
\usepackage[left = 1cm, right = 1cm, top = 1cm, bottom = 1cm]{geometry}
\usepackage{amsmath}
\usepackage{amsfonts}


\def \useR{$[$R$]$ }

%% эконометрические сокращения
\def \hb{\hat{\beta}}
\def \b{\beta}
\def \hs{\hat{s}}
\def \hy{\hat{y}}
\def \hY{\hat{Y}}
\def \he{\hat{\varepsilon}}
\def \v1{\vec{1}}
\def \e{\varepsilon}
\def \z{z}
\def \hVar{\widehat{\Var}}
\def \hCorr{\widehat{\Corr}}
\def \hCov{\widehat{\Cov}}
\def \cN{\mathcal{N}}
\renewcommand{\P}{\mathbb{P}}

%% лаг
\renewcommand{\L}{\mathrm{L}}


\begin{document}

\begin{enumerate}
\thispagestyle{empty}

\item Как известно, Фрекен Бок любит пить коньяк по утрам. За прошедшие пять дней она записала, сколько рюмочек коньяка выпила утром, $x_i$, и видела ли она в этот день привидение, $y_i$,

\begin{tabular}{c|ccccc}
$y_i$ & 1 & 0 & 1 & 0 & 0 \\
\hline
$x_i$ & 2 & 1 & 3 & 1 & 0
\end{tabular}

Зависимость между $y_i$ и $x_i$ описывается пробит-моделью, $\P(y_i=1)=F(\beta_1 + \beta_2 x_i)$.

\begin{enumerate}
\item Выпишите логарифмическую функцию правдоподобия
\item Выпишите условия первого порядка для оценки $\beta_1$ и $\beta_2$
\end{enumerate}


\item Приведите пример небольшого набора данных для которого оценки логит модели $\P(y_i=1)=F(\beta_1 + \beta_2 x_i)$ не существуют. В наборе данных должны присутствовать хотя бы одно наблюдение $y_i=0$ и хотя бы одно наблюдение $y_i=1$.

\item Почему в пробит-модели предполагается, что $\e_i \sim \cN(0;1)$, а не $\e_i \sim \cN(0;\sigma^2)$ как в линейной регрессии?

\item Исследователь Вениамин пытается понять, как логарифм количества решённых им по эконометрике задач зависит от количества съеденных им пирожков. Для этого он собрал 100 наблюдений. Первые 50 наблюдений --- относятся к пирожкам с мясом, а последние 50 наблюдений --- к пирожкам с повидлом. Вениамин считает, что ожидаемое количество решённых задач не зависит от начинки пирожков, а только от их количества, т.е. $y_i = \beta x_i + u_i$. Однако он полагает, что для пирожков с мясом --- $u_i\sim \cN(0;\sigma^2_M)$, а для пирожков с повидлом --- $u_i\sim \cN(0;\sigma^2_J)$.

\begin{enumerate}
\item Выпишите логарифмическую функцию правдоподобия
\item Выпишите условия первого порядка для оценки $\beta$, $\sigma^2_M$, $\sigma^2_J$
\end{enumerate}

\item При оценке логит модели
\(
\P(y_i=1)=\Lambda(\b_1+\b_2 x_i)
\)
по 500 наблюдениям оказалось, что $\hb_1=0.7$ и $\hb_2=3$. Оценка ковариационной матрицы коэффициентов имеет вид
\[
\begin{pmatrix}
  0.04 & 0.01 \\
  0.01 & 0.09
\end{pmatrix}
\]

\begin{enumerate}
\item Проверьте гипотезу о незначимости коэффициента $\hb_2$
\item Найдите предельный эффект роста $x_i$ на вероятность $\P(y_i=1)$ при $x_i=-0.5$
\item Найдите максимальный предельный эффект роста $x_i$ на вероятность $\P(y_i=1)$
\item Постройте точечный прогноз вероятности $\P(y_i=1)$ если $x_i = -0.5$
\item Найдите стандартную ошибку построенного прогноза
\end{enumerate}

\item После долгих изысканий Вениамин пришёл к выводу, что $\beta=0$, т.е. что логарифм количества решенных им по эконометрике за вечер задач имеет нормальное распределение $y_i$ с математическим ожиданием ноль. Однако он по прежнему уверен, что дисперсия $y_i$ зависит от того, какие пирожки он ел в этом вечер. Для пирожков с повидлом $y_i \sim \cN(0;\sigma^2_J)$, а для пирожков с мясом --- $y_i\sim \cN(0;\sigma^2_M)$. Всего 100 наблюдений. Первые 50 вечеров относятся к пирожкам с мясом, последние 50 вечеров --- к пирожкам с повидлом:
\[
\sum_{i=1}^{50} y_i = 10, \; \sum_{i=1}^{50} y_i^2 = 100, \;
\sum_{i=51}^{100} y_i = -10, \; \sum_{i=51}^{100} y_i^2 = 300
\]
\begin{enumerate}
\item Найдите оценки $\sigma^2_M$, $\sigma^2_J$, которые получит Вениамин.
\item Помогите Вениамину проверить гипотезу $\sigma^2_M = \sigma^2_J$ с помощью тестов отношения правдоподобия, множителей Лагранжа и Вальда.
\end{enumerate}

\end{enumerate}

\end{document}
