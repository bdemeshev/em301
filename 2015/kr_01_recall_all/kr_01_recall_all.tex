\documentclass[12pt,a4paper]{article}
\usepackage[utf8]{inputenc}
\usepackage[russian]{babel}
\usepackage[OT1]{fontenc}
\usepackage{amsmath}
\usepackage{amsfonts}
\usepackage{amssymb}

\usepackage[left=1cm,right=1cm,top=1cm,bottom=1cm]{geometry}



\DeclareMathOperator{\tr}{tr}
\DeclareMathOperator{\E}{\mathbb{E}}
\let\P\relax
\DeclareMathOperator{\P}{\mathbb{P}}
\DeclareMathOperator{\Var}{\mathbb{V}ar}
\DeclareMathOperator{\Cov}{\mathbb{C}ov}

\begin{document}

\pagestyle{empty}

Праздник 1. Вспомнить всё!

\begin{enumerate}
\item Найдите длины векторов $a=(1,1,1,1)$ и $b=(1,2,3,4)$ и косинус угла между ними. Найдите один любой вектор, перпенидкулярный вектору $a$.
\item Сформулируйте теорему о трёх перпендикулярах
\item Для матрицы 

\[
A=\begin{pmatrix}
10 & 15 \\
15 & 26 \\
\end{pmatrix} 
\]

\begin{enumerate}
\item Найдите собственные числа и собственные векторы матрицы
\item Найдите $\det (A)$, $\tr(A)$
\item Найдите обратную матрицу, $A^{-1}$, ее собственные векторы и собственные числа
\end{enumerate}

\item Известно, что $X$ --- матрица размера $n \times k$ и $n>k$, известно, что $X'X$ обратима. Рассмотрим матрицу $H=X(X'X)^{-1}X'$. Укажите размер матрицы $H$, найдите $H^{2015}$, $\tr(H)$, $\det(H)$, собственные числа матрицы $H$. Штрих означает транспонирование.

\item Для случайных величин $X$ и $Y$ заданы следующие значения: $\E(X)=1$, $\E(Y)=4$, $\E(XY)=8$, $\Var(X)=\Var(Y)=16$. Для случайных величин $U=X+Y$ и $V=X-Y$ вычислите: 
\begin{enumerate}
\item $\E(U)$, $\Var(U)$, $\E(V)$, $\Var(V)$, $\Cov(U,V)$ 
\item Можно ли утверждать, что случайные величины $U$ и $V$ независимы? 
\end{enumerate}

\item Вася ведёт блог. Обозначим $X_i$ --- количество слов в $i$--ой записи. После первого года он по 200 своим записям обнаружил, что $\bar{X}_{200}=95$ и выборочное стандартное отклонение равно $300$ слов. На уровне значимости $\alpha=0.15$ проверьте гипотезу о том, что $\mu=100$ против альтернативной гипотезы $\mu\neq 100$. Постройте $85$-ти процентный доверительный интервал для $\mu$.

\item Саша и Маша решают одну и ту же задачу. Саша правильно решит задачу с вероятностью $0.8$, Маша, независимо от Саши (!), с вероятностью $0.7$. Какова вероятность того, что Маша верно решила задачу, если задачу верно решил только кто-то один из них?

\end{enumerate}

\newpage


\pagestyle{empty}

Праздник 1. Вспомнить всё!

\begin{enumerate}
\item Найдите длины векторов $a=(1,1,1,1)$ и $b=(1,2,3,4)$ и косинус угла между ними. Найдите один любой вектор, перпенидкулярный вектору $a$.
\item Сформулируйте теорему о трёх перпендикулярах
\item Для матрицы 

\[
A=\begin{pmatrix}
10 & 15 \\
15 & 26 \\
\end{pmatrix} 
\]

\begin{enumerate}
\item Найдите собственные числа и собственные векторы матрицы
\item Найдите $\det (A)$, $\tr(A)$
\item Найдите обратную матрицу, $A^{-1}$, ее собственные векторы и собственные числа
\end{enumerate}

\item Известно, что $X$ --- матрица размера $n \times k$ и $n>k$, известно, что $X'X$ обратима. Рассмотрим матрицу $H=X(X'X)^{-1}X'$. Укажите размер матрицы $H$, найдите $H^{2015}$, $\tr(H)$, $\det(H)$, собственные числа матрицы $H$. Штрих означает транспонирование.


\item Для случайных величин $X$ и $Y$ заданы следующие значения: $\E(X)=1$, $\E(Y)=4$, $\E(XY)=8$, $\Var(X)=\Var(Y)=16$. Для случайных величин $U=X+Y$ и $V=X-Y$ вычислите: 
\begin{enumerate}
\item $\E(U)$, $\Var(U)$, $\E(V)$, $\Var(V)$, $\Cov(U,V)$ 
\item Можно ли утверждать, что случайные величины $U$ и $V$ независимы? 
\end{enumerate}

\item Вася ведёт блог. Обозначим $X_i$ --- количество слов в $i$--ой записи. После первого года он по 200 своим записям обнаружил, что $\bar{X}_{200}=95$ и выборочное стандартное отклонение равно $300$ слов. На уровне значимости $\alpha=0.15$ проверьте гипотезу о том, что $\mu=100$ против альтернативной гипотезы $\mu\neq 100$. Постройте $85$-ти процентный доверительный интервал для $\mu$.

\item Саша и Маша решают одну и ту же задачу. Саша правильно решит задачу с вероятностью $0.8$, Маша, независимо от Саши (!), с вероятностью $0.7$. Какова вероятность того, что Маша верно решила задачу, если задачу верно решил только кто-то один из них? 
\end{enumerate}


\end{document}