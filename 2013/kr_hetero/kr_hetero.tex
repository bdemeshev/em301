\documentclass[12pt,a4paper]{article}
\usepackage[utf8]{inputenc}
\usepackage[russian]{babel}

\usepackage{amsmath}
\usepackage{amsfonts}
\usepackage{amssymb}
\usepackage{graphicx}
\usepackage[left=2cm,right=2cm,top=2cm,bottom=2cm]{geometry}
\begin{document}
Маленькая контрольная по гетероскедастичности.

\begin{enumerate}
\item Желая протестировать наличие гетероскедастичности в модели $y_i=\beta_1+\beta_2 x_i +\beta_3 z_i +\beta_4 w_i +\varepsilon_i$, эконометресса Глафира решила провести тест Уайта и получила  во вспомогательной регрессии $R^2=0.50$. Глафира строит модель удоя по 200 коровам. Помогите ей провести тест на уровне значимости 5\%.
\item На всякий случай эконометресса Глафира решила подстраховаться и провести тест Голдфельда-Квандта. Но она совсем забыла, как его делать. Напомните Глафире, как провести тест Голдфельда-Квандта, если она подозревает, что дисперсия $Var(\varepsilon_i)$ возрастает с ростом $z_i$. Чётко напишите гипотезы $H_0$, $H_a$, методику проведения теста, правило согласно которому отвергается или не отвергается $H_0$.
\item Имеются три наблюдения, $x=(1,2,2)'$, $y=(2,1,0)'$. Предполагая, что в модели $y_i=\beta x_i + \varepsilon_i$ имеется гетероскедастичность вида $Var(\varepsilon_i)=\sigma^2 x_i^4$ найдите:
\begin{enumerate}
\item Обычную МНК-оценку параметра $\beta$
\item Самую эффективную среди несмещенных оценку параметра $\beta$
\item Во сколько раз отличается истинная дисперсия этих двух оценок?
\item Во сколько раз отличаются оценки дисперсий этих оценок, если дисперсии оценивается без поправки на гетероскедастичность в обоих случаях?
\end{enumerate}
\end{enumerate}




\end{document}