\documentclass[pdftex,12pt,a4paper]{article}

\input{title_bor}


\def \useR{$[$R$]$ }

%% эконометрические сокращения
\def \hb{\hat{\beta}}
\def \b{\beta}
\def \hs{\hat{s}}
\def \hy{\hat{y}}
\def \hY{\hat{Y}}
\def \he{\hat{\varepsilon}}
\def \v1{\vec{1}}
\def \e{\varepsilon}
\def \z{z}
\def \hVar{\widehat{\Var}}
\def \hCorr{\widehat{\Corr}}
\def \hCov{\widehat{\Cov}}
\def \cN{\mathcal{N}}
\renewcommand{\P}{\mathbb{P}}

%% лаг
\renewcommand{\L}{\mathrm{L}}


\begin{document}



\begin{enumerate}
\item  Случайные величины $X_{1}$, ..., $X_{n}$ --- независимы и одинаково распределены с функцией плотности $f(t)=\frac{\theta \cdot\left(\ln  t\right)^{\theta -1}}{t} $  при  $t\in
\left[1;e\right]$. По выборке из 100 наблюдений оказалось, что $\sum{\ln(\ln(X_{i}))}=-30$ 
\begin{enumerate}
\item Найдите ML оценку параметра $\theta$
\item Постройте 95\% доверительный интервал для $\theta$
\item С помощью LR, LM и W теста проверьте гипотезу о том, что $\theta=1$.
\end{enumerate}

\item Величины $X_{1}$, ..., $X_{n}$ --- независимы и нормально распределены, $N(\mu,\sigma^2)$. По 100 наблюдениям $\sum X_i=100$ и  $\sum X_i^2=900$. 
\begin{enumerate}
\item Найдите ML оценки неизвестных параметров $\mu$ и $\sigma^2$.
\item Постройте 95\%-ые доверительные интервалы для $\mu$ и $\sigma^2$
\item С помощью LR, LM и W теста проверьте гипотезу о том, что $\sigma^2=1$.
\item С помощью LR, LM и W теста проверьте гипотезу о том, что $\sigma^2=1$ и одновременно $\mu=2$.
\end{enumerate}

\end{enumerate}


\vspace{20pt}

Всех участников правдоподобной контрольной с древнерусским эконометрическим праздником! 

\vspace{20pt}

Сегодня \textbf{Аксинья-полухлебница}.

\vspace{20pt}

<<На Аксинью гадали о ценах на хлеб в ближайшее время и на будущий урожай: брали печёный хлеб и взвешивали его сначала вечером, а потом утром. Коли вес оставался неизменным — цена на хлеб не изменится. Если за ночь вес уменьшался — значит, хлеб подешевеет, а если увеличивался, то подорожает>>

\begin{flushright}
Wikipedia
\end{flushright}


\end{document}