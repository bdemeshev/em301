\documentclass[pdftex,12pt,a4paper]{article}

\input{/home/boris/science/tex_general/title_bor_utf8}

\title{Эконометрика на Python'е}
\author{Борис Демешев\footnote{\href{mailto:boris.demeshev@gmail.com}{boris.demeshev@gmail.com}}}
\date{\today}



\begin{document}
\maketitle
\parindent=0 pt % отступ равен 0

\todolist

\section{О python'е}

Голый Питон не пригоден для анализа данных. 
Работать с данными позволяют множество библиотек.
Любая наша программа будет начинаться с комманд
\begin{verbatim}
>>> import numpy as np
>>> import pandas as pd
>>> import matplotlib.pyplot as plt
\end{verbatim}




Вопросы по статистическим методам лучше задавать на \url{http://stats.stackexchange.com/}


Вопросы по Питону --- на \url{http://stackoverflow.com/}


Вопросы по \LaTeX'у --- на \url{http://tex.stackexchange.com/}

\section{Три стиля работы и графики}

На практике используются три стиля работы

\begin{enumerate}
\item Интерактивная работа. Вы вводите одну команду~--- она немедленно исполняется. Результаты вычислений и построенные графики видны сразу.
\item Написание программы. Вы пишите длинную программу, запускаете её. Она думает некоторое время о чём-то своём и выводит результаты  работы~--- разные числа и графики.
\item Создание \LaTeX документа с кодом на python'е с помощью Pweave.
\end{enumerate}

В интерактивном режиме имеет смысл попросить python выводить графики сразу. Делается это командой
\begin{verbatim}
>>> plt.ion()
\end{verbatim}


При создании \LaTeX документов с помощью Pweave интерактивный режим включать не имеет смысла. Обычно вставка картинки выглядит так

\todo{Как сказать Pweave'у, что код исполнять не надо?}



\section{Генерация случайных чисел, базовые операции}


Питон можно использовать как калькулятор:
\begin{verbatim}
>>> print(2*(5+17))
44
\end{verbatim}


Есть несколько особенностей.
\begin{enumerate}
\item Возведение в степень обозначается **
\begin{verbatim}
>>> print(2**10)
1024
\end{verbatim}

\item По умолчанию результат деления целого числа на целое считается целым, например
\begin{verbatim}
>>> print(90/8)
11
\end{verbatim}


\end{enumerate}



\section{Загрузка данных}

Исходные данные нужно хранить в самом простом формате: в текстовом. О другом формате можно задуматься, только  если загрузка данных занимает слишком много времени.

Загрузить содержимое текстового файла в DataFrame с именем mydata можно командой:
\begin{verbatim}
>>> mydata=pd.read_csv('file_with_data.csv')
\end{verbatim}


Выводим информацию о наборе данных. Если наблюдений много --- будут напечатаны только названия переменных (столбцов DataFrame), если наблюдений мало --- то будут напечатаны сами наблюдения. 
\begin{verbatim}
>>> print mydata
    abc    defm   ghi
0   423.  4.00   234.
1   3243  4324   324.
2   234.  324.   234.
3   32.0  423.   423.
4   234.  432.   423.
5   234.  435.   452.
6   234.  342.   1324
7   322.  12.0   4.00
8   234.  234.   124.
9   NaN   NaN    NaN 
10  NaN   NaN    NaN 
\end{verbatim}


Узнаем тип каждой переменной:
\begin{verbatim}
>>> print mydata.dtypes
abc      float64
 defm    float64
 ghi     float64
\end{verbatim}




\section{Простые графики и описательные статистики}



\section{Оценка эконометрических моделей}




\end{document}