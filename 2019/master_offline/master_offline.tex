\documentclass[12pt]{article}

\usepackage{tikz} % картинки в tikz
\usepackage{microtype} % свешивание пунктуации

\usepackage{array} % для столбцов фиксированной ширины

\usepackage{indentfirst} % отступ в первом параграфе

\usepackage{sectsty} % для центрирования названий частей
\allsectionsfont{\centering}

\usepackage{amsmath, amssymb, amsthm} % куча стандартных математических плюшек

\usepackage{amsfonts}

\usepackage{dcolumn} % 

\usepackage{comment}

\usepackage[top=2cm, left=1.2cm, right=1.2cm, bottom=2cm]{geometry} % размер текста на странице

\usepackage{lastpage} % чтобы узнать номер последней страницы

\usepackage{enumitem} % дополнительные плюшки для списков
%  например \begin{enumerate}[resume] позволяет продолжить нумерацию в новом списке
\usepackage{caption}


\usepackage{hyperref} % гиперссылки

\usepackage{multicol} % текст в несколько столбцов


\usepackage{fancyhdr} % весёлые колонтитулы
\pagestyle{fancy}
\lhead{Эконометрика, НИУ-ВШЭ}
\chead{экзамен}
\rhead{2019-12-26}
\lfoot{Вариант $\xi$}
\cfoot{Ни пуха, ни пера!}
\rfoot{\thepage/3}
\renewcommand{\headrulewidth}{0.4pt}
\renewcommand{\footrulewidth}{0.4pt}



\usepackage{todonotes} % для вставки в документ заметок о том, что осталось сделать
% \todo{Здесь надо коэффициенты исправить}
% \missingfigure{Здесь будет Последний день Помпеи}
% \listoftodos - печатает все поставленные \todo'шки


% более красивые таблицы
\usepackage{booktabs}
% заповеди из докупентации:
% 1. Не используйте вертикальные линни
% 2. Не используйте двойные линии
% 3. Единицы измерения - в шапку таблицы
% 4. Не сокращайте .1 вместо 0.1
% 5. Повторяющееся значение повторяйте, а не говорите "то же"



\usepackage{fontspec}
\usepackage{polyglossia}

\setmainlanguage{russian}
\setotherlanguages{english}

% download "Linux Libertine" fonts:
% http://www.linuxlibertine.org/index.php?id=91&L=1
\setmainfont{Linux Libertine O} % or Helvetica, Arial, Cambria
% why do we need \newfontfamily:
% http://tex.stackexchange.com/questions/91507/
\newfontfamily{\cyrillicfonttt}{Linux Libertine O}

\AddEnumerateCounter{\asbuk}{\russian@alph}{щ} % для списков с русскими буквами
\setlist[enumerate, 2]{label=\asbuk*),ref=\asbuk*}

%% эконометрические сокращения
\let\P\relax
\DeclareMathOperator{\Cov}{\mathbb{C}ov}
\DeclareMathOperator{\Corr}{\mathbb{C}orr}
\DeclareMathOperator{\Var}{\mathbb{V}ar}
\DeclareMathOperator{\E}{\mathbb{E}}
\DeclareMathOperator{\P}{\mathbb{P}}
\DeclareMathOperator{\tr}{trace}
\def \hb{\hat{\beta}}
\def \hs{\hat{\sigma}}
\def \htheta{\hat{\theta}}
\def \s{\sigma}
\def \hy{\hat{y}}
\def \hY{\hat{Y}}
\def \v1{\vec{1}}
\def \e{\varepsilon}
\def \he{\hat{\e}}
\def \z{z}
\def \hVar{\widehat{\Var}}
\def \hCorr{\widehat{\Corr}}
\def \hCov{\widehat{\Cov}}
\def \cN{\mathcal{N}}





\def \putyourname{\fbox{
    \begin{minipage}{42em}
      Фамилия, имя, номер группы:\vspace*{3ex}\par
      \noindent\dotfill\vspace{2mm}
    \end{minipage}
  }
}

\def \checktable{
\begin{minipage}{42em}
\begin{tabular}{|m{1.8cm}|m{1.8cm}|m{1.8cm}|m{1.8cm}|m{1.8cm}|m{1.8cm}|}
\hline
Тест & 1 &  2 & 3 & 4 &  Итого \\ \hline
&  &  &  &  & \\
 &  &   & & & \\
 \hline
\end{tabular} $\leftarrow$ для проверяющего!
\end{minipage}
}

\def \testtable{
\begin{minipage}{42em}
\vspace{4pt}

Ответы на тест:

\vspace{2pt}
\begin{tabular}{|m{1cm}|m{1cm}|m{1cm}|m{1cm}|m{1cm}|m{1cm}|m{1cm}|m{1cm}|m{1cm}|m{1cm}|}
\hline
1 & 2 &  3 & 4 & 5 & 6 & 7 & 8 & 9 & 10 \\ 
\hline
 &  &   &  &  &  &  &  &  &  \\ 
 &  &   &  &  &  &  &  &  &  \\ 
\hline
\end{tabular}
\end{minipage}

}





% [1][3] 1 = one argument, 3 = value if missing
% эта магия создаёт окружение answerlist
% именно в окружении answerlist записаны варианты ответов в подключаемых exerciseXX
% просто \begin{answerlist} сделает ответы в три столбца
% если ответы длинные, то надо в них руками сделать
% \begin{answerlist}[1] чтобы они шли в один столбец
\newenvironment{answerlist}[1][3]{
\begin{multicols}{#1}
\begin{enumerate}[label=\fbox{\emph{\Alph*}},ref=\emph{\alph*}]
}
{
\end{enumerate}
\end{multicols}
}

\newenvironment{answerlist1}{
\begin{enumerate}[label=\fbox{\emph{\Alph*}},ref=\emph{\alph*}]
}
{
\end{enumerate}
}



\excludecomment{solution} % without solutions

\theoremstyle{definition}
\newtheorem{question}{Вопрос}




\begin{document}

% \checktable

\putyourname

% \testtable


\begin{enumerate}
\item Исследователь рассматривает уравнение зависимости расходов на питание (W) от доходов (Income), с учетом сезона. Переменная сезон (S) принимает следующие значения: 1 – зима, 2 – весна, 3 – лето и 4 – осень. Исследователь предполагает, что в каждый сезон может выполняться своя линейная зависимость. 

\begin{enumerate}
    \item (2 балла) Выпишите уравнение оцениваемой модели. Укажите смысл всех включенных в модель переменных.
    \item (2 балла) Как проверить гипотезу о единой линейной зависимости расходов на питание для всех сезонов? Выпишите аккуратно основную и альтернативную гипотезы, формулу расчета статистики и способ проверки.
\end{enumerate}

% q2 done
\item Начинающий исследователь Елисей исследует зависимость успехов в учёбе своих однокурсников, $G_i$, от времени, которое они тратят на учёбу, $T_i$. По выборке из 100 человек он смог оценить следующую регрессию:
\[
\hat G_i = 30 + 6T_i
\]

Елисей был бы рад полученному результату, но тут на лекции по эконометрике ему рассказали про эндогенность и пропущенные переменные, и он решил, что в его модели эти проблемы точно есть. Изучив литературу, он узнал, что на успехи в учёбе кроме времени влияют ещё и способности студента, $A_i$, при этом способности коррелированы со временем, которое студент тратит на учёбу.

\begin{enumerate}
\item Проверьте, является ли найденная Елисеем оценка коэффициента при времени состоятельной;
\item Если оценка не состоятельна, то предложите способ получения состоятельной оценки;
\item Найдите асимптотическую величину смещения оценки, если $\Cov(G_i, A_i) = 6$,
$\Cov(T_i, A_i) = 4$, $\Var(G_i) = 16$, $\Var(A_i) = 100$, $\Var(T_i) = 49$.
\end{enumerate}




\item По 24 наблюдениям была оценена модель:

\[
\widehat{Y}_i=15-4Z_i+3W_i
\]

Известно, что случайные ошибки нормально распределены, $RSS=180$, и

\[
(X'X)^{-1} =
\begin{pmatrix}{}
  0.216 & -0.112 & -0.075 \\ 
  -0.112 & 0.119 & 0.021 \\ 
  -0.075 & 0.021 & 0.047 \\ 
  \end{pmatrix}
\]


\begin{enumerate}
\item (1 балл) Проверьте гипотезу $H_0: \beta_Z = 0$ против $H_a: \beta_Z \neq 0$ на уровне значимости~5\%.
\item (3 балла) Проверьте гипотезу $H_0: \beta_Z + \beta_W = 0$  против $H_a: \beta_Z + \beta_W \neq 0$ на уровне значимости~5\%.
\item (2 балла) Выпишите использованные при проверке гипотез предпосылки о случайных ошибках модели.
\end{enumerate}


\newpage
\item Исследовательница Глафира изучает зависимость спроса на молоко от цены молока и дохода семьи. В её распоряжении есть следующие переменные:

\begin{itemize}
\item $price$ — цена молока в рублях за литр
\item $income$ — ежемесячный доход семьи в тысячах рублей
\item $milk$ — расходы семьи на молоко за последние семь дней в рублях
\end{itemize}

В данных указано, проживает ли семья в сельской или городской местности. Поэтому Глафира оценила три регрессии: (All) — по всем данным, (Urban) — по городским семьям, (Rural) — по сельским семьям.


\begin{tabular}{lD{.}{.}{3}D{.}{.}{3}D{.}{.}{3}}
\toprule
 & 
\multicolumn{1}{c}{(All)} & 
\multicolumn{1}{c}{(Urban)} & 
\multicolumn{1}{c}{(Rural)}\\
\midrule
(Intercept)    & -1.765       & -4.059       & -0.155      \\
               & (4.943)      & (6.601)      & (7.812)     \\
income         &  0.308^{***} &  0.341^{***} &  0.281^{***}\\
               & (0.052)      & (0.072)      & (0.079)     \\
price          & -0.383^{*}   & -0.352       & -0.391      \\
               & (0.161)      & (0.253)      & (0.221)     \\
\midrule
R-squared      &    0.304 &   0.356 &    0.273\\
adj. R-squared &    0.290 &   0.325 &    0.245\\
sigma          &    4.912 &   4.857 &    5.036\\
F              &   21.216 &  11.593 &    9.744\\
P-value        &    0.000 &   0.000 &    0.000\\
RSS            & 2340.080 & 990.839 & 1318.741\\
n observations &  100     &  45     &   55    \\
\bottomrule
\end{tabular}




\begin{enumerate}
\item (1 балл) Проверьте значимость в целом регрессии (All) на 5\%-ом уровне значимости.
\item (2 балла) На 5\%-ом уровне значимости проверьте гипотезу, что зависимость спроса на молоко является единой для городской и сельской местности.
\end{enumerate}


\newpage

\item Рассмотрим стационарный процесс, удовлетворяющий уравнению $Y_t = 1 + 0.6Y_{t-1} + u_t - 0.3 u_{t-1}$, 
где $u_t$ — белый шум с $u_t \sim \cN(0; 4)$.

\begin{enumerate}
	\item Найдите $\E(Y_t)$, $\Var(Y_t)$.
	\item Найдите первые два значения автокорреляционной и частной автокорреляционной функций.
	\item Постройте 95\% интервальный прогноз для $Y_{101}$, если известно, что $Y_{100} = 2$, $u_{100}=1$.
\end{enumerate}



\item По 200 наблюдениям исследователь Иннокентий оценил модель логистической регрессии для вероятности
сдать экзамен по метрике:
\[
\hat \P(Y_i = 1) = \Lambda(1.5 + 0.3X_i - 0.4 D_i),
\]
где $Y_i$ — бинарная переменная равная 1, если студент сдал экзамен;
$X_i$ — количество часов подготовки студента; $D_i$ — бинарная переменная равная 1,
если студент пробовал пиццу «четыре сыра» в новой столовой.

Оценка ковариационной матрицы оценок коэффициентов имеет вид:

\[
\begin{pmatrix}
0.04 & -0.01 & 0 \\
-0.01 & 0.01 & 0 \\
0 & 0 & 0.09 \\
\end{pmatrix}
\]

\begin{enumerate}
  \item Проверьте гипотезу о том, что количество часов подготовки не влияет на вероятность сдать экзамен.
  \item Посчитайте предельный эффект увеличения каждого регрессора на вероятность сдать экзамен для студента не пробовавшего пиццу и готовившегося 24 часа.
Кратко, одной-двумя фразами, прокомментируйте смысл полученных цифр.
%  \item Постройте 95\%-й доверительный интервал для разницы вероятностей сдать экзамен двумя студентами, если
%  оба студента готовились 20 часов, однако один пробовал пиццу, а второй — нет.
  \item При каком значении $D_i$ предельный эффект увеличения $X_i$ на вероятность сдать экзамен максимален,
  если $X_i=20$?
\end{enumerate}









\end{enumerate}


\end{document}
