\documentclass[12pt, a4paper]{article}\usepackage[]{graphicx}\usepackage[]{color}
%% maxwidth is the original width if it is less than linewidth
%% otherwise use linewidth (to make sure the graphics do not exceed the margin)
\makeatletter
\def\maxwidth{ %
  \ifdim\Gin@nat@width>\linewidth
    \linewidth
  \else
    \Gin@nat@width
  \fi
}
\makeatother

\definecolor{fgcolor}{rgb}{0.345, 0.345, 0.345}
\newcommand{\hlnum}[1]{\textcolor[rgb]{0.686,0.059,0.569}{#1}}%
\newcommand{\hlstr}[1]{\textcolor[rgb]{0.192,0.494,0.8}{#1}}%
\newcommand{\hlcom}[1]{\textcolor[rgb]{0.678,0.584,0.686}{\textit{#1}}}%
\newcommand{\hlopt}[1]{\textcolor[rgb]{0,0,0}{#1}}%
\newcommand{\hlstd}[1]{\textcolor[rgb]{0.345,0.345,0.345}{#1}}%
\newcommand{\hlkwa}[1]{\textcolor[rgb]{0.161,0.373,0.58}{\textbf{#1}}}%
\newcommand{\hlkwb}[1]{\textcolor[rgb]{0.69,0.353,0.396}{#1}}%
\newcommand{\hlkwc}[1]{\textcolor[rgb]{0.333,0.667,0.333}{#1}}%
\newcommand{\hlkwd}[1]{\textcolor[rgb]{0.737,0.353,0.396}{\textbf{#1}}}%
\let\hlipl\hlkwb

\usepackage{framed}
\makeatletter
\newenvironment{kframe}{%
 \def\at@end@of@kframe{}%
 \ifinner\ifhmode%
  \def\at@end@of@kframe{\end{minipage}}%
  \begin{minipage}{\columnwidth}%
 \fi\fi%
 \def\FrameCommand##1{\hskip\@totalleftmargin \hskip-\fboxsep
 \colorbox{shadecolor}{##1}\hskip-\fboxsep
     % There is no \\@totalrightmargin, so:
     \hskip-\linewidth \hskip-\@totalleftmargin \hskip\columnwidth}%
 \MakeFramed {\advance\hsize-\width
   \@totalleftmargin\z@ \linewidth\hsize
   \@setminipage}}%
 {\par\unskip\endMakeFramed%
 \at@end@of@kframe}
\makeatother

\definecolor{shadecolor}{rgb}{.97, .97, .97}
\definecolor{messagecolor}{rgb}{0, 0, 0}
\definecolor{warningcolor}{rgb}{1, 0, 1}
\definecolor{errorcolor}{rgb}{1, 0, 0}
\newenvironment{knitrout}{}{} % an empty environment to be redefined in TeX

\usepackage{alltt}

% If you can't see cyrillic letters in R-studio choose
% File-Reopen with encoding
% utf8 is the preferred encoding


\input{title_bor_utf8_knitr}

\def \useR{$[$R$]$ }

%% эконометрические сокращения
\def \hb{\hat{\beta}}
\def \b{\beta}
\def \hs{\hat{s}}
\def \hy{\hat{y}}
\def \hY{\hat{Y}}
\def \he{\hat{\varepsilon}}
\def \v1{\vec{1}}
\def \e{\varepsilon}
\def \z{z}
\def \hVar{\widehat{\Var}}
\def \hCorr{\widehat{\Corr}}
\def \hCov{\widehat{\Cov}}
\def \cN{\mathcal{N}}
\renewcommand{\P}{\mathbb{P}}

%% лаг
\renewcommand{\L}{\mathrm{L}}





\usepackage[bibencoding = auto, backend = biber,
sorting = none]{biblatex}

\addbibresource{probability_dna.bib}

\def \RR{\mathbb{R}}
\def \cN{\mathcal{N}}

\title{Заметки к семинарам по эконометрике}
\author{Винни-Пух}
\date{\today}


% делаем короче интервал в списках
\setlength{\itemsep}{0pt}
\setlength{\parskip}{0pt}
\setlength{\parsep}{0pt}


\DeclareMathOperator{\Med}{Med}


\usepackage{answers}

\newtheorem{problem}{Задача}
\numberwithin{problem}{section}

\Newassociation{sol}{solution}{solution_file}
% sol — имя окружения внутри задач
% solution — имя окружения внутри solution_file
% solution_file — имя файла в который будет идти запись решений
% можно изменить далее по ходу
\Opensolutionfile{solution_file}[all_solutions]
% в квадратных скобках фактическое имя файла
\IfFileExists{upquote.sty}{\usepackage{upquote}}{}
\begin{document}

% \maketitle % ставим сюда название, автора и время создания

\section{МНК — это\ldots}



Минитеория:

\begin{enumerate}
\item Истинная модель. Например, $y_i = \beta_1 + \beta_2 x_i + \beta_3 z_i + u_i$.
\item Формула для прогнозов. Например, $\hy_i = \hb_1 + \hb_2 x_i + \hb_3 z_i$.
\item Метод наименьших квадратов, $\sum (y_i - \hy_i)^2 \to \min$.
\end{enumerate}

Задачи:

\begin{problem}
Каждый день Маша ест конфеты и решает задачи по эконометрике. Пусть $x_i$ — количество решённых задач, а $y_i$ — количество съеденных конфет.

\begin{tabular}{cc}
\toprule
$x_i$ & $y_i$ \\
\midrule
1 & 1 \\
2 & 2 \\
2 & 4 \\
\bottomrule
\end{tabular}

\begin{enumerate}
\item Рассмотрим модель $y_i = \beta x_i + u_i$:
\begin{enumerate}
\item Найдите МНК-оценку $\hb$ для имеющихся трёх наблюдений.
\item Нарисуйте исходные точки и полученную прямую регрессии.
\item Выведите формулу для $\hb$ в общем виде для $n$ наблюдений.
\end{enumerate}

\item Рассмотрим модель $y_i = \beta_1 + \beta_2 x_i + u_i$:
\begin{enumerate}
\item Найдите МНК-оценки $\hb_1$ и $\hb_2$ для имеющихся трёх наблюдений.
\item Нарисуйте исходные точки и полученную прямую регрессии.
\item Выведите формулу для $\hb_2$ в общем виде для $n$ наблюдений.
\end{enumerate}

\end{enumerate}


\begin{sol}
\end{sol}
\end{problem}



\section{Хочу ещё задач!}




\Closesolutionfile{solution_file}
\section{Решения}
\begin{solution}{1.1}
\end{solution}


\section{Источники мудрости}
\printbibliography[heading=none]


\end{document}
