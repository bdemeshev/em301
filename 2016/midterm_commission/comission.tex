\element{comission}{ % в фигурных скобках название группы вопросов
 %\AMCnoCompleteMulti
  \begin{questionmult}{pr1} % тип вопроса (questionmult — множественный выбор) и в фигурных — номер вопроса

    В множественной регрессии с двумя регрессорами выборочные корреляции между зависимой переменной и регрессорами составили: $\widehat{\Corr}(Y, X_1) = 0.7$, $\widehat{\Corr}(Y, X_2) = 0.2$. Тогда $R^2$ будет равен

 \begin{multicols}{3} % располагаем ответы в 3 колонки
   \begin{choices} % опция [o] не рандомизирует порядок ответов
      \correctchoice{Не хватает данных для ответа}
      \wrongchoice{0.49}
      \wrongchoice{0.04}
      \wrongchoice{0.81}
      \wrongchoice{0.25}
      \wrongchoice{0.9}
   \end{choices}
  \end{multicols}
  \end{questionmult}
}



\element{comission}{ % в фигурных скобках название группы вопросов
 %\AMCnoCompleteMulti
  \begin{questionmult}{pr2} % тип вопроса (questionmult — множественный выбор) и в фигурных — номер вопроса

    Исследователь Феофан оценил регрессию $Y$ на $Z$ и получил, что $\hat{Y_i} = 10 + 2Z_i$. После этого, он оценил регрессию $Y$ на $Z$ и новую переменную $X$. Известна выборочная ковариация, $\widehat{\Cov}(Y,X) = 0$. Выберете верное утверждение:

 \begin{multicols}{2} % располагаем ответы в 3 колонки
   \begin{choices} % опция [o] не рандомизирует порядок ответов
      \correctchoice{$R^2$ не снизился по сравнению с исходной моделью}
      \wrongchoice{Коэффициент при переменной $Z$ не изменился}
      \wrongchoice{Коэффициент при переменной $X$ равен нулю}
      \wrongchoice{$R^2_{adj}$ вырос и стал равен 1}
      \wrongchoice{Коэффициент при $Z$ стал незначимым}
      \wrongchoice{Коэффициент при $Z$ остался значимым}
   \end{choices}
  \end{multicols}
  \end{questionmult}
}




\element{comission}{ % в фигурных скобках название группы вопросов
 %\AMCnoCompleteMulti
  \begin{questionmult}{pr3} % тип вопроса (questionmult — множественный выбор) и в фигурных — номер вопроса

    Исследовательница Алевтина изучает зависимость размера порции в мишленовском ресторане от его звёздности, $star$ (от 1 до 3), и уровня цен, $price$. Она оценила модель вида $size_i = \beta_1 + \beta_2 star2_i + \beta_3 star3_i + \beta_4 price_i + u_i$, где $star1, star2, star3$ - дамми-переменные, равные 1 для ресторанов с соответствующим числом звезд, и 0 иначе. Алевтина считает, что размер порции уменьшается вдвое с каждой дополнительной звездой. Какую гипотезу ей нужно проверить?

 \begin{multicols}{2} % располагаем ответы в 3 колонки
   \begin{choices} % опция [o] не рандомизирует порядок ответов
      \correctchoice{ $H_0 : \beta_1 = 2\beta_2 = 4\beta_3$ }
      \wrongchoice{ $H_0 : \beta_3 = 4, \beta_2 = 2, \beta_1 = 1$ }
      \wrongchoice{ $H_0 : 4\beta_1 = 2\beta_2 = \beta_3$ }
      \wrongchoice{ $H_0 : \beta_1 < \beta_2 < \beta_3$ }
      \wrongchoice{ $H_0 : \beta_1 = 2\beta_2 = 3\beta_3$ }
      \wrongchoice{ $H_0 : 3\beta_1 = 2\beta_2 = \beta_3$ }
   \end{choices}
  \end{multicols}
  \end{questionmult}
}




\element{comission}{ % в фигурных скобках название группы вопросов
 %\AMCnoCompleteMulti
  \begin{questionmult}{pr4} % тип вопроса (questionmult — множественный выбор) и в фигурных — номер вопроса

    Исследовательница Алевтина вновь изучает зависимость размера порции в мишленовском ресторане от его звёздности, $star$ (от 1 до 3), и уровня цен, $price$. Она оценила модель вида $size_i = \beta_1 + \beta_2 star2_i + \beta_3 star3_i + \beta_4 price_i + u_i$, где $star1, star2, star3$ - дамми-переменные, равные 1 для ресторанов с соответствующим числом звезд, и 0 иначе. У Алевтины есть $n$ наблюдений. С помощью какой статистики она будет проверять гипотезу об отсутствии влияния звёздности на размер порции?

 \begin{multicols}{3} % располагаем ответы в 3 колонки
   \begin{choices} % опция [o] не рандомизирует порядок ответов
      \correctchoice{ $F_{2, n-4}$ }
      \wrongchoice{ $t_{n-4}$ }
      \wrongchoice{ $t_{n-3}$ }
      \wrongchoice{ $F_{3, n-4}$ }
      \wrongchoice{ $F_{2, n-3}$ }
      \wrongchoice{ $F_{3, n-3}$ }
   \end{choices}
  \end{multicols}
  \end{questionmult}
}



\element{comission}{ % в фигурных скобках название группы вопросов
 %\AMCnoCompleteMulti
  \begin{questionmult}{pr5} % тип вопроса (questionmult — множественный выбор) и в фигурных — номер вопроса

    Исследователь Валериан оценил модель зависимости дохода человека, $income$, от его возраста, $age$, и получил следующую зависимость $\widehat{income}_i = 150 + 900 age_i - 10 age_i^2$. Все коэффициенты значимы на 5\% уровне значимости. В каком возрасте доход достигает максимума?

 \begin{multicols}{2} % располагаем ответы в 3 колонки
   \begin{choices} % опция [o] не рандомизирует порядок ответов
      \correctchoice{ 45 лет }
      \wrongchoice{ 90 лет }
      \wrongchoice{ 30 лет }
      \wrongchoice{ Недостаточно данных }
      \wrongchoice{ Доход возрастает с ростом возраста, у функции нет максимума }
      \wrongchoice{ Доход убывает с ростом возраста, у функции нет максимума }
   \end{choices}
  \end{multicols}
  \end{questionmult}
}



\element{comission}{ % в фигурных скобках название группы вопросов
 %\AMCnoCompleteMulti
  \begin{questionmult}{pr6} % тип вопроса (questionmult — множественный выбор) и в фигурных — номер вопроса

    В регрессии с четырьмя регрессорами, оцененной по 21 наблюдению, оказалось, что $TSS = 250$, $R^2_{adj} = 0.75$. Тогда $R^2$ равен

 \begin{multicols}{3} % располагаем ответы в 3 колонки
   \begin{choices} % опция [o] не рандомизирует порядок ответов
      \correctchoice{ 0.8 }
      \wrongchoice{ 0.2 }
      \wrongchoice{ 0.6 }
      \wrongchoice{ 0.96 }
      \wrongchoice{ 0.7 }
      \wrongchoice{ 0.3 }
   \end{choices}
  \end{multicols}
  \end{questionmult}
}



\element{comission}{ % в фигурных скобках название группы вопросов
 %\AMCnoCompleteMulti
  \begin{questionmult}{pr7} % тип вопроса (questionmult — множественный выбор) и в фигурных — номер вопроса

    Регрессия с тремя регрессорами оценена по 100 наблюдениям. Какая из этих гипотез НЕ может быть проверена при помощи статистики, имеющей $F_{2,96}$ распределение?

 \begin{multicols}{3} % располагаем ответы в 3 колонки
   \begin{choices} % опция [o] не рандомизирует порядок ответов
      \correctchoice{ $H_0: \beta_2 = 2 \beta_3$ }
      \wrongchoice{ $H_0: \beta_2 = \beta_3 = 0$ }
      \wrongchoice{ $H_0: \beta_1 = 1; \beta_3 = 5$ }
      \wrongchoice{ $H_0: \beta_2 = 3 \beta_3 = 5$ }
      \wrongchoice{ $H_0: \beta_1 + \beta_2 = 10; \beta_3 = 1$ }
      \wrongchoice{ $H_0: \beta_1 = \beta_2 = \beta_3$ }
   \end{choices}
  \end{multicols}
  \end{questionmult}
}



\element{comission}{ % в фигурных скобках название группы вопросов
 %\AMCnoCompleteMulti
  \begin{questionmult}{pr8} % тип вопроса (questionmult — множественный выбор) и в фигурных — номер вопроса

    Исследователь выполнил второй шаг в PE-тесте МакКиннона. В регрессии $\ln Y_i$ на исходные регрессоры и $Z_i = \hat Y_i - \exp(\widehat{\ln Y_i})$ коэффициент при $Z_i$ оказался значимым. А в регрессии $Y_i$ на исходные регрессоры и $W_i = \ln \hat Y_i - \widehat{\ln Y_i}$ коэффициент при $W_i$ оказался незначимым. Из результатов следует сделать вывод, что

 \begin{multicols}{2} % располагаем ответы в 3 колонки
   \begin{choices} % опция [o] не рандомизирует порядок ответов
      \correctchoice{следует предпочесть линейную модель}
      \wrongchoice{следует предпочесть логарифмическую модель}
      \wrongchoice{следует предпочесть полулогарифмическую модель}
      \wrongchoice{тесты противоречат друг другу, ни одна из моделей не предпочитается}
      \wrongchoice{в исходной модели пропущен регрессор $Z_i$}
      \wrongchoice{в исходной модели пропущен регрессор $W_i$}
   \end{choices}
  \end{multicols}
  \end{questionmult}
}



\element{comission}{ % в фигурных скобках название группы вопросов
 %\AMCnoCompleteMulti
  \begin{questionmult}{pr9} % тип вопроса (questionmult — множественный выбор) и в фигурных — номер вопроса

    Истинной является модель $Y_i = \beta_1 + \beta_2 X_i + u_i$. Глафира оценивает две регрессии: $\hat Y_i = \hb_1 + \hb_2X_i$ и  $\hat Y_i = \hat \gamma_1 + \hat \gamma_2 X_i + \hat \gamma_3 Z_i$ с помощью МНК. Известна выборочная корреляция $\widehat{\Corr}(X_i, Z_i) = -0.2$. Тогда оценка $\hat \gamma_2$ является

 \begin{multicols}{1} % располагаем ответы в 3 колонки
   \begin{choices} % опция [o] не рандомизирует порядок ответов
      \correctchoice{состоятельной, но неэффективной оценкой для $\beta_2$}
      \wrongchoice{несостоятельной и неэффективной оценкой для $\beta_2$}
      \wrongchoice{несостоятельной, но эффективной оценкой для $\beta_2$}
      \wrongchoice{состоятельной, но смещенной на $-0.2 \hat \gamma_3$ относительно $\beta_2$ оценкой}
      \wrongchoice{состоятельной, но смещенной на $-0.2$ относительно $\beta_2$ оценкой}
      \wrongchoice{смещённой на $-0.2 \hat \gamma_3$ относительно $\beta_2$, но эффективной оценкой}
   \end{choices}
  \end{multicols}
  \end{questionmult}
}



\element{comission}{ % в фигурных скобках название группы вопросов
 %\AMCnoCompleteMulti
  \begin{questionmult}{pr10} % тип вопроса (questionmult — множественный выбор) и в фигурных — номер вопроса

    Если для регрессора используется преобразование Бокса-Кокса с параметром $\theta = -1$, а для зависимой переменной — с параметром $\lambda = 0$, то регрессионное уравнение представимо в виде

 \begin{multicols}{2} % располагаем ответы в 3 колонки
   \begin{choices} % опция [o] не рандомизирует порядок ответов
      \correctchoice{ $\ln Y_i = \beta_1 + \beta_2 \frac{1}{X_i} + u_i$ }
      \wrongchoice{ $Y_i = \beta_1 + \beta_2 \frac{1}{X_i} + u_i$ }
      \wrongchoice{ $\ln Y_i = \beta_1 + \beta_2 X_i + u_i$ }
      \wrongchoice{ $Y_i = \beta_1 + \beta_2 X_i + u_i$ }
      \wrongchoice{ $\ln Y_i = \beta_1 + \beta_2 (X_i - 1) + u_i$ }
      \wrongchoice{ $\ln Y_i = \beta_1 + \beta_2 X^2_i + u_i$ }
   \end{choices}
  \end{multicols}
  \end{questionmult}
}
