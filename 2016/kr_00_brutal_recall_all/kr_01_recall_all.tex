\documentclass[12pt,a4paper]{article}

\usepackage[top=3cm, left=2cm, right=2cm]{geometry} % размер текста на странице

\usepackage{tikz} % картинки в tikz
\usepackage{microtype} % свешивание пунктуации

\usepackage{array} % для столбцов фиксированной ширины

\usepackage{indentfirst} % отступ в первом параграфе

\usepackage{sectsty} % для центрирования названий частей
\allsectionsfont{\centering}

\usepackage{amsmath} % куча стандартных математических плюшек

\usepackage{multicol} % текст в несколько колонок

\usepackage{lastpage} % чтобы узнать номер последней страницы

\usepackage{enumitem} % дополнительные плюшки для списков
%  например \begin{enumerate}[resume] позволяет продолжить нумерацию в новом списке

\usepackage{amsmath}
\usepackage{amssymb}


\usepackage{fontspec}
\usepackage{polyglossia}

\setmainlanguage{russian}
\setotherlanguages{english}

% download "Linux Libertine" fonts:
% http://www.linuxlibertine.org/index.php?id=91&L=1
\setmainfont{Linux Libertine O} % or Helvetica, Arial, Cambria
% why do we need \newfontfamily:
% http://tex.stackexchange.com/questions/91507/
\newfontfamily{\cyrillicfonttt}{Linux Libertine O}

\AddEnumerateCounter{\asbuk}{\russian@alph}{щ} % для списков с русскими буквами




% \usepackage[left=1cm,right=1cm,top=1cm,bottom=1cm]{geometry}

\usepackage{fancyhdr} % весёлые колонтитулы
\pagestyle{fancy}
\lhead{Эконометрика, праздник номер 1!}
\chead{Вспомнить всё!}
\rhead{12.09.2016}
\lfoot{}
\cfoot{}
\rfoot{\thepage/\pageref{LastPage}}
\renewcommand{\headrulewidth}{0.4pt}
\renewcommand{\footrulewidth}{0.4pt}

\DeclareMathOperator{\tr}{tr}
\DeclareMathOperator{\E}{\mathbb{E}}
\let\P\relax
\DeclareMathOperator{\P}{\mathbb{P}}
\DeclareMathOperator{\Var}{\mathbb{V}ar}
\DeclareMathOperator{\Cov}{\mathbb{C}ov}

\begin{document}


Сегодня 256-ой день года, всех с днём программиста! :) А ещё 12 сентября в 490 году до нашей эры Фидиппид добежал из Марафона в Афины с криком «Νενικήκαμεν»\footnote{Ликуйте! Мы победили!}!


\begin{enumerate}
\item Найдите длины векторов $a=(1,1,1)$ и $b=(1,2,3)$ и косинус угла между ними. Найдите один любой вектор, перпенидкулярный вектору $b$.

\item Сформулируйте теорему о трёх перпендикулярах и обратную к ней

\item На плоскости $\alpha$ лежит прямая $\ell$. Вне плоскости $\alpha$ лежит точка $C$. Ромео проецирует точку $C$ на прямую $\ell$ и получает точку $R$. Джульетта проецирует точку $C$ сначала на плоскость $\alpha$, а затем проецирует полученную точку $A$ на прямую $\ell$. После двух действий Джульетта получает точку $D$. Обязательно ли $R$ и $D$ совпадают?

\item Для матрицы

\[
A=\begin{pmatrix}
5 & 4 \\
4 & 5 \\
\end{pmatrix}
\]

\begin{enumerate}
\item Найдите собственные числа и собственные векторы матрицы
\item Найдите $\det (A)$, $\tr(A)$
\item Найдите собственные числа матрицы $A^{2016}$, $\det (A^{2016})$ и $\tr(A^{2016})$
\end{enumerate}

\item Известно, что $X$ --- матрица размера $n \times k$ и $n>k$, известно, что $X'X$ обратима. Рассмотрим матрицу $H=X(X'X)^{-1}X'$. Укажите размер матрицы $H$, найдите $H^{2016}$, $\tr(H)$, $\det(H)$, собственные числа матрицы $H$. Штрих означает транспонирование.

\item Занудная халява: известно, что $\Cov(X, Y)=5$, $\Var(X)=10$, $\Var(Y)=20$, $\E(X)=10$, $\E(Y)=-10$. Найдите $\Cov(X+2Y, Y-X)$, $\Var(X+2Y)$, $\E(X+2Y)$.

\item За 100 дней Ромео посчитал все глубокие вздохи Джульетты. Настроение Джульетты столь спонтанно, что глубокие вздохи за разные дни можно считать независимыми. В~сумме оказалось 890 вздохов. Сумма квадратов оказалась равна 8000. Постройте 95\%-ый доверительный интервал для математического ожидания ежедневного количества глубоких вздохов Джульетты. На~уровне значимости 5\%-ов проверьте гипотезу, что математическое ожидание равно~9.

\item Ромео подкидывает монетку два раза. Если монетка выпадает орлом, то Ромео кладет в мешок черный шар, если решкой --— белый. Джульетта не~знает, как выпадала монетка, и~достает шары из мешка наугад по очереди. Первый шар оказался черного цвета. Какова вероятность того, что второй шар Джульетты будет белым?

\end{enumerate}





\end{document}
