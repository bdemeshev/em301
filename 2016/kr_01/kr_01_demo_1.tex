\documentclass[12pt]{article}\usepackage[]{graphicx}\usepackage[svgnames]{xcolor}
%% maxwidth is the original width if it is less than linewidth
%% otherwise use linewidth (to make sure the graphics do not exceed the margin)
\makeatletter
\def\maxwidth{ %
  \ifdim\Gin@nat@width>\linewidth
    \linewidth
  \else
    \Gin@nat@width
  \fi
}
\makeatother

\definecolor{fgcolor}{rgb}{0.345, 0.345, 0.345}
\newcommand{\hlnum}[1]{\textcolor[rgb]{0.686,0.059,0.569}{#1}}%
\newcommand{\hlstr}[1]{\textcolor[rgb]{0.192,0.494,0.8}{#1}}%
\newcommand{\hlcom}[1]{\textcolor[rgb]{0.678,0.584,0.686}{\textit{#1}}}%
\newcommand{\hlopt}[1]{\textcolor[rgb]{0,0,0}{#1}}%
\newcommand{\hlstd}[1]{\textcolor[rgb]{0.345,0.345,0.345}{#1}}%
\newcommand{\hlkwa}[1]{\textcolor[rgb]{0.161,0.373,0.58}{\textbf{#1}}}%
\newcommand{\hlkwb}[1]{\textcolor[rgb]{0.69,0.353,0.396}{#1}}%
\newcommand{\hlkwc}[1]{\textcolor[rgb]{0.333,0.667,0.333}{#1}}%
\newcommand{\hlkwd}[1]{\textcolor[rgb]{0.737,0.353,0.396}{\textbf{#1}}}%
\let\hlipl\hlkwb

\usepackage{framed}
\makeatletter
\newenvironment{kframe}{%
 \def\at@end@of@kframe{}%
 \ifinner\ifhmode%
  \def\at@end@of@kframe{\end{minipage}}%
  \begin{minipage}{\columnwidth}%
 \fi\fi%
 \def\FrameCommand##1{\hskip\@totalleftmargin \hskip-\fboxsep
 \colorbox{shadecolor}{##1}\hskip-\fboxsep
     % There is no \\@totalrightmargin, so:
     \hskip-\linewidth \hskip-\@totalleftmargin \hskip\columnwidth}%
 \MakeFramed {\advance\hsize-\width
   \@totalleftmargin\z@ \linewidth\hsize
   \@setminipage}}%
 {\par\unskip\endMakeFramed%
 \at@end@of@kframe}
\makeatother

\definecolor{shadecolor}{rgb}{.97, .97, .97}
\definecolor{messagecolor}{rgb}{0, 0, 0}
\definecolor{warningcolor}{rgb}{1, 0, 1}
\definecolor{errorcolor}{rgb}{1, 0, 0}
\newenvironment{knitrout}{}{} % an empty environment to be redefined in TeX

\usepackage{alltt}

\usepackage[top=3cm, left=2cm, right=2cm]{geometry} % размер текста на странице

\usepackage[box, % запрет на перенос вопросов
%nopage,
insidebox, % ставим буквы в квадратики
separateanswersheet, % добавляем бланк ответов
nowatermark, % отсутствие надписи "Черновик"
indivanswers,  % показываем верные ответы
%answers,
lang=RU,
nopage, % убираем оформление страницы (идентификаторы для распознавания)
completemulti]{automultiplechoice}

\usepackage{tikz} % картинки в tikz
\usepackage{microtype} % свешивание пунктуации

\usepackage{array} % для столбцов фиксированной ширины

\usepackage{indentfirst} % отступ в первом параграфе

\usepackage{sectsty} % для центрирования названий частей
\allsectionsfont{\centering}

\usepackage{amsmath} % куча стандартных математических плюшек

\usepackage{multicol} % текст в несколько колонок

\usepackage{lastpage} % чтобы узнать номер последней страницы

\usepackage{enumitem} % дополнительные плюшки для списков
%  например \begin{enumerate}[resume] позволяет продолжить нумерацию в новом списке










\usepackage{fancyhdr} % весёлые колонтитулы
\pagestyle{fancy}
\lhead{Эконометрика, контрольная 1}
\chead{}
\rhead{xx.11.2016, демо-вариант 1}
\lfoot{}
\cfoot{}
\rfoot{\thepage/\pageref{LastPage}}
\renewcommand{\headrulewidth}{0.4pt}
\renewcommand{\footrulewidth}{0.4pt}



\usepackage{todonotes} % для вставки в документ заметок о том, что осталось сделать
% \todo{Здесь надо коэффициенты исправить}
% \missingfigure{Здесь будет Последний день Помпеи}
% \listoftodos --- печатает все поставленные \todo'шки


% более красивые таблицы
\usepackage{booktabs}
% заповеди из докупентации:
% 1. Не используйте вертикальные линни
% 2. Не используйте двойные линии
% 3. Единицы измерения - в шапку таблицы
% 4. Не сокращайте .1 вместо 0.1
% 5. Повторяющееся значение повторяйте, а не говорите "то же"



\usepackage{fontspec}
\usepackage{polyglossia}

\setmainlanguage{russian}
\setotherlanguages{english}

% download "Linux Libertine" fonts:
% http://www.linuxlibertine.org/index.php?id=91&L=1
\setmainfont{Linux Libertine O} % or Helvetica, Arial, Cambria
% why do we need \newfontfamily:
% http://tex.stackexchange.com/questions/91507/
\newfontfamily{\cyrillicfonttt}{Linux Libertine O}

\AddEnumerateCounter{\asbuk}{\russian@alph}{щ} % для списков с русскими буквами


%% эконометрические сокращения
\DeclareMathOperator{\plim}{plim}
\DeclareMathOperator{\Cov}{Cov}
\DeclareMathOperator{\Corr}{Corr}
\DeclareMathOperator{\Var}{Var}
\DeclareMathOperator{\E}{E}
\def \hb{\hat{\beta}}
\def \hs{\hat{\sigma}}
\def \htheta{\hat{\theta}}
\def \s{\sigma}
\def \hy{\hat{y}}
\def \hY{\hat{Y}}
\def \v1{\vec{1}}
\def \e{\varepsilon}
\def \he{\hat{\e}}
\def \z{z}
\def \hVar{\widehat{\Var}}
\def \hCorr{\widehat{\Corr}}
\def \hCov{\widehat{\Cov}}
\def \cN{\mathcal{N}}


\AddEnumerateCounter{\asbuk}{\russian@alph}{щ} % для списков с русскими буквами
\setlist[enumerate, 2]{label=\asbuk*),ref=\asbuk*}
\IfFileExists{upquote.sty}{\usepackage{upquote}}{}
\begin{document}

\element{demo_15}{ % в фигурных скобках название группы вопросов
 \AMCnoCompleteMulti
  \begin{questionmult}{1} % тип вопроса (questionmult --- множественный выбор) и в фигурных --- номер вопроса
  При добавлении новой переменной скорректированный $R^2$
% \begin{multicols}{3} % располагаем ответы в 3 колонки
   \begin{choices} % опция [o] не рандомизирует порядок ответов
      \wrongchoice{обязательно вырастет}
      \wrongchoice{обязательно упадёт}
       \correctchoice{может как вырасти, так и упасть}
      \end{choices}
%  \end{multicols}
  \end{questionmult}
}


\element{demo_15}{ % в фигурных скобках название группы вопросов
 \AMCnoCompleteMulti
  \begin{questionmult}{2} % тип вопроса (questionmult --- множественный выбор) и в фигурных --- номер вопроса
  При добавлении новой переменной коэффициент детерминации $R^2$:
% \begin{multicols}{3} % располагаем ответы в 3 колонки
   \begin{choices} % опция [o] не рандомизирует порядок ответов
      \correctchoice{обязательно вырастет}
      \wrongchoice{обязательно упадёт}
      \wrongchoice{может как вырасти, так и упасть}
      \end{choices}
%  \end{multicols}
  \end{questionmult}
}


\element{demo_15}{ % в фигурных скобках название группы вопросов
 \AMCnoCompleteMulti
  \begin{questionmult}{3} % тип вопроса (questionmult --- множественный выбор) и в фигурных --- номер вопроса
  Для проверки гипотезы о значимости коэффициентов при мультиколлинеарности стандартные $t$-статистики
% \begin{multicols}{3} % располагаем ответы в 3 колонки
   \begin{choices} % опция [o] не рандомизирует порядок ответов
      \correctchoice{можно использовать, т.к. они по прежнему имеют $t$-распределение}
      \wrongchoice{нельзя использовать т.к. они не имеют $t$-распределения}
      \end{choices}
%  \end{multicols}
  \end{questionmult}
}

\element{demo_15}{ % в фигурных скобках название группы вопросов
 \AMCnoCompleteMulti
  \begin{questionmult}{4} % тип вопроса (questionmult --- множественный выбор) и в фигурных --- номер вопроса
  При условной гетероскедастичности и наблюдениях, представляющих случайную выборку, оценки МНК
 \begin{multicols}{2} % располагаем ответы в 3 колонки
   \begin{choices} % опция [o] не рандомизирует порядок ответов
      \correctchoice{остаются состоятельными}
      \wrongchoice{перестают быть состоятельными}
      \end{choices}
  \end{multicols}
  \end{questionmult}
}


\element{demo_15}{ % в фигурных скобках название группы вопросов
 \AMCnoCompleteMulti
  \begin{questionmult}{5} % тип вопроса (questionmult --- множественный выбор) и в фигурных --- номер вопроса
  При условной гетероскедастичности и наблюдениях, представляющих случайную выборку, оценки МНК
 \begin{multicols}{2} % располагаем ответы в 3 колонки
   \begin{choices} % опция [o] не рандомизирует порядок ответов
      \correctchoice{остаются несмещёнными}
      \wrongchoice{перестают быть несмещёнными}
      \end{choices}
  \end{multicols}
  \end{questionmult}
}


\element{demo_15}{ % в фигурных скобках название группы вопросов
 \AMCnoCompleteMulti
  \begin{questionmult}{6} % тип вопроса (questionmult --- множественный выбор) и в фигурных --- номер вопроса
  При предпосылке о нормально распределенных ошибках в классической линейной регрессионной модели оценки коэффициентов уравнения с помощью МНК и оценки с помощью максимального правдоподобия
% \begin{multicols}{3} % располагаем ответы в 3 колонки
   \begin{choices} % опция [o] не рандомизирует порядок ответов
      \correctchoice{совпадают}
      \wrongchoice{отличаются}
      \end{choices}
%  \end{multicols}
  \end{questionmult}
}

\element{demo_15}{ % в фигурных скобках название группы вопросов
 \AMCcompleteMulti
  \begin{questionmult}{7} % тип вопроса (questionmult --- множественный выбор) и в фигурных --- номер вопроса
  При условной гетероскедастичности использование робастных стандартных ошибок позволяет
% \begin{multicols}{3} % располагаем ответы в 3 колонки
   \begin{choices} % опция [o] не рандомизирует порядок ответов
      \wrongchoice{устранить смещённость оценок коэффициентов}
      \wrongchoice{устранить несостоятельность оценок коэффициентов}
      \end{choices}
%  \end{multicols}
  \end{questionmult}
}

\element{demo_15}{ % в фигурных скобках название группы вопросов
 \AMCcompleteMulti
  \begin{questionmult}{8} % тип вопроса (questionmult --- множественный выбор) и в фигурных --- номер вопроса
  При автокорреляции первого порядка в ошибках использование робастных стандартных ошибок Нью-Веста позволяет
% \begin{multicols}{3} % располагаем ответы в 3 колонки
   \begin{choices} % опция [o] не рандомизирует порядок ответов
     \wrongchoice{устранить смещённость оценок коэффициентов}
     \wrongchoice{устранить несостоятельность оценок коэффициентов}
      \end{choices}
%  \end{multicols}
  \end{questionmult}
}

\element{demo_15}{ % в фигурных скобках название группы вопросов
 \AMCnoCompleteMulti
  \begin{questionmult}{9} % тип вопроса (questionmult --- множественный выбор) и в фигурных --- номер вопроса
  Если нарушена только предпосылка $\E(u_i) = 0$, то при оценке модели $y_i = \beta_1 + \beta_2 x_i + u_i$ оценка $\hat \beta_2$ окажется
 \begin{multicols}{2} % располагаем ответы в 3 колонки
   \begin{choices} % опция [o] не рандомизирует порядок ответов
      \correctchoice{несмещённой}
      \wrongchoice{смещённой}
      \end{choices}
 \end{multicols}
  \end{questionmult}
}


\element{demo_15}{ % в фигурных скобках название группы вопросов
 \AMCnoCompleteMulti
  \begin{questionmult}{10} % тип вопроса (questionmult --- множественный выбор) и в фигурных --- номер вопроса
  Если все выборочные корреляции между регрессорами по модулю меньше 0.1 то строгая мультиколлинеарность
 \begin{multicols}{2} % располагаем ответы в 3 колонки
   \begin{choices} % опция [o] не рандомизирует порядок ответов
      \correctchoice{возможна}
      \wrongchoice{невозможна}
      \end{choices}
 \end{multicols}
  \end{questionmult}
}

\element{demo_16}{ % в фигурных скобках название группы вопросов
 \AMCnoCompleteMulti
  \begin{questionmult}{1} % тип вопроса (questionmult --- множественный выбор) и в фигурных --- номер вопроса
  Мультиколлинеарность приводит к смещению оценок коэффициентов регрессии.
 \begin{multicols}{1} % располагаем ответы в 3 колонки
   \begin{choices}[o] % опция [o] не рандомизирует порядок ответов
      \wrongchoice{верно}
      \correctchoice{не верно}
      \end{choices}
  \end{multicols}
  \end{questionmult}
}

\element{demo_16}{ % в фигурных скобках название группы вопросов
 \AMCnoCompleteMulti
  \begin{questionmult}{2} % тип вопроса (questionmult --- множественный выбор) и в фигурных --- номер вопроса
Мультиколлинеарность приводит к смещению оценок дисперсий коэффициентов регрессии.
 \begin{multicols}{1} % располагаем ответы в 3 колонки
   \begin{choices}[o] % опция [o] не рандомизирует порядок ответов
      \wrongchoice{верно}
      \correctchoice{не верно}
      \end{choices}
  \end{multicols}
  \end{questionmult}
}

\element{demo_16}{ % в фигурных скобках название группы вопросов
 \AMCnoCompleteMulti
  \begin{questionmult}{3} % тип вопроса (questionmult --- множественный выбор) и в фигурных --- номер вопроса
  Мультиколлинеарность приводит к высокой дисперсии оценок коэффициентов.
 \begin{multicols}{1} % располагаем ответы в 3 колонки
   \begin{choices}[o] % опция [o] не рандомизирует порядок ответов
      \correctchoice{верно}
      \wrongchoice{не верно}
      \end{choices}
  \end{multicols}
  \end{questionmult}
}

\element{demo_16}{ % в фигурных скобках название группы вопросов
 \AMCnoCompleteMulti
  \begin{questionmult}{4} % тип вопроса (questionmult --- множественный выбор) и в фигурных --- номер вопроса
Для устранения мультиколлинеарности применяется обобщенный метод наименьших квадратов.
 \begin{multicols}{1} % располагаем ответы в 3 колонки
   \begin{choices}[o] % опция [o] не рандомизирует порядок ответов
      \wrongchoice{верно}
      \correctchoice{не верно}
      \end{choices}
  \end{multicols}
  \end{questionmult}
}

\element{demo_16}{ % в фигурных скобках название группы вопросов
 \AMCnoCompleteMulti
  \begin{questionmult}{5} % тип вопроса (questionmult --- множественный выбор) и в фигурных --- номер вопроса
Признаком мультиколлинеарности является значимость модели в целом при незначимости отдельных коэффициентов.  
 \begin{multicols}{1} % располагаем ответы в 3 колонки
   \begin{choices}[o] % опция [o] не рандомизирует порядок ответов
      \correctchoice{верно}
      \wrongchoice{не верно}
      \end{choices}
  \end{multicols}
  \end{questionmult}
}

\element{demo_16}{ % в фигурных скобках название группы вопросов
 \AMCnoCompleteMulti
  \begin{questionmult}{6} % тип вопроса (questionmult --- множественный выбор) и в фигурных --- номер вопроса
  В случае гетероскедастичности применение стандартных ошибок в форме Уайта помогает сделать оценки коэффициентов эффективными.
 \begin{multicols}{1} % располагаем ответы в 3 колонки
   \begin{choices}[o] % опция [o] не рандомизирует порядок ответов
      \wrongchoice{верно}
      \correctchoice{не верно}
      \end{choices}
  \end{multicols}
  \end{questionmult}
}

\element{demo_16}{ % в фигурных скобках название группы вопросов
 \AMCnoCompleteMulti
  \begin{questionmult}{7} % тип вопроса (questionmult --- множественный выбор) и в фигурных --- номер вопроса
  Тест Дарбина-Уотсона применим тольков случае автокорреляции первого порядка.
 \begin{multicols}{1} % располагаем ответы в 3 колонки
   \begin{choices}[o] % опция [o] не рандомизирует порядок ответов
      \correctchoice{верно}
      \wrongchoice{не верно}
      \end{choices}
  \end{multicols}
  \end{questionmult}
}

\element{demo_16}{ % в фигурных скобках название группы вопросов
 \AMCnoCompleteMulti
  \begin{questionmult}{8} % тип вопроса (questionmult --- множественный выбор) и в фигурных --- номер вопроса
  Нулевая гипотеза в тесте Дарбина-Уотсона -- наличие автокорреляции.
 \begin{multicols}{1} % располагаем ответы в 3 колонки
   \begin{choices}[o] % опция [o] не рандомизирует порядок ответов
      \wrongchoice{верно}
      \correctchoice{не верно}
      \end{choices}
  \end{multicols}
  \end{questionmult}
}

\element{demo_16}{ % в фигурных скобках название группы вопросов
 \AMCnoCompleteMulti
  \begin{questionmult}{9} % тип вопроса (questionmult --- множественный выбор) и в фигурных --- номер вопроса
  Если регрессор коррелирован с ошибкой модели, то оценки коэффициентов становятся несостоятельными.
 \begin{multicols}{1} % располагаем ответы в 3 колонки
   \begin{choices}[o] % опция [o] не рандомизирует порядок ответов
      \wrongchoice{верно}
      \correctchoice{не верно}
      \end{choices}
  \end{multicols}
  \end{questionmult}
}

\element{demo_16}{ % в фигурных скобках название группы вопросов
 \AMCnoCompleteMulti
  \begin{questionmult}{10} % тип вопроса (questionmult --- множественный выбор) и в фигурных --- номер вопроса
  В случае автокорреляции оценки дисперсий коэффициентов оказываются смещенными.
 \begin{multicols}{1} % располагаем ответы в 3 колонки
   \begin{choices}[o] % опция [o] не рандомизирует порядок ответов
      \wrongchoice{верно}
      \correctchoice{не верно}
      \end{choices}
  \end{multicols}
  \end{questionmult}
}














\element{combat_15}{ % в фигурных скобках название группы вопросов
 \AMCcompleteMulti
  \begin{questionmult}{1} % тип вопроса (questionmult --- множественный выбор) и в фигурных --- номер вопроса
  Предпосылка об отсутствии систематической ошибки в модели означает, что для всех наблюдений
 \begin{multicols}{2} % располагаем ответы в 3 колонки
   \begin{choices} % опция [o] не рандомизирует порядок ответов
      \wrongchoice{$\Var(\varepsilon_i)=0$}
      \wrongchoice{$\Var(\varepsilon_i) \neq 0$}
      \end{choices}
  \end{multicols}
  \end{questionmult}
}

\element{combat_15}{ % в фигурных скобках название группы вопросов
 \AMCnoCompleteMulti
  \begin{questionmult}{2} % тип вопроса (questionmult --- множественный выбор) и в фигурных --- номер вопроса
  Стандартные ошибки в форме Уайта в случае гетероскедастичности помогают устранить несостоятельность оценок коэффициентов
 \begin{multicols}{2} % располагаем ответы в 3 колонки
   \begin{choices} % опция [o] не рандомизирует порядок ответов
      \correctchoice{неверно}
      \wrongchoice{верно}
      \end{choices}
  \end{multicols}
  \end{questionmult}
}

\element{combat_15}{ % в фигурных скобках название группы вопросов
 \AMCnoCompleteMulti
  \begin{questionmult}{3} % тип вопроса (questionmult --- множественный выбор) и в фигурных --- номер вопроса
  Незначимость всех коэффициентов регрессии
 \begin{multicols}{2} % располагаем ответы в 3 колонки
   \begin{choices} % опция [o] не рандомизирует порядок ответов
      \correctchoice{может быть не связана с мультиколлинеарностью}
      \wrongchoice{обязательно свидетельствует о наличии мультиколлинеарности}
      \end{choices}
  \end{multicols}
  \end{questionmult}
}


\element{combat_15}{ % в фигурных скобках название группы вопросов
 \AMCnoCompleteMulti
  \begin{questionmult}{4} % тип вопроса (questionmult --- множественный выбор) и в фигурных --- номер вопроса
  После применения МНК к модели $y_i=\beta x_i + \varepsilon_i$   сумма $ESS+RSS$
 \begin{multicols}{2} % располагаем ответы в 3 колонки
   \begin{choices} % опция [o] не рандомизирует порядок ответов
      \correctchoice{может быть не равна $TSS$}
      \wrongchoice{обязательно равна $TSS$}
      \end{choices}
  \end{multicols}
  \end{questionmult}
}

\element{combat_15}{ % в фигурных скобках название группы вопросов
 \AMCnoCompleteMulti
  \begin{questionmult}{5} % тип вопроса (questionmult --- множественный выбор) и в фигурных --- номер вопроса
  При наличии ошибок измерения зависимой переменной МНК-оценки коэффициентов модели
 \begin{multicols}{2} % располагаем ответы в 3 колонки
   \begin{choices} % опция [o] не рандомизирует порядок ответов
      \correctchoice{состоятельны}
      \wrongchoice{несостоятельны}
      \end{choices}
  \end{multicols}
  \end{questionmult}
}

\element{combat_15}{ % в фигурных скобках название группы вопросов
 \AMCnoCompleteMulti
  \begin{questionmult}{6} % тип вопроса (questionmult --- множественный выбор) и в фигурных --- номер вопроса
  Если выполнены все предпосылки теоремы Гаусса-Маркова, но остатки модели не подчиняются нормальному закону распределения, то МНК-оценки коэффициентов регрессии являются
 \begin{multicols}{2} % располагаем ответы в 3 колонки
   \begin{choices} % опция [o] не рандомизирует порядок ответов
      \correctchoice{несмещёнными}
      \wrongchoice{смещёнными}
      \end{choices}
  \end{multicols}
  \end{questionmult}
}


\element{combat_15}{ % в фигурных скобках название группы вопросов
 \AMCnoCompleteMulti
  \begin{questionmult}{7} % тип вопроса (questionmult --- множественный выбор) и в фигурных --- номер вопроса
  Индексы вздутия дисперсии (VIF) в случае отсутствия мультиколлинеарности лежат в интервале
 \begin{multicols}{2} % располагаем ответы в 3 колонки
   \begin{choices} % опция [o] не рандомизирует порядок ответов
      \correctchoice{$[1;+\infty)$}
      \wrongchoice{$[0;1]$}
      \end{choices}
  \end{multicols}
  \end{questionmult}
}


\element{combat_15}{ % в фигурных скобках название группы вопросов
 \AMCnoCompleteMulti
  \begin{questionmult}{8} % тип вопроса (questionmult --- множественный выбор) и в фигурных --- номер вопроса
  Нулевая гипотеза в тесте Дарбина-Уотсона состоит в
 \begin{multicols}{2} % располагаем ответы в 3 колонки
   \begin{choices} % опция [o] не рандомизирует порядок ответов
      \correctchoice{отсутствии автокорреляции}
      \wrongchoice{наличии автокорреляции}
      \end{choices}
  \end{multicols}
  \end{questionmult}
}

\element{combat_15}{ % в фигурных скобках название группы вопросов
 \AMCnoCompleteMulti
  \begin{questionmult}{9} % тип вопроса (questionmult --- множественный выбор) и в фигурных --- номер вопроса
  Если в модель добавили незначимый фактор, то коэффициент детерминации $R^2$
 \begin{multicols}{3} % располагаем ответы в 3 колонки
   \begin{choices} % опция [o] не рандомизирует порядок ответов
      \correctchoice{вырастет}
      \wrongchoice{упадёт}
      \wrongchoice{не изменится}
      \end{choices}
  \end{multicols}
  \end{questionmult}
}

\element{combat_15}{ % в фигурных скобках название группы вопросов
 \AMCnoCompleteMulti
  \begin{questionmult}{10} % тип вопроса (questionmult --- множественный выбор) и в фигурных --- номер вопроса
  При диагностике автокорреляции третьего порядка тест Бройша-Годфри
 \begin{multicols}{3} % располагаем ответы в 3 колонки
   \begin{choices} % опция [o] не рандомизирует порядок ответов
      \correctchoice{применим}
      \wrongchoice{неприменим}
      \end{choices}
  \end{multicols}
  \end{questionmult}
}


\element{combat_15_v2}{ % в фигурных скобках название группы вопросов
 \AMCnoCompleteMulti
  \begin{questionmult}{11} % тип вопроса (questionmult --- множественный выбор) и в фигурных --- номер вопроса
  Нормальность остатков является одной из предпосылок теоремы Гаусса-Маркова
 \begin{multicols}{3} % располагаем ответы в 3 колонки
   \begin{choices} % опция [o] не рандомизирует порядок ответов
      \correctchoice{неверно}
      \wrongchoice{верно}
      \end{choices}
  \end{multicols}
  \end{questionmult}
}


\element{combat_15_v2}{ % в фигурных скобках название группы вопросов
 \AMCcompleteMulti
  \begin{questionmult}{12} % тип вопроса (questionmult --- множественный выбор) и в фигурных --- номер вопроса
  В случае гетероскедастичности применение стандартных ошибок в форме Уайта позволяет сделать оценки коэффициентов регрессии
 \begin{multicols}{3} % располагаем ответы в 3 колонки
   \begin{choices} % опция [o] не рандомизирует порядок ответов
      \wrongchoice{несмещёнными}
      \wrongchoice{состоятельными}
      \wrongchoice{эффективными}
      \end{choices}
  \end{multicols}
  \end{questionmult}
}

\element{combat_15_v2}{ % в фигурных скобках название группы вопросов
 \AMCnoCompleteMulti
  \begin{questionmult}{13} % тип вопроса (questionmult --- множественный выбор) и в фигурных --- номер вопроса
  С помощью МНК оценивается модель $y_i = \beta_1 + \beta_2 x_i + \varepsilon_i$. Наблюдения представляют собой случайную выборку, и $\Cov(\varepsilon_i, x_i) = 1$. В этом случае $\plim \hat \beta_2^{ols}$
 \begin{multicols}{3} % располагаем ответы в 3 колонки
   \begin{choices} % опция [o] не рандомизирует порядок ответов
      \correctchoice{не равен $\beta_2$}
      \wrongchoice{равен $\beta_2$}
      \end{choices}
  \end{multicols}
  \end{questionmult}
}


\element{combat_15_v2}{ % в фигурных скобках название группы вопросов
 \AMCnoCompleteMulti
  \begin{questionmult}{14} % тип вопроса (questionmult --- множественный выбор) и в фигурных --- номер вопроса
  После применения МНК к модели $y_i=\beta x_i+\varepsilon_i$ сумма остатков $\sum \hat \varepsilon_i$
 \begin{multicols}{3} % располагаем ответы в 3 колонки
   \begin{choices} % опция [o] не рандомизирует порядок ответов
      \correctchoice{не равна нулю}
      \wrongchoice{равна нулю}
      \end{choices}
  \end{multicols}
  \end{questionmult}
}

\element{combat_15_v2}{ % в фигурных скобках название группы вопросов
 \AMCnoCompleteMulti
  \begin{questionmult}{15} % тип вопроса (questionmult --- множественный выбор) и в фигурных --- номер вопроса
  В результате применения МНК к модели $y_i=\beta_1 + \beta_2 x_i+\varepsilon_i$ сумма $\sum x_i \hat \varepsilon_i$
 \begin{multicols}{3} % располагаем ответы в 3 колонки
   \begin{choices} % опция [o] не рандомизирует порядок ответов
      \correctchoice{обязательно равна нулю}
      \wrongchoice{может быть не равна нулю}
      \end{choices}
  \end{multicols}
  \end{questionmult}
}

\element{combat_15_v2}{ % в фигурных скобках название группы вопросов
 \AMCnoCompleteMulti
  \begin{questionmult}{16} % тип вопроса (questionmult --- множественный выбор) и в фигурных --- номер вопроса
  В случае мультиколлинеарности применение гребневой регрессии (ridge-regression) делает оценки коэффициентов
 \begin{multicols}{3} % располагаем ответы в 3 колонки
   \begin{choices} % опция [o] не рандомизирует порядок ответов
      \correctchoice{смещёнными}
      \wrongchoice{несмещёнными}
      \end{choices}
  \end{multicols}
  \end{questionmult}
}

\element{combat_15_v2}{ % в фигурных скобках название группы вопросов
 \AMCnoCompleteMulti
  \begin{questionmult}{17} % тип вопроса (questionmult --- множественный выбор) и в фигурных --- номер вопроса
  В случае мультиколлинеарности оценки дисперсий коэффициентов модели становятся
 \begin{multicols}{3} % располагаем ответы в 3 колонки
   \begin{choices} % опция [o] не рандомизирует порядок ответов
      \correctchoice{несмещёнными}
      \wrongchoice{смещёнными}
      \end{choices}
  \end{multicols}
  \end{questionmult}
}

\element{combat_15_v2}{ % в фигурных скобках название группы вопросов
 \AMCcompleteMulti
  \begin{questionmult}{18} % тип вопроса (questionmult --- множественный выбор) и в фигурных --- номер вопроса
  После применения МНК к исходной модели дополнительно можно оценить модель $\ln(\hat \varepsilon_i^2 )=\gamma_1 + \gamma_2 \ln(x_i) + u_i$  для диагностики
 \begin{multicols}{2} % располагаем ответы в 3 колонки
   \begin{choices} % опция [o] не рандомизирует порядок ответов
      \correctchoice{гетероскедастичности}
      \wrongchoice{автокорреляции}
      \wrongchoice{мультиколлинеарности}
      \end{choices}
  \end{multicols}
  \end{questionmult}
}

\element{combat_15_v2}{ % в фигурных скобках название группы вопросов
 \AMCcompleteMulti
  \begin{questionmult}{19} % тип вопроса (questionmult --- множественный выбор) и в фигурных --- номер вопроса
  Для сравнения качества моделей $y_i=\beta_1+ \beta_2 x_i+\varepsilon_i$ и $\ln(y_i)=\gamma_1+ \gamma_2 x_i + \varepsilon_i$, оцененных на одном наборе данных, используют
 \begin{multicols}{2} % располагаем ответы в 3 колонки
   \begin{choices} % опция [o] не рандомизирует порядок ответов
      \wrongchoice{коэффициент детерминации $R^2$}
      \wrongchoice{скорректированный коэффициент $R_{adj}^2$}
      \end{choices}
  \end{multicols}
  \end{questionmult}
}


\element{combat_15_v2}{ % в фигурных скобках название группы вопросов
 \AMCnoCompleteMulti
  \begin{questionmult}{20} % тип вопроса (questionmult --- множественный выбор) и в фигурных --- номер вопроса
  При диагностике автокорреляции первого порядка тест Бройша-Годфри
 \begin{multicols}{2} % располагаем ответы в 3 колонки
   \begin{choices} % опция [o] не рандомизирует порядок ответов
      \correctchoice{применим}
      \wrongchoice{неприменим}
      \end{choices}
  \end{multicols}
  \end{questionmult}
}




















\element{combat_15_v3}{ % в фигурных скобках название группы вопросов
 \AMCnoCompleteMulti
  \begin{questionmult}{1} % тип вопроса (questionmult --- множественный выбор) и в фигурных --- номер вопроса
  Ошибки измерения независимой переменной являются одной из причин
 \begin{multicols}{2} % располагаем ответы в 3 колонки
   \begin{choices} % опция [o] не рандомизирует порядок ответов
      \correctchoice{эндогенности}
      \wrongchoice{мультиколлинеарности}
      \wrongchoice{автокорреляции}
      \end{choices}
  \end{multicols}
  \end{questionmult}
}

\element{combat_15_v3}{ % в фигурных скобках название группы вопросов
 \AMCnoCompleteMulti
  \begin{questionmult}{2} % тип вопроса (questionmult --- множественный выбор) и в фигурных --- номер вопроса
  Во временных рядах гетероскедастичность
 \begin{multicols}{2} % располагаем ответы в 3 колонки
   \begin{choices} % опция [o] не рандомизирует порядок ответов
      \correctchoice{возможна}
      \wrongchoice{невозможна}
      \end{choices}
  \end{multicols}
  \end{questionmult}
}

\element{combat_15_v3}{ % в фигурных скобках название группы вопросов
 \AMCnoCompleteMulti
  \begin{questionmult}{3} % тип вопроса (questionmult --- множественный выбор) и в фигурных --- номер вопроса
  Оценка $\hb$ называется несмещённой, если с ростом числа наблюдений она стремится к истинной $\beta$
 \begin{multicols}{2} % располагаем ответы в 3 колонки
   \begin{choices} % опция [o] не рандомизирует порядок ответов
      \correctchoice{неверно}
      \wrongchoice{верно}
      \end{choices}
  \end{multicols}
  \end{questionmult}
}


\element{combat_15_v3}{ % в фигурных скобках название группы вопросов
 \AMCnoCompleteMulti
  \begin{questionmult}{4} % тип вопроса (questionmult --- множественный выбор) и в фигурных --- номер вопроса
  С ростом числа наблюдений распределение статистики Дарбина-Уотсона стремится к
 \begin{multicols}{2} % располагаем ответы в 3 колонки
   \begin{choices} % опция [o] не рандомизирует порядок ответов
      \correctchoice{особому распределению Дарбина-Уотсона}
      \wrongchoice{стандартному нормальному распределению}
      \end{choices}
  \end{multicols}
  \end{questionmult}
}

\element{combat_15_v3}{ % в фигурных скобках название группы вопросов
 \AMCcompleteMulti
  \begin{questionmult}{5} % тип вопроса (questionmult --- множественный выбор) и в фигурных --- номер вопроса
  Одним из способов борьбы с нестрогой мультиколлинеарностью является
 %\begin{multicols}{2} % располагаем ответы в 3 колонки
   \begin{choices} % опция [o] не рандомизирует порядок ответов
      \correctchoice{увеличение количества наблюдений}
      \wrongchoice{деление всех регрессоров на одно и то же большое число}
      \wrongchoice{использование взвешенного МНК}
    \end{choices}
 %\end{multicols}
  \end{questionmult}
}

\element{combat_15_v3}{ % в фигурных скобках название группы вопросов
 \AMCnoCompleteMulti
  \begin{questionmult}{6} % тип вопроса (questionmult --- множественный выбор) и в фигурных --- номер вопроса
 Нестрогая мультиколлинеарность нарушает теорему Гаусса-Маркова
 \begin{multicols}{2} % располагаем ответы в 3 колонки
   \begin{choices} % опция [o] не рандомизирует порядок ответов
      \correctchoice{неверно}
      \wrongchoice{верно}
      \end{choices}
  \end{multicols}
  \end{questionmult}
}


\element{combat_15_v3}{ % в фигурных скобках название группы вопросов
 \AMCnoCompleteMulti
  \begin{questionmult}{7} % тип вопроса (questionmult --- множественный выбор) и в фигурных --- номер вопроса
  Если ошибки распределены не нормально, то МНК-оценки коэффициентов регрессии
 \begin{multicols}{2} % располагаем ответы в 3 колонки
   \begin{choices} % опция [o] не рандомизирует порядок ответов
      \correctchoice{эффективны и несмещены}
      \wrongchoice{неэффективны и смещены}
      \wrongchoice{эффективны и смещены}
      \wrongchoice{неэффективны и несмещены}
    \end{choices}
  \end{multicols}
  \end{questionmult}
}


\element{combat_15_v3}{ % в фигурных скобках название группы вопросов
 \AMCnoCompleteMulti
  \begin{questionmult}{8} % тип вопроса (questionmult --- множественный выбор) и в фигурных --- номер вопроса
 Если в модели присутствуют лаги независимой переменной, то тест Дарбина-Уотсона
 \begin{multicols}{2} % располагаем ответы в 3 колонки
   \begin{choices} % опция [o] не рандомизирует порядок ответов
      \correctchoice{применим}
      \wrongchoice{неприменим}
      \end{choices}
  \end{multicols}
  \end{questionmult}
}

\element{combat_15_v3}{ % в фигурных скобках название группы вопросов
 \AMCnoCompleteMulti
  \begin{questionmult}{9} % тип вопроса (questionmult --- множественный выбор) и в фигурных --- номер вопроса
  При гетероскедастичности оценки коэффициентов
 \begin{multicols}{2} % располагаем ответы в 3 колонки
   \begin{choices} % опция [o] не рандомизирует порядок ответов
      \correctchoice{остаются несмещёнными}
      \wrongchoice{в среднем завышены}
      \wrongchoice{в среднем занижены}
      \end{choices}
  \end{multicols}
  \end{questionmult}
}

\element{combat_15_v3}{ % в фигурных скобках название группы вопросов
 \AMCnoCompleteMulti
  \begin{questionmult}{10} % тип вопроса (questionmult --- множественный выбор) и в фигурных --- номер вопроса
  У разностного  уравнения $y_t=0.1y_{t-1} + \e_t +0.4\e_{t-1}$
 \begin{multicols}{3} % располагаем ответы в 3 колонки
   \begin{choices} % опция [o] не рандомизирует порядок ответов
      \correctchoice{есть единственное стационарное решение}
      \wrongchoice{нет стационарных решений}
      \wrongchoice{есть бесконечное количество стационарных решений}
    \end{choices}
  \end{multicols}
  \end{questionmult}
}























\element{combat_15_v4}{ % в фигурных скобках название группы вопросов
 \AMCcompleteMulti
  \begin{questionmult}{1} % тип вопроса (questionmult --- множественный выбор) и в фигурных --- номер вопроса
  В теореме Гаусса-Маркова предполагается, что ошибки
 \begin{multicols}{2} % располагаем ответы в 3 колонки
   \begin{choices} % опция [o] не рандомизирует порядок ответов
      \wrongchoice{имеют нулевое математическое ожидание и единичную дисперсию}
      \wrongchoice{имеют ненулевое математическое ожидание и неединичную дисперсию}
    \end{choices}
  \end{multicols}
  \end{questionmult}
}

\element{combat_15_v4}{ % в фигурных скобках название группы вопросов
 \AMCnoCompleteMulti
  \begin{questionmult}{2} % тип вопроса (questionmult --- множественный выбор) и в фигурных --- номер вопроса
  При условной гетероскедастичности оценки коэффициентов
 \begin{multicols}{2} % располагаем ответы в 3 колонки
   \begin{choices} % опция [o] не рандомизирует порядок ответов
      \correctchoice{состоятельны}
      \wrongchoice{несостоятельны}
      \end{choices}
  \end{multicols}
  \end{questionmult}
}

\element{combat_15_v4}{ % в фигурных скобках название группы вопросов
 \AMCcompleteMulti
  \begin{questionmult}{3} % тип вопроса (questionmult --- множественный выбор) и в фигурных --- номер вопроса
  Двухшаговый метод наименьших квадратов --- это стандартный способ борьбы с
 \begin{multicols}{2} % располагаем ответы в 3 колонки
   \begin{choices} % опция [o] не рандомизирует порядок ответов
      \correctchoice{эндогенностью}
      \wrongchoice{автокорреляцией}
      \wrongchoice{гетероскедастичностью}
      \end{choices}
  \end{multicols}
  \end{questionmult}
}


\element{combat_15_v4}{ % в фигурных скобках название группы вопросов
 \AMCnoCompleteMulti
  \begin{questionmult}{4} % тип вопроса (questionmult --- множественный выбор) и в фигурных --- номер вопроса
  После применения МНК к модели $y_i=\beta x_i + \varepsilon_i$   сумма $\sum \he_i^2$
 \begin{multicols}{2} % располагаем ответы в 3 колонки
   \begin{choices} % опция [o] не рандомизирует порядок ответов
      \correctchoice{равна нулю}
      \wrongchoice{не обязательно равна нулю}
      \end{choices}
  \end{multicols}
  \end{questionmult}
}

\element{combat_15_v4}{ % в фигурных скобках название группы вопросов
 \AMCnoCompleteMulti
  \begin{questionmult}{5} % тип вопроса (questionmult --- множественный выбор) и в фигурных --- номер вопроса
  При автокорреляции МНК-оценки коэффициентов являются
 \begin{multicols}{2} % располагаем ответы в 3 колонки
   \begin{choices} % опция [o] не рандомизирует порядок ответов
      \correctchoice{несмещёнными}
      \wrongchoice{смещёнными}
      \end{choices}
  \end{multicols}
  \end{questionmult}
}

\element{combat_15_v4}{ % в фигурных скобках название группы вопросов
 \AMCnoCompleteMulti
  \begin{questionmult}{6} % тип вопроса (questionmult --- множественный выбор) и в фигурных --- номер вопроса
  Нестрогая мультиколлинеарность --- это одно из нарушений предпосылок теоремы Гаусса-Маркова
 \begin{multicols}{2} % располагаем ответы в 3 колонки
   \begin{choices} % опция [o] не рандомизирует порядок ответов
      \correctchoice{неверно}
      \wrongchoice{верно}
      \end{choices}
  \end{multicols}
  \end{questionmult}
}


\element{combat_15_v4}{ % в фигурных скобках название группы вопросов
 \AMCnoCompleteMulti
  \begin{questionmult}{7} % тип вопроса (questionmult --- множественный выбор) и в фигурных --- номер вопроса
  В случае мультиколлинеарности оценки дисперсий коэффициентов являются
 \begin{multicols}{3} % располагаем ответы в 3 колонки
   \begin{choices} % опция [o] не рандомизирует порядок ответов
      \correctchoice{несмещёнными}
      \wrongchoice{в среднем завышенными}
      \wrongchoice{в среднем заниженными}
      \end{choices}
  \end{multicols}
  \end{questionmult}
}


\element{combat_15_v4}{ % в фигурных скобках название группы вопросов
 \AMCnoCompleteMulti
  \begin{questionmult}{8} % тип вопроса (questionmult --- множественный выбор) и в фигурных --- номер вопроса
  Тест Дарбина-Уотсона в регрессии без свободного члена
 \begin{multicols}{2} % располагаем ответы в 3 колонки
   \begin{choices} % опция [o] не рандомизирует порядок ответов
      \correctchoice{неприменим}
      \wrongchoice{применим}
      \end{choices}
  \end{multicols}
  \end{questionmult}
}

\element{combat_15_v4}{ % в фигурных скобках название группы вопросов
 \AMCnoCompleteMulti
  \begin{questionmult}{9} % тип вопроса (questionmult --- множественный выбор) и в фигурных --- номер вопроса
  Нулевой гипотезой в тесте Уайта является гипотеза о
 \begin{multicols}{2} % располагаем ответы в 3 колонки
   \begin{choices} % опция [o] не рандомизирует порядок ответов
      \correctchoice{гомоскедастичности}
      \wrongchoice{гетероскедастичности}
      \wrongchoice{наличии автокорреляции}
      \wrongchoice{отсутствии автокорреляции}
    \end{choices}
  \end{multicols}
  \end{questionmult}
}

\element{combat_15_v4}{ % в фигурных скобках название группы вопросов
 \AMCnoCompleteMulti
  \begin{questionmult}{10} % тип вопроса (questionmult --- множественный выбор) и в фигурных --- номер вопроса
   Величина $R^2_{adj}$ показывает, какую долю дисперсии зависимой переменной объясняют использованные регрессоры
 \begin{multicols}{3} % располагаем ответы в 3 колонки
   \begin{choices} % опция [o] не рандомизирует порядок ответов
      \correctchoice{неверно}
      \wrongchoice{верно}
      \end{choices}
  \end{multicols}
  \end{questionmult}
}



\section*{Часть 1. Тест.}

\onecopy{1}{

\cleargroup{demo_a}
\copygroup[10]{demo_15}{demo_a}
\shufflegroup{demo_a}
\insertgroup{demo_a}

}

\section*{Часть 2. Задачи.}


\begin{enumerate}


\item В течение 30 дней Василий записывал количество пойманных им покемонов, $y_i$, и количество решённых задач по эконометрике, $x_i$. Оказалось, что $\sum x_i^2 = ...$, $\sum y_i^2 = ...$, $\sum x_i = ... $, $\sum y_i = ...$ и $\sum x_i y_i = ...$. Василий предполагает корректность линейной модели $y_i = \beta_1 + \beta_2 x_i + \e_i$.
\begin{enumerate}
\item Найдите МНК-оценки коэффициентов регресси
\item Найдите $RSS$, $ESS$, $TSS$ и $R^2$
\item Предполагая нормальность остатков проверьте значимость каждого коэффициента на уровне значимости 5\%.
\end{enumerate}

\item Регрессионная модель  задана в матричном виде при помощи уравнения $y=X\beta+\varepsilon$, где $\beta=(\beta_1,\beta_2,\beta_3)'$.
Известно, что $\E(\varepsilon)=0$  и  $\Var(\varepsilon)=\sigma^2\cdot I$.
Известно также, что

$y=\left(
\begin{array}{c}
1\\
2\\
3\\
4\\
5
\end{array}\right)$,
$X=\left(\begin{array}{ccc}
1 & 0 & 0 \\
1 & 0 & 0 \\
1 & 1 & 0 \\
1 & 1 & 0 \\
1 & 1 & 1
\end{array}\right)$.


Для удобства расчетов приведены матрицы


$X'X=\left(
\begin{array}{ccc}
5 & 3 & 1\\
3 & 3 & 1\\
1 & 1 & 1
\end{array}\right)$ и $(X'X)^{-1}=\frac{1}{2}\left(
\begin{array}{ccc}
1 & -1 & 0 \\
-1 & 2 & -1 \\
0 & -1 & 3
\end{array}\right)$.

\begin{enumerate}
\item Найдите вектор МНК-оценок коэффициентов $\hb$.
\item Найдите несмещенную оценку для неизвестного параметра $\sigma^2$.
\end{enumerate}




\item Рассмотрим модель парной регрессии.
\begin{enumerate}
\item Сформулируйте теорему Гаусса-Маркова для случая парной регрессии
\item Поясните смысл понятий несмещённости, линейности и эффективности оценок
\item Приведите пример ситуации, в которой МНК-оценки не существуют
\end{enumerate}


\end{enumerate}

\end{document}
