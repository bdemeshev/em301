\documentclass[pdftex,12pt,a4paper]{article}

\input{/home/boris/science/tex_general/title_bor_utf8}

\title{С 1 апреля!}
\date{}

\begin{document}
\maketitle
\parindent=0 pt % отступ равен 0



\begin{enumerate}

\item Рождается старичком, умирает младенцем, сегодня празднует день рождения, но не Гоголь. 
\begin{enumerate}
\item Кто это?
\item Опишите внешний вид, характер, или нарисуйте его :)
\end{enumerate}

\item Для борьбы с гетероскедастичностью в модели $y_i=\beta_1+\beta_2 x_i+\varepsilon_i$ исследователь перешёл к модели $\tilde{y}_i=\beta_1 \frac{1}{z_i}+\beta_2 \tilde{x}_i+\tilde{\varepsilon}_i$, где $\tilde{x}_i=x_i/z_i$, $\tilde{y}_i=y_i/z_i$, $\tilde{\varepsilon}_i=\varepsilon_i/z_i$. 

Какой вид гетероскедастичности предполагался?

\item Василий Аспушкин провёл два разных теста на гетероскедастичность на одном уровне значимости. Оказалось, что в одном из них $H_0$ отвергается, а в другом --- нет. 
\begin{enumerate}
\item Почему это могло случиться?
\item Какой же вывод о гетероскедастичности следует сделать Василию? Что можно сказать об уровне значимости предложенного Вами способа сделать вывод?
 
\end{enumerate}


\item Писатель Василий Аспушкин пишет Большой Роман. Количество страниц, которое он пишет ежедневно, зависит от количества съеденных пирожков, выпитого лимонада и числа посещений Музы. 
\[
Stranitsi_i = \beta_1 + \beta_2 Pirojki_i + \beta_3 Limonad_i + \beta_4 Musa_i + \varepsilon_i
\]

Когда идёт дождь, Василий Аспушкин очень волнуется: он ошибочно считает, что музы плохо летают в дождь. Поэтому в дождливые дни дисперсия $\varepsilon_i$ может быть выше. 


\begin{enumerate}
\item Отсортировав имеющиеся наблюдения по количеству осадков в день, Настойчивый издатель построил регрессию по 40 самым дождливым дням и получил $RSS=\sum_i (y_i-\hat{y}_i)^2=360$. В регрессии по 40 самым сухим дням $RSS=252$. Всего имеется 100 наблюдений. Проверьте гипотезу о гомоскедастичности. Как называется соответствующий тест?

\item Василий Аспушкин оценил по 100 наблюдениям исходную модель с помощью МНК. А затем построил регрессию квадрата стьюдентизированных остатков на количество осадков и константу. Во второй регрессии $R^2=0.3$. Проверьте гипотезу о гомоскедастичности. 

\item Предположим, что дисперсия ошибок линейно зависит от количества осадков. 
\begin{enumerate}
\item Как будет выглядеть функция максимального правдоподобия для оценивания коэффициентов исходной модели?
\item Опишите процедуру доступного обобщенного метода наименьших квадратов (FGLS, feasible generalized least squares) применительно к данной ситуации
\end{enumerate}
\end{enumerate}
Hint: Функция плотности одномерного нормального распределения имеет вид 
\[
f(x)=\frac{1}{\sqrt{2\pi}\sigma}\exp\left(-\frac{(x-\mu)^2}{2\sigma^2}  \right)
\]
% многомерное
%f(x)=(2\pi)^{-n/2} \det(\Omega)^{-1/2} \exp\left(-\frac{1}{2}(x-\mu)'\Omega^{-1}(x-\mu)\right)


\item В курсе теории вероятностей изучался тест о равенстве математических ожиданий по двум нормальным выборкам при предпосылке о равенстве дисперсий. Предложите состоятельный способ тестировать гипотезу о равенстве математических ожиданий без предпосылки равенства дисперсий.


\end{enumerate}




\end{document}